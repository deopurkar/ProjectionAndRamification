\documentclass[11pt,reqno]{amsart}
\usepackage[margin = 1.3 in]{geometry}
%\usepackage[frenchmath,defaultmathsizes]{mathastext}
\usepackage{
  hyperref,
  amsmath,
  amssymb,
  amsthm,
  thmtools,
  microtype,
  mathrsfs,
  enumitem,
  stmaryrd,
  diagbox,
}
\usepackage[draft]{showlabels}
\usepackage[table]{xcolor}
\usepackage{tikz, tikz-cd}

\setlength{\parskip}{.25em}

\newcommand*\justify{%
  \fontdimen2\font=0.4em% interword space
  \fontdimen3\font=0.2em% interword stretch
  \fontdimen4\font=0.1em% interword shrink
  \fontdimen7\font=0.1em% extra space
  \hyphenchar\font=`\-% allowing hyphenation
}


\usepackage{graphicx}

\linespread{1.15}

\usepackage{eucal}

%\usepackage[all]{xy}

%\usepackage[draft]{showlabels}

\theoremstyle{plain}
\newtheorem{theorem}{Theorem}[section]
\newtheorem{proposition}[theorem]{Proposition}
\newtheorem{lemma}[theorem]{Lemma}
\newtheorem{conjecture}[theorem]{Conjecture} 
\newtheorem{corollary}[theorem]{Corollary}
\theoremstyle{definition}
\newtheorem{definition}[theorem]{Definition}
\newtheorem{observation}{Observation}
\newtheorem{example}[theorem]{Example}
\newtheorem{exercise}[theorem]{Exercise}
\newtheorem{counterexample}[theorem]{Counterexample}
\newtheorem{convention}[theorem]{Convention}
\newtheorem{question}[theorem]{Question}
\theoremstyle{remark}
\newtheorem{notation}[theorem]{Notation}
\numberwithin{equation}{section}



\title{Projection and Ramification}
\author{Anand Deopurkar, Eduard Duryev, \& Anand Patel}



%-----------------------------------------------
\def\labelitemi{--}
\newcommand{\todo}[1]{\fbox{ToDo: #1}}
\renewcommand{\k}{k}
\DeclareMathOperator{\id}{id}
\DeclareMathOperator{\Bl}{Bl}
\DeclareMathOperator{\Br}{Br}
\DeclareMathOperator{\ProjBun}{ProjBun}
\renewcommand{\Vec}{\operatorname{Vec}}
\DeclareMathOperator{\Def}{Def}
\DeclareMathOperator{\res}{Res}
\DeclareMathOperator{\Quot}{Quot}
\DeclareMathOperator{\Hilb}{Hilb}
\DeclareMathOperator{\sing}{Sing}
\DeclareMathOperator{\dm}{dim}
\DeclareMathOperator{\F}{\mathbf F}

\newcommand{\cO}{{\mathcal O}}
\renewcommand{\to}{{\longrightarrow}}

\renewcommand{\sectionautorefname}{\S}
\renewcommand{\subsectionautorefname}{\S}
% Let us keep this minimial
% Let us also define things only if they are previously undefined.

% Common theorem-like environments
\ifcsname theorem\endcsname{}\else\declaretheorem[parent=section]{theorem}\fi
\ifcsname corollary\endcsname{}\else\declaretheorem[sibling=theorem]{corollary}\fi
\ifcsname lemma\endcsname{}\else\declaretheorem[sibling=theorem]{lemma}\fi
\ifcsname proposition\endcsname{}\else\declaretheorem[sibling=theorem]{proposition}\fi
\ifcsname conjecture\endcsname{}\else\declaretheorem[sibling=theorem]{conjecture}\fi
\ifcsname problem\endcsname{}\else\declaretheorem[sibling=theorem]{problem}\fi
\ifcsname question\endcsname{}\else\declaretheorem[sibling=theorem]{question}\fi
\ifcsname definition\endcsname{}\else\declaretheorem[sibling=theorem, style=definition]{definition}\fi
\ifcsname exercise\endcsname{}\else\declaretheorem[sibling=theorem, style=definition]{exercise}\fi
\ifcsname example\endcsname{}\else\declaretheorem[sibling=theorem, style=definition]{example}\fi
\ifcsname remark\endcsname{}\declaretheorem[sibling=theorem, style=remark]{remark}\fi

% Common abbreviations

% Absolutely standard rings and fields
\providecommand {\N}{{\bf N}}
\providecommand {\Z}{{\bf Z}}
\providecommand {\Q}{{\bf Q}}
\providecommand {\R}{{\bf R}}
\providecommand {\C}{{\bf C}}

% Common spaces grassmannian
\renewcommand {\P}{{\bf P}}
\providecommand {\Gr}{{\bf Gr}}
\providecommand {\A}{{\bf A}}

% Groups
\providecommand{\SL}{\operatorname{SL}}
\providecommand{\GL}{\operatorname{GL}}
\providecommand{\PGL}{\operatorname{PGL}}
\providecommand{\Gm}{{\bf G}_m}

% f \from G \to H reads much better than f \colon G \to H
\providecommand {\from}{{\colon}}

% Absolutely standard notation
\providecommand{\spec}{\operatorname{Spec}}
\providecommand{\proj}{\operatorname{Proj}}
\providecommand{\coker}{\operatorname{coker}}
% Kernel is already defined
\providecommand{\Blowup}{\operatorname{Bl}}
\providecommand{\Hom}{\operatorname{Hom}}
\providecommand{\Ext}{\operatorname{Ext}}
\providecommand{\Tor}{\operatorname{Tor}}
\providecommand{\End}{\operatorname{End}}
\providecommand{\Aut}{\operatorname{Aut}}
\providecommand{\codim}{\operatorname{codim}}
% Dim is already defined
\providecommand{\Pic}{\operatorname{Pic}}
\providecommand{\Sym}{\operatorname{Sym}}
\providecommand{\rk}{\operatorname{rk}}
\declaretheorem[sibling=theorem,style=remark]{remark}
\numberwithin{equation}{section}
\declaretheorem[title=Theorem, style=plain]{maintheorem}
\declaretheorem[title=Theorem-Example, style=plain]{exampletheorem}
\renewcommand{\themaintheorem}{\Alph{maintheorem}}
\renewcommand{\theexampletheorem}{\Alph{exampletheorem}}

\renewcommand{\O}{\mathcal O}
\newcommand{\G}{\mathbf G}
\newcommand{\td}{\widetilde}
\newcommand{\Frac}{{\mathrm{Frac}}\,}
\newcommand{\Jac}{{\textrm{Jac}}}
\DeclareMathOperator{\Ram}{Ram}
\newcommand{\fm}{\mathfrak m}
\newcommand{\smvee}{\raise0.5ex\hbox{$\scriptscriptstyle\vee$}}
\DeclareMathOperator{\ord}{ord}

\newcommand{\compl}[1]{\widehat{#1}}
\newcommand{\Spec}{{\text{\rm Spec}\,}}
\newcommand{\Spf}{{\text{\rm Spf}\,}}
\renewcommand {\o}[1]{\overline{#1}}
\newcommand{\Proj}{{\text{\rm Proj}\,}}
\newcommand{\git}{\sslash}
% -----------------------------------------------

%%% BEGIN DOCUMENT

\begin{document}

\begin{abstract}
    When a projective variety is linearly projected to a projective space of the same dimension, a ramification divisor forms. We study basic properties of this projection-ramification assignment, and uncover enumerative phenomena extending the classical appearance of Catalan numbers in the geometry of rational normal curves.
\end{abstract}


\maketitle
\tableofcontents


\section{Introduction}\label{sec:intro}
Let $X \subset \P^n$ be a smooth projective variety of dimension $r$, not contained in any hyperplane.
Projection from a general $(n-r-1)$-dimensional linear subspace $L \subset \P^n$ defines a finite surjective map
\begin{align*}
  p_{L}\from X \to \P^{r}.
\end{align*}
Associated to $p_L$ is its ramification divisor $R_L \subset X$.
A simple Riemann--Hurwitz calculation shows that $R_L$ lies in the linear series $|K_X + (r+1)H|$, where $K_X$ is the canonical class, and $H$ is the hyperplane class on $X$.
The association $L \mapsto R_L$ defines a rational map
\[ \rho_X \from \Gr(n-r, n+1) \dashrightarrow |K_X + (r+1)H|, \]
which we call the \emph{projection-ramification} map.
The goal of this paper is to explore the relationship between the geometry of $X$ and the properties of $\rho_X$.

A simple argument shows that $\rho_X$ is itself a linear projection of $\Gr(n-r, n+1)$ in its Pl\"ucker embedding.
When $X$ is a smooth curve over a field of characteristic $0$, the map $\rho_X$ is regular everywhere on $\Gr(n-r, n+1)$.
When $X$ is a rational normal curve, the map $\rho_X$ is also finite.
In this case, the ramification divisor of a map $\P^1 \to \P^1$ of degree $n$ represented by the rational function $f/g$ is cut out by the \emph{Wronskian} expression, namely the degree $(2n-2)$ polynomial $f'g - g'f$.
Since $\rho_X$ is regular, its degree is the degree of the Grassmannian, which in this case is the Catalan number $\frac{(2n-2)!}{n!(n-1)!}$.
When $X$ has dimension 2 or more, $\rho_X$ may not be regular on the entire Grassmannian, which makes it difficult to understand.
Nevertheless, it appears that the geometry of $\rho_X$ is related to some fascinating areas of classical projective geometry, and the enumerative questions surrounding $\rho_X$ hint at a rich underlying structure. 

\subsection{Maximal variation}
Our focus is the following question.
\begin{question}\label{q:maxvar}
  Is $\rho_{X}$ generically finite onto its image? In other words, does the image of $\rho_X$ have maximal possible dimension?
\end{question}

To our knowledge, this question first appeared in the work of Flenner and Manaresi \cite{MANAFlenn:}.
Our first result answers this question affirmatively for a large class of varieties.
We say that $X \subset \P^n$ is \emph{incompressible} if for every $(n-r-1)$-dimensional linear subspace $L \subset \P^n$, the projection map $p_L \from X \dashrightarrow \P^r$ is dominant.
Recall that the dual variety $X^* \subset {\P^n}^*$ is the closure of the locus of hyperplanes in $\P^n$ whose intersection with the smooth part of $X$ is singular.
\begin{maintheorem}\label{thm:main}
  Let $X \subset \P^n$ be a non-degenerate, normal, projective variety over a field of characteristic zero.
  Suppose at least one of the following holds:
  \begin{enumerate}
  \item\label{item:incomp} $X$ is incompressible, 
  \item\label{item:dual} the dual variety $X^* \subset {\P^n}^*$ is a hypersurface.
  \end{enumerate}
  Then $\rho_{X}$ is generically finite onto its image.
\end{maintheorem} 
We do not assume that $X$ is smooth in the statement of \autoref{thm:main}.
This requires defining $\rho_X$ more carefully.
To state the conclusion informally, if we move a generic $L \subset \P^n$ of complementary dimension, then the ramification locus $R_L \subset X$ also moves.

The hypotheses in \autoref{thm:main} are sufficient, but not necessary.
Indeed, consider $X = \P^{r-1} \times \P^1 \subset \P^{2r-1}$, embedded by the Segre embedding, for $r \geq 3$.
Then $X$ is neither incompressible nor is $X^*$ a hypersurface, and yet $\rho_X$ is dominant (see \autoref{Thm:Examples}).

To our knowledge, the known results about maximal variation operate under condition \eqref{item:incomp} in \autoref{thm:main}.
For example, in \cite{MANAFlenn:}, the authors deduce maximal variation under the condition that for every $(n-r-1)$-dimensional linear subspace $L \subset \P^n$, the join $J(L, X)$ equals $\P^n$, or under the condition that $X$ is smooth and the twisted normal bundle $N_{X/\P^n}(-1)$ is ample.
Either condition implies that $X$ is incompressible, and hence falls under condition \eqref{item:incomp}.
If $X$ is a curve or a smooth complete intersection, then $X$ is incompressible, and covered by condition \eqref{item:incomp}.

\autoref{thm:main} substantially increases the class of varieties where we now know maximal variation.
For example, it is easy to see that if $X$ is a smooth surface over a field of characteristic $0$, then $X^*$ is a hypersurface.
Therefore, maximal variation holds for all surfaces, although incompressibility may not (The cubic surface scroll $X \subset \P^4$ is the smallest counter example, as the projection from the directrix line of $X$ is not dominant).
As another source of new examples, take a sufficiently high degree Veronese re-embedding $X \subset \P^N$ of any smooth $X$.
Then $X^*$ is divisorial, and hence $X$ is covered under \autoref{thm:main}.
But $X \subset \P^N$ will be compressible.

Given that maximal variation holds in such a large class of varieties, it is natural to wonder if it always holds.
This is not the case.
\begin{maintheorem}
  \label{Thm:Counterexamples}
  There exist smooth, non-degenerate, rational normal scrolls $X^{r} \subset \P^{n}$ of every dimension $r \geq 4$ and degree $d \geq r+1$ such that the projection-ramification map $\rho_{X}$ is not generically finite onto its image.
\end{maintheorem}
\autoref{Thm:Counterexamples} provides the first known examples of varieties with non-maximal variation of ramification divisors.
We describe the rational normal scrolls in \autoref{Thm:Counterexamples} in \autoref{sec:proof_of_second_result}; they include some of general moduli.

We now turn our attention to cases where the projection-ramification map $\rho_X$ may be dominant.
The next result classifies $X \subset \P^n$ for which the source and the target of $\rho_X$ are of the same dimension.
\begin{maintheorem}\label{theorem:minimaldegree}
  Let $X \subset \P^{n}$ be a smooth, non-degenerate projective variety of dimension $r$ over a field of characteristic zero.
  We have the inequality
  \[ \dim \Gr(n-r, n+1) \leq \dim |K_X + (r+1)H|,\]
  where equality holds if and only if $X$ is a variety of minimal degree, that is $\deg X = n-r+1$.
\end{maintheorem}
Recall the list of smooth varieties of minimal degree: rational normal curves, quadric hypersurfaces, the Veronese surface in $\P^5$, and rational normal scrolls.
By \autoref{thm:main}, $\rho_X$ is dominant for the first three, so we are led to investigate the scrolls.
It came to us as a surprise that $\rho_X$ is \emph{not} dominant for all scrolls (see \autoref{Thm:Counterexamples}).
Nevertheless, it is dominant for most scrolls, which we now make precise.

Recall that if $X \subset \P^n$ is a smooth rational normal scroll, then $X$ is isomorphic to the projectivization of an ample vector bundle $E$ on $\P^1$, and the embedding is given by the complete linear series $|\O_{\P E}(1)|$.
\begin{maintheorem}
  \label{thm:rationalnormalscrolls}
  Let $X = \P E \subset \P^n$ be a rational normal scroll, where $E$ is a ample vector bundle of rank $r$ on $\P^1$, general in its moduli.
  If $\deg E = a \cdot (r-1) + b \cdot (2r-1) + 1$ for non-negative integers $a, b$, then the projection-ramification map $\rho_X$ is dominant for $X$.
  In particular, the conclusion holds if $E$ is general of degree at least $(r-1)(2r-1) + 1$.
\end{maintheorem}
Thus, at least among the general scrolls, the projection-ramification map is dominant except possibly in small degrees.
We prove \autoref{thm:rationalnormalscrolls} by degeneration, using the theory of limit linear series of higher rank developed by Teixidor i Bigas \cite{tei:} and Osserman \cite{oss:14}.
In the course of proving the theorem, we also demonstrate dominance in the case of $E = \O(1)^{r}$ and $E = \O(2)^r$.

\subsection{Enumerative problems}
\autoref{theorem:minimaldegree} and \autoref{thm:rationalnormalscrolls} motivate a gamut of enumerative questions.
\begin{question}\label{q:degree}
 When $X \subset \P^n$ is a variety of minimal degree, what is the degree of $\rho_X$?
\end{question}


The following result summarizes our knowledge of the answers to \autoref{q:degree}.
\begin{maintheorem}\label{Thm:Examples}\mbox{}
\begin{enumerate}
  \item If $X \subset \P^{n}$ is a rational normal curve, then $\rho_X$ is regular and $\deg \rho_{X} = \frac{(2n-2)!}{n!(n-1)!}$.
  \item  If $X \subset \P^{n}$ is a quadric hypersurface, then $\rho_{X}$ is an isomorphism.

  \item  If  $X = \P^{r-1} \times \P^{1} \hookrightarrow \P^{2r-1}$ is the Segre embedding, then $ \rho_{X}$ is birational.

  \item  If $X \subset \P^{5}$ is the Veronese surface, then $ \deg \rho_{X} = 3$.
  \item If $X \subset \P^{5}$ is a general quartic surface scroll, then $\deg \rho_{X} = 2$.
  \item If $X = \P(\O_{\P^{1}}(1) \oplus \O_{\P^{1}}(k+1)) \subset \P^{k+3}$ is the surface scroll with most imbalanced splitting type, then $\rho_{X}$ is birational.
  \item If $X = \P(\O_{\P^{1}}(1) \oplus \O_{\P^{1}}(1) \oplus \O_{\P^{1}}(k+1)) \subset \P^{k+5}$ is the threefold scroll with most imbalanced splitting type, then $\rho_{X}$ is birational.
\end{enumerate} 
\end{maintheorem}

For $X$ of dimension $1$, namely a rational normal curve, the projection-ramification map
\[ \rho_X \from \Gr(2, n+1) \to \P^{2n-2}\]
is regular, and defined by the Pl\"ucker line bundle on the Grassmannian.
Therefore, its degree is the top self-intersection of the Pl\"ucker line bundle, which in this case is the Catalan number $\frac{(2n-2)!}{n!(n-1)!}$.

For $X$ of codimension $1$, namely a quadric hypersurface, the projection-ramification map
\[ \rho_X \from \Gr(n, n+1) = \P^{n} \to {\P^n}^* \]
is again regular, and is in fact the duality isomorphism induced by the (non-degenerate) quadric $X$.

The case of the Veronese surface and of the quartic surface scroll in \autoref{Thm:Examples} are particularly delightful; these are treated in \autoref{sec:enumerativeproblems}.
They involve intricate classical projective geometry that intertwines cubic plane curves, Steinerians and Cayleyans, and apolarity.

The cases of the most unbalanced surface and threefold scrolls follow from direct calculation.
Note, however, that for the most unbalanced scroll in dimension 4 and higher, the projection-ramification map is not dominant.
For scrolls, $\rho_X$ is not regular.
Furthermore, the complexity of the base locus of $\rho_X$ effectively blocks any straightforward application of the excess intersection formula.

A smooth rational normal scroll $X \subset \P^n$ of degree $d$ and dimension $r$ is isomorphic to the projectivization of an ample vector bundle $E$ on $\P^1$, which in turn is isomorphic to a direct sum $\O(a_1) \oplus \dots \oplus \O(a_r)$ for positive integers $a_1, \dots, a_r$ satisfying $d = a_1 + \dots + a_r$.
Let $\Sigma_{r,d}$ be the set of $r$-term partitions of $d$.
We get a function $\rho \from \Sigma_{r,d} \to \Z_{\geq 0}$ defined by
\[ \rho(a_1, \dots, a_r) = \deg \rho_X,\]
for $X = \P (\O(a_1) \oplus \dots \oplus \O(a_r))$.
The set $\Sigma_{r,d}$ has a partial ordering $\prec$ given by dominance.
If $(a_1, \dots, a_r) \prec (b_1, \dots, b_r)$, then the scroll $\P(\O(b_1) \oplus \dots \oplus \O(b_r))$ isotrivially specializes to the scroll $\P(\O(a_1) \oplus \dots \oplus \O(a_r))$.
By the lower semi-continuity of degrees of rational maps, we get
\[ \rho(a_1, \dots, a_r) \leq \rho(b_1,\dots, b_r).\]
\autoref{thm:rationalnormalscrolls} implies that, at least if $d$ is sufficiently large compared to $r$, then $\rho$ is not identically zero.
\autoref{Thm:Examples} determines the value of $\rho$ for the partitions $(n)$, $(1, \dots, 1)$, $(1,k+1)$, $(1,1,k+1)$, and $(2,2)$.
The following table lists some more values of $\rho$ computed using randomized calculations over finite fields using the computer algebra systems \texttt{Macaulay2} and \texttt{MAGMA}.
We plan to return to a more complete enumerative investigation of $\rho$ in a future paper.
\begin{table}
  \centering
  \rowcolors{2}{gray!10}{white}

  \begin{tabular}{l| r r r r}
    \rowcolor{gray!25}
    \diagbox{$a_1$}{$a_2$} & 1 & 2 & 3 & 4\\
    \hline
    1 & 1 & & &\\
    2 & 1 & 2 & &\\
    3 & 1 & 6 & 22 &\\
    4 & 1 & 17 & 92 & 422\\
  \end{tabular}
  
  \caption{Degree of $\rho_X$ for $X = \P(\O(a_1) \oplus \O(a_2))$} \label{tab:computation}
\end{table}


\subsection{Further remarks and questions}
One of the  central enumerative problems concerning branch divisors, originating in the work of Hurwitz, is to compute the number of branched covers of the projective line with specified branch set in $\P^1$.
This number is called the Hurwitz number.
As is well known, the Hurwitz numbers are difficult to compute, but they exhibit remarkable structure.
There is a related question of computing the number of rational functions on $\P^1$ with a prescribed ramification set.
This question is much more elementary, and yields the Catalan numbers, as we have seen.

In higher dimensions, however, the analogue of the Hurwitz problem is expected to be much less interesting, thanks to Chisini's conjecture (now Kulikov's theorem \cite{1064-5632-63-6-A03}).
Kulikov's theorem asserts that a branched cover $S \to \P^2$ with generic branching is uniquely determined by its branch divisor $B \subset \P^{2}$, with finitely many well-understood exceptions.
In contrast, the enumerative problem regarding ramification divisors persists in all dimensions, thanks to \autoref{theorem:minimaldegree}, and poses a significant challenge.
In some sense, the ``branch'' and ``ramification'' enumerative stories trade places, at least in terms of difficulty, but perhaps also in terms of structure.

The projection-ramification map generalizes the Wronski map
\[ \rho \from \Gr(2, n+1) \to \P^{2n-2}.\]
The geometry surrounding the Wronski map has received a lot of attention, thanks to the B. and M. Shapiro conjecture.
This conjecture states that the pre-image of any point in $\P^{2n-2}$ defined by a set of $(2n-2)$ real points on $\P^1$ consists entirely of real points in $\Gr(2, n+1)$ \cite{sottile2000} (the conjecture has been proved by Eremenko and Gabrielov \cite{Erem/Gabr1}).
\autoref{theorem:minimaldegree} potentially sets the stage for a higher-dimensional generalization of the body of work around the Shapiro conjecture.

The study of $\rho_X$ in positive characteristic is likely to bring new surprises and require different techniques.
We do not know if \autoref{thm:main} or \autoref{theorem:minimaldegree} holds in positive characteristic; our proof certainly does not.
The answers to the enumerative questions \autoref{q:degree} do depend on the characteristic, even in the simplest case of rational normal curves, due to the presence of inseparable covers \cite{MR2218904}.

\subsection{Notation and conventions} We work over an algebraically closed field
$k$ of characteristic $0$ (We use Bertini's theorem and generic smoothness. We
also appeal to the Kodaira Vanishing theorem.) By a {\sl proper variety}, we mean a proper, integral, finite-type $k$-scheme. For any scheme $X$, we let $X^{\rm sm}$ denote its smooth locus. If $F$ is a coherent sheaf, we let $P(F)$ denote its sheaf of principal parts. We will let $e\from H^{0}(X,F) \to P(F)$ denote the natural evaluation morphism -- we suppress the dependence on $F$. If $s$ is a global section of a locally free sheaf, we let $v(s)$ denote the vanishing scheme of $s$.  If $L$ is a line bundle, we let $|L|$ denote the projective space $\P(H^{0}(L))$.  If $L$ is a line bundle on a smooth variety $Y$, and $s \in H^{0}(Y,L)$ is a section, then the {\sl singular scheme} $\sing(v(s))$ of $s$ is the vanishing scheme of $e(s) \in H^{0}(Y,P(L))$; if $K$, the kernel sheaf of $e \from H^{0}(Y,L) \otimes \cO_{Y} \to P(L)$, is locally free, then $\sing(v(s))$ is the largest closed subscheme $T \subset Y$  such that $s\from \cO_{T} \to H^{0}(Y,L) \otimes \cO_{T}$ factors through $K|_{T}$.  

\section{The projection-ramification map}\label{sec:prmap}
In this section, we define a projection-ramification map for a pair $(X, L)$ consisting of a proper, normal, variety $X$ and a sufficiently positive line bundle $L$ on $X$.
For $X \subset \P^n$, taking $L = \O(1)$ recovers the projection-ramification map introduced in \autoref{sec:intro}.
Working with abstract pairs, however, offers more flexibility that is helpful in inductive proofs.

Let $X$ be a proper variety of dimension $r$ over an algebraically closed field $k$ of characteristic zero.
A \emph{linear series} on $X$ is a pair $(L, W)$ consisting of a line bundle $L$ on $X$ and a subspace $W \subset H^0(X, L)$.
The \emph{complete linear series} associated to $L$ is $(L, W)$ with $W = H^0(X, L)$.
A \emph{projection} is a linear series $(L, V)$ with $\dim V = r+1$.
A \emph{projection of $(L, W)$} is a projection $(L, V)$ with $V \subset W$.
As a convention, we use $V$ for projections and $W$ for more general linear series.

\begin{definition}  \label{definition:properlyramified}
We say that a projection $(L,V)$ is \emph{properly ramified} if the evaluation homomorphism
\[e \from V \otimes \O_{X} \to P(L)\]
is an isomorphism over a general point in $X$.  If $(L,V)$ is properly ramified, its \emph{ramification divisor}
\[R(L,V) \subset X\]
is the closure of the scheme defined by the determinant of $e \from V \otimes \O_{X^{\rm sm}} \to P(L)|_{X^{\rm sm}}$.
\end{definition}
In most cases, $L$ is clear from context, so we drop it from the notation and denote the ramification divisor simply by $R(V)$.

A projection $(L, V)$ gives the evaluation map
\[e \from V \otimes \O_X \to L.\]
The evaluation map yields a map $p_{V,L} \from X \dashrightarrow \P V$, regular on the non-empty open set of $X$ where $e$ is surjective.
The following is an easy observation, whose proof we skip.
\begin{proposition}\label{prop:proj}
  The projection $(L, V)$ is properly ramified if and only if the map on tangent spaces induced by $p_{V,L}$ is generically an isomorphism.
  In characteristic zero, this is equivalent to the condition that $p_{V,L}$ is dominant.
\end{proposition}

For a fixed $(L, W)$, the set of all projections of $(L, W)$ are parametrized by the Grassmannian $\Gr(r+1, W)$.
The property of being properly ramified is a Zariski open condition on the Grassmannian.

We now define a map that assigns to a projection its ramification divisor.
To do so, we interpret the ramification divisor as an element of a linear series.

Assume, furthermore, that $X$ is normal.
Let $K_X$ be the canonical sheaf of $X$.
Denoting by $i \from X^{\rm sm} \to X$ the inclusion, $K_X$ is given by the push-forward
\[ K_X = i_* K_{X^{\rm sm}}.\]
Note that, since $X$ is normal, the complement of $X^{\rm sm} \subset X$ has codimension at least 2.
The sheaf $K_X$ is coherent, reflexive, and satisfies Serre's S2 condition.

Let $L$ be a line bundle on $X$.
The sheaf $P(L)$ is locally free of rank $(r+1)$ on $X^{\rm sm}$, and we have a canonical isomorphism
\[ \bigwedge^{r+1} P(L) |_{X^{\rm sm}} \cong K_{X^{\rm sm}} \otimes L^{r+1}.\]
Given a subspace $V \subset H^0(X, L)$, we apply $\bigwedge^{r+1}$ to the evaluation map
\[ e \from V \otimes \O_{X^{\rm sm}} \to P(L)|_{X^{\rm sm}},\]
to get
\[ \det e \from \det V \otimes \O_{X^{\rm sm}} \to K_{X^{\rm sm}} \otimes L^{r+1}. \]
By applying $i_*$ and taking global sections, we get
\begin{equation}\label{eqn:ramsection}
  r_V \from \det V \to H^0(X, K_X \otimes L^{r+1}).
\end{equation}
If $(L, V)$ is properly ramified, then this map is non-zero, and hence gives a point of the projective space $\P H^0(X, K_X \otimes L^{r+1})^*$.
Doing the same construction universally over the Grassmannian $\Gr = \Gr(r+1, W)$ yields a map
\begin{equation}\label{eqn:rammap}
  r \from \det \mathcal V \to H^0(X, K_X \otimes L^{r+1}) \otimes \O_{\Gr},
\end{equation}
where $\mathcal V \subset W \otimes \O_{\Gr}$ is the universal sub-bundle of rank $(r+1)$.
Let $U \subset \Gr$ be the open subset of properly ramified projections.
Then the map in \eqref{eqn:rammap} is non-zero at every point of $U$, and defines a map $U \to \P H^0(X, K_X \otimes L^{r+1})^*$ given by the surjection
\begin{equation}\label{eqn:rammapfamily}
  H^0(X, K_X \otimes L^{r+1})^* \otimes \O_{U} \to \det \mathcal V|_U^*.
\end{equation}
Note that $U$ is non-empty if and only if $W$ separates tangent vectors at a general point of $X$.
\begin{definition}
  \label{def:ProjectionRamification}
  Let $(L, W)$ be a linear series that separates tangent vectors at a general point of $X$.
  The \emph{projection-ramification} map for $(L,W)$ is the rational map
  \[
    \rho_{(X,L,W)} \from \Gr(r+1, W) \dashrightarrow \P H^0(X, K_X \otimes L^{r+1})^*
  \]
  defined on the non-empty open subset of properly ramified maps by \eqref{eqn:rammapfamily}.
\end{definition}
If any of $X$, $L$, or $W$ are clear from context, we drop them from the notation.
In particular, for a non-degenerate $X \subset \P^n$, we denote by $\rho_X$ the map $\rho_{X,L,W}$  with $L = \O_X(1)$ and $W$ the image in $H^0(X, L)$ of $H^0(\P^n, \O(1))$.

Note that the map \eqref{eqn:rammapfamily} factors as
\[ \det \mathcal V \xrightarrow{a} \bigwedge^{r+1} W \otimes \O_{\Gr} \xrightarrow{b} H^0(X, K_X \otimes L^{r+1}) \otimes \O_{\Gr},\]
where $a$ is $\wedge^{r+1}$ applied to the universal inclusion $\mathcal V \subset W \otimes \O_{\Gr}$, and $b$ is induced by $\wedge^{r+1}$ applied to the evaluation map $e \from W \otimes \O_{X} \to P(L)$.
The map $a$ defines the Pl\"ucker embedding
\[ i \from \Gr(r+1, W) \to \P \left(\bigwedge^{r+1}W^*\right),\]
and the map $b$ defines a linear projection
\[ p \from \P \left(\bigwedge^{r+1}W^*\right) \dashrightarrow \P H^0(X, K_X \otimes L^{r+1}).\]
Thus, $\rho_{X,L,W}$ factors as the Pl\"ucker embedding followed by a linear projection.

\section{Maximal variation for incompressible and non-defective $X$}
\label{sec:proof_of_theorem:main}
The goal of this section is to prove \autoref{thm:main}.
We begin by proving part \eqref{item:incomp}, which is substantially easier.
\begin{proposition}[\autoref{thm:main}~\eqref{item:incomp}]
  \label{prop:incompress}
  Let $X \subset \P^n$ be a non-degenerate, normal, incompressible projective variety over a field of characteristic zero.
  Then $\rho_X$ is a finite map.
\end{proposition}
\begin{proof}
  Set $L = \O(1)$ and let $W \subset H^0(X, L)$ be the image of $H^0(\P^n, \O(1))$.
  Let $V \subset W$ be an $(r+1)$-dimensional subspace.
  Since $X$ is incompressible, the projection map $p_{V,L} \from X \dashrightarrow \P V$ induced by $(L, V)$ is dominant.
  By \autoref{prop:proj}, this implies that $(L, V)$ is properly ramified.
  Since $V$ was arbitrary, the projection-ramification map 
  \[ \rho \from \Gr(r+1, W) \to |K_X + (r+1) H|\]
  is regular.
  Since the Picard rank of a Grassmannian is $1$, a regular map from a Grassmannian is either constant or finite.
  It is easy to check that $\rho$ is not constant; so it must be finite.
\end{proof}

For the proof of part \eqref{item:dual} of \autoref{thm:main}, we proceed inductively by showing that a general $(n-r-1)$-dimensional linear subspace which is incident to $X$ is an isolated point in its fiber under $\rho$.
Again, it is more convenient to work with the more abstract set-up of a linear series, allowing for series that are not very ample.

Let $X$ be a proper variety of dimension $r$, and let $(L, W)$ be a linear series on $X$.
For an ideal sheaf $I \subset \O_X$ we denote by $W \otimes I$ the subspace of $W$ consisting of the sections that vanish modulo $I$. %he
More precisely, if $K$ is the kernel of the evaluation map
\[ W \otimes \O_X \to L \otimes \O_X/I,\]
then $W \otimes I = H^0(X, K)$.
In particular, for $W = H^0(X, L)$, we have $W \otimes I = H^0(X, L \otimes I)$.
For $s \in W \otimes I$, the vanishing locus $v(s)$ refers to the vanishing locus of $s$ as a section of $L$.
We set $|W| = \P W^*$, the space of one-dimensional subspaces of $W$, and likewise $|W \otimes I| = \P (W \otimes I)^*$.
For a complete linear series, we write $|L|$ for $|W|$.
Note that $v(s) = v(\lambda s)$ for a non-zero scalar $\lambda$, so it causes no ambiguity to talk about $v(s)$ for $s \in |W|$.

\subsection{Non-defective linear series}\label{sec:non-defectivity}
We study a positivity property of linear series that generalizes the property of having a divisorial dual.
\begin{definition}
  \label{definition:Genericallynon-defective} 
  We say that a linear series $(L, W)$ is \emph{non-defective} if,  for a general point $x \in X$ either $W \otimes \mathfrak m_x^2 = 0$, or there exists $s \in W \otimes \mathfrak m_x^2$ such that $v(s)$ has an isolated singularity at $x$.
\end{definition}
Note that for $s \in |W|$, the condition that $v(s)$ have an isolated singularity at $x$ is a Zariski open condition on $|W|$.
Therefore, if there exists an $s \in |W \otimes \mathfrak m_x^2|$ such that $v(s)$ has an isolated singularity at $x$, then a general $s \in |W \otimes \mathfrak m_x^2|$ has the same property.
\begin{remark}
  Let $x$ be a point of $X$.
  Suppose there exists $s \in |W|$ with an isolated singularity at $x$.
  It may be tempting to conclude from this that $(L, W)$ is non-defective.
  This is not necessarily true!
  For example, take $X = \F_3$.
  Denote by $E$ the section of self-intersection $-3$ and $F$ the fiber of the projection $\F_3 \to \P^1$.
  Let $L = \O_X(E + 2F)$ and $W = H^0(X, L)$.
  For $x \in E$, the general member of $|W \otimes \mathfrak m_x^2|$ has an isolated singularity at $x$, but the same is not true for a general $x \in X$.
\end{remark}

\begin{remark}
  Suppose $(L, W)$ is non-defective.
  Let $x \in X$ be general, and let $s \in |W|$ be such that $v(s)$ has an isolated singularity at $x$.
  For all such $s$, it may be the case $v(s)$ has singularities away from $x$, even along a positive dimensional locus.
  For example, let $\pi \from X \to \P^2$  be the blow-up at a point, and $E$ the exceptional divisor.
  The complete linear series associated to $L = \pi^* \O(2) \otimes \O(2E)$ is non-defective, but for every global section of $L$, the singular locus of $v(s)$ contains $E$.
\end{remark}


We now define the conormal variety of a linear series, which plays an important role in our analysis of non-defectivity.
Let $K$ be the kernel of the evaluation map
\[ e \from W \otimes \O_X \to P(L).\]
Let $U \subset X$ be an open subset such that $K|_U$ is locally free and the dual of the inclusion 
\[W^* \otimes \O_U \to K|_U^*\]
is a surjection.
This surjection defines a closed embedding $\P(K|_U) \subset U \times |W|$.
The \emph{conormal variety of $(L,W)$}, denoted by $P_{L,W}$, is the closure of $\P(K|_U)$ in $X \times |W|$.

\begin{proposition}\label{prop:dimension}
  \label{prop:dimP}
  Suppose $(L, W)$ is non-defective.
  If $\dim W \geq r+2$, then $P_{L,W}$ is irreducible of dimension $\dim W - 2$.
  If $\dim W \leq r+1$, then $P_{L,W}$ is empty.
\end{proposition} 

\begin{proof}
  Set $n = \dim |W| = \dim W - 1$.
  Let $k$ be the (generic) rank of $K$, namely the rank of the locally free sheaf $K|_U$.
  Then $k \geq n-r$.
  The statement of the proposition is equivalent to showing that if $k > 0$, then $k = n-r$.

  For brevity, set $P = P_{L,W}$.
  Consider the projection $\sigma \from P \to |W|$, obtained by restricting the second projection $X \times |W| \to |W|$.
  For $s \in |W|$, we view $\sigma^{-1}(s)$ as a subscheme of $X$.
  We then have
  \begin{align*}
    \sigma^{-1}(s) \cap U = \sing(v(s)) \cap U.
  \end{align*}

  Suppose $r>0$.
  Then $P$ is non-empty and irreducible, since it is the closure of a non-empty and irreducible variety.
  Since $(L,W)$ is non-defective, a general point $(x,s) \in P$ is such that $x$ is an isolated point of $\sing(v(s))$.
  Therefore, $\sigma \from P \to |W|$ is generically finite onto its image.
  We conclude that $\dim P \leq \dim |W|$, and hence $k \leq n-r+1$.

  To show that $k = n-r$, it suffices to show that $\sigma \from P \to |W|$ is not surjective.
  We do so using Bertini's theorem.
  Let $B \subset X$ denote the union of the base locus of $|W|$ and the singular locus of $X$.
  Then $B$ is a proper closed subset of $X$.
  Let $P^B \subset P$ be the pre-image of $B$ under the projection $\pi \from P \to X$.
  By the definition of $P$, the map $\pi \from P \to X$ is surjective, and hence $P^B$ is a proper closed subset of $P$.
  Since $P$ is irreducible, we have $\dim P^B < \dim P \leq \dim |W|$, so the projection $P^B \to |W|$ cannot be dominant.
  Let $s \in |W|$ be general, in particular, not in the image of $P^B \to |W|$.
  By Bertini's theorem $v(s)$ is non-singular away from $B$.
  Thus, for any $x \in X$, the point $(x, s) \in X \times |W|$ does not lie in $P$.
  For $x \in B$, this is because $s$ is not in the image of $P^B$, and for $x \not \in B$, this is because $v(s)$ is non-singular at $x$.
  We conclude that $s$ does not lie in the image of $P \to |W|$.
  Hence $P \to |W|$ is not surjective.
\end{proof} 

\begin{proposition}
  \label{prop:dimensionCriterion}
  Let $(L, W)$ be a linear series with $\dim W \geq r+2$, and let $P = P_L$ be its conormal variety.
  The projection $\sigma\from P \to |W|$ is generically finite onto its image if and only if $(L, W)$ is non-defective. 
\end{proposition}

\begin{proof}
  Since $\dim W \geq r+2$, the conormal variety $P = P_{L,W}$ is non-empty.
  Let $(x,s) \in P$ be a general point.
  We may assume that $x \in U$.
  Then $x$ is a singular point of $v(s)$, and it is an isolated singularity of $v(s)$ if and only if $(x,s)$ is an isolated point in the fiber of $\sigma \from P \to |W|$ over $s$.
  The conclusion follows.
\end{proof}

The following observation relates non-defectivity with the non-degeneracy of the dual.
\begin{proposition}\label{prop:non-deg-dual}
  Let $X \subset \P^n$ be a non-degenerate projective variety.  Let $L = \O_X(1)$ and $W \subset H^0(X, L)$ the image of $H^0(\P^n, \O(1))$.
  Then $(L, W)$ is non-defective if and only if the dual variety $X^* \subset {\P^n}^*$ is a hypersurface.
\end{proposition}
\begin{proof}
  Since $X \subset \P^n$ is not contained in a hyperplane, we have $\dim W = n+1 \geq r+1$.
  Since $(L, W)$ is very ample, it separates tangent vectors on $X$, so the evaluation map
  \[ e \from W \otimes \O_X \to P(L)  \]
  is surjective.
  It follows that the rank of the kernel is $n-r$, and hence
  \[ \dim P_{L,W} = (n-r - 1) + r = n-1.\]
  By definition, the dual variety $X^* \subset {\P^n}^* = |W|$ is the image of the conormal variety under the projection $P_{L,W} \to |W|$.
  By \autoref{prop:dimensionCriterion}, $(L, W)$ is non-defective if and only if $\dim X^* = n-1$.
\end{proof}

\begin{proposition}\label{prop:ordinarydoublepoint}
  Let $(L, W)$ be a non-defective linear series on $X$ with $\dim W \geq r+2$.
  Let $x \in X$ be a general point.
  Then there exists $s \in |W|$ such that $v(s)$ has an ordinary double point singularity at $x$.
\end{proposition}
\begin{proof}
  By \autoref{prop:dimensionCriterion}, the projection $\sigma\from P \to |W|$ is generically finite onto its image. 
  Let $(x,s) \in P$ be a general point.
  Since our ground field is of characteristic zero, we may assume that $P$ is smooth at $(x,s)$, that $x \in U \cap X^{\rm sm}$, and $\sigma \from P \to |W|$ is a local immersion at $(x,s)$.
  This implies that $x \in \sing(v(s))$ is isolated, and also that $x$ is a reduced point of the scheme $\sing(v(s))$.
  These two properties show that $v(s)$ possesses an ordinary double point at $x$.
  To see this, choose local coordinates $(x_{1}, ..., x_{n})$ so that the complete local ring ${\widehat{\O}_{X,x}}$ is isomorphic to $k\llbracket x_{1},\dots, x_{r}\rrbracket$.
  After choosing a local trivialization for $L$ around $x$, the section $s$ corresponds to a power series $s(x_1,\dots,x_r)$ contained in $\mathfrak m_x^2 \widehat \O_{X,x}$.
  The germ of $\sing(v(s))$ at $x$ is cut out by the power series $\frac{\partial s}{\partial x_1}, \dots, \frac{\partial s}{\partial x_r}$.
  Since the germ of $\sing(v(s))$ at $x$ is the reduced point $x$, we get that $\frac{\partial s}{\partial x_1}, \dots, \frac{\partial s}{\partial x_r}$ are linearly independent as elements of $\mathfrak m_x / \mathfrak m_x^2$.
  From this, it is easy to check that the tangent cone of $s(x_1, \dots, x_r)$ at $x$ is a non-degenerate quadric cone.
\end{proof}


\begin{proposition}
  \label{prop:genericSeparateTangents}
  If $(L, W)$ is a non-defective linear series with $\dim W \geq r+1$, then $W$ separates tangent vectors at a general point $x \in X$.
  That is, the evaluation map
  \[e_{x}\from W \otimes \O_X \to L/\mathfrak m_x^2 L\]
  is surjective for general $x \in X$.
\end{proposition}
\begin{proof}
  By the definition of $P(L)$, we have a natural isomorphism
  \[ P(L)|_x = L/\mathfrak m_x^2 L,\]
  so it suffices to show that the evaluation map
  \[ e \from W \otimes \O_X \to P(L)\]
  is surjective at $x$.
  Let $k$ be the generic rank of $K$, the kernel of $e$.
  From the proof of \autoref{prop:dimP}, we get
  \[  k = \dim W - r - 1.\]
  Since $(r+1)$ is the generic rank of $P(L)$, we conclude that $e$ is generically surjective.
\end{proof}
\begin{corollary}\label{cor:properlyramified}
  Suppose $(L, W)$ is a non-defective linear series on $X$ with $\dim W \geq r+1$.
  Then there exists a properly ramified projection $(L,V)$ of $(L, W)$.
\end{corollary}
\begin{proof}
  This follows immediately from \autoref{prop:genericSeparateTangents}.
\end{proof}
As a consequence of \autoref{cor:properlyramified}, the projection-ramification rational map $\rho_{X,L, W}$ is defined for a non-defective linear series $(L, W)$ with $\dim W \geq r+1$.

Let $\pi \from \widetilde X \to X$ be the blow-up at a point $x \in X$, and $E \subset \widetilde X$ the exceptional divisor.
A linear series $(L, W)$ on $X$ gives a linear series $(\widetilde L, \widetilde W)$ as follows.
Take $\widetilde L = \pi^* L \otimes \O_{\widetilde X}(-E)$.
Note that $H^0(X, L) = H^0(\widetilde X, \pi^*L)$, so we may think of $W$ as a subspace of $H^0(\widetilde X, \pi^*L)$.
Take $\widetilde W = W \otimes \O_{\widetilde X}(-E)$ with its natural inclusion $\widetilde W \subset H^0(\widetilde X, \widetilde L)$.
\begin{proposition}
  \label{prop:blowuppoint}
  In the setup above, if $(L, W)$ is non-defective, $\dim W \geq r+2$, and $x \in X$ is general, then $(\widetilde L, \widetilde W)$ is also non-defective.
\end{proposition}

\begin{proof}
  Let $y$ be a general point of $\widetilde X$.
  We have the equality
  \[ \widetilde W \otimes \mathfrak m_y^2 = W \otimes \mathfrak m_x \cdot \mathfrak m_y^2. \]
  By \autoref{prop:genericSeparateTangents}, for a general $y \in X$, we have
  \[ \dim (W \otimes \mathfrak m_y^2) = \dim W - (r+1).\]
  Since $x \in X$ is general, we get
  \[ \dim (W \otimes \mathfrak m_x \cdot \mathfrak m_y^2) = \dim W - (r+2).\]
  If $\dim W = r+2$, then we get $\widetilde W \otimes \mathfrak m_y^2 = 0$, so we are done.
  Assume that $\dim W \geq r+3$.
  Then $\dim (W \otimes \mathfrak m_y^2) \geq 2$.
  Since $(L, W)$ is non-defective, a general $s \in W \otimes \mathfrak m_y^2$ is such that $v(s)$ has an isolated singularity at $y$.
  Moreover, since $\dim (W \otimes \mathfrak m_y^2) \geq 2$, for every $x \in X$, there exists $s \in V$ such that $v(s)$ passes through $x$.
  Hence, as $x \in X$ is general, there exists $s \in W \otimes \mathfrak m_y^2$ such that $v(s)$ has an isolated singularity at $y$ and passes through $x$.
  That is, there exists $s \in \widetilde W \otimes \mathfrak m_y^2 $ that has an isolated singularity at $y$.
  We conclude that $(\widetilde L, \widetilde W)$ is non-defective.
\end{proof}



\subsection{Maximal variation for non-defective pairs}

In this section, we prove part~\eqref{item:dual} of \autoref{thm:main}.
In fact, we prove a more general result (\autoref{thm:mainMain}).

As before, $X$ is a proper, normal variety of dimension $r$ over an algebraically closed field of characteristic zero.
\begin{theorem}
  \label{thm:mainMain}
  Let $(L, W)$ be a non-defective linear series on $X$ with $\dim W \geq r+2$.
  Then the projection-ramification map $\rho_{X,L,W}$ is generically finite onto its image.
\end{theorem}

For the proof, we need two lemmas, which are essentially local computations.
Throughout, $X$, $L$, and $W$ are as in the statement of \autoref{thm:mainMain}.

\begin{lemma}\label{lemma:tangentconeRam}
  Let $x \in X$ be a general point and $V \subset W \otimes \mathfrak m_x$ a general $(r+1)$-dimensional subspace.
  Then $V$ is properly ramified, and the ramification divisor $R(V)$ has an ordinary double point singularity at $x$.
\end{lemma}

\begin{proof}
  Using \autoref{prop:ordinarydoublepoint} and \autoref{prop:genericSeparateTangents}, we get a basis $(s_{1}, ..., s_{n}, t)$ of $V$ satisfying the following two conditions:
  \begin{enumerate}
      \item $s_{1}, \dots, s_{n}$ generate $L \otimes ({\mathfrak m}_{x}/{\mathfrak m}^{2}_{x})$, and
      \item $v(t)$ has an ordinary double point singularity at $x$.
    \end{enumerate}  

    Let $\widehat{\O}_{X,x}$ denote the completion of the local ring at $x \in X$ along its maximal ideal.  Upon trivializing $L$, we may regard $s_{i}$ and $t$ as elements of $\widehat{\O}_{X,x}$, and can also assume  $\widehat{\O}_{X,x} = k\llbracket s_{1}, \dots s_{n}\rrbracket$.
    In the bases $(s_1, \dots, s_n, t)$ for $V$ and $(1, s_1, \dots, s_n)$ for $P(L)$, the evaluation map 
\begin{align*}
  e\from V \otimes \widehat{\O}_{X,x} \to P(L) \otimes \widehat{\O}_{X,x}
\end{align*}
has the matrix
\begin{align}\label{matrix}
\begin{pmatrix}
  s_{1} & s_{2} & \dots & t \\
  1 & 0 & \dots & \partial_{1}t \\
  0 & 1 & \dots & \partial_{2}t \\
  \vdots & \vdots & \vdots & \vdots \\
  0 & 0 & \dots & \partial_{n}t
\end{pmatrix},
\end{align}
where $\partial_{i}$ denotes $\frac{\partial}{\partial s_{i}}$.
The determinant of the matrix \eqref{matrix}
\begin{align*}
  t - \sum_{i}s_{i}\partial_{i}t
\end{align*}
is an analytic local equation for the ramification divisor $R(V)$ near $x$.
Evidently, $R(V)$ shares the same tangent cone as $v(t)$ at $x$.
The proposition follows.
\end{proof}

\begin{lemma}\label{lemma:basepointfree}
  Let $x \in X$ be a general point and $V \subset W$ an $(r+1)$-dimensional subspace with a basis $(u, a_1,\dots, a_{r-1},b)$ where
  \begin{enumerate}
    \item $u$ does not vanish at $x$,
    \item $a_1, \dots , a_{r-1}$ vanish at $x$, and reduce to linearly independent elements of $L \otimes ({\mathfrak m}_{x}/{\mathfrak m}^{2}_{x})$, and
    \item $v(b)$ has an ordinary double point at $x$. 
  \end{enumerate}
  Then $R(V)$ contains $x$ and is smooth at $x$.
\end{lemma}

\begin{proof}
  That $R(V)$ contains $x$ is clear since $V \otimes \mathfrak{m}^{2}_{x} \neq 0$.

  For smoothness, we again work in the completion $\widehat{\O}_{X,x}$.
  After trivializing $L$, we assume $u, a_{1}, ..., b$ are elements of $\widehat{\O}_{X,x}$.
  We choose an element $z \in \widehat{\O}_{X,x}$ such that $(a_1, \dots, a_{r-1}, z)$ forms a system of coordinates, that is $\widehat{\O}_{X,x} \cong k\llbracket a_{1}, \dots , a_{r-1}, z \rrbracket$.
  With respect to the given basis of $V$ and the basis $1, a_1, \dots, a_{r-1}, z$ for $P(L)$, the evaluation map
  \begin{align*}
  e\from V \otimes \widehat{\O}_{X,x} \to P(L) \otimes \widehat{\O}_{X,x}
  \end{align*}
  has the matrix
\begin{align}\label{matrix2}
\begin{pmatrix}
  u & a_{1} & a_{2} & \dots & b \\
  \partial_{1}u & 1 & 0 & \dots & \partial_{1}b \\
  \partial_{2}u & 0 & 1 & \dots & \partial_{2}b \\
  \vdots & \vdots & \vdots & \vdots \\
  \partial_{z}u  & 0 & 0 & \dots & \partial_{z}b
\end{pmatrix}
\end{align}
The determinant of the matrix \eqref{matrix2} is the analytic local equation for $R(V)$.
It is given by
\begin{align*}
   \bar{u} \cdot \partial_{z}b \pm \partial_{z}u \cdot \bar{b},
 \end{align*} 
 where, for $r \in \widehat{\O}_{X,x}$ we set
 \[\bar{r} = r - a_{1}\partial_{1}r - a_{2}\partial_{2}r - \dots - z \partial_{z} r.\]
 Since $b \in {\mathfrak m}^{2}_{x}$, we get that $\bar{b} \in {\mathfrak m}^{2}_{x}$, and so $\partial_{z}b \in {\mathfrak m}_{x}$.
 Furthermore, since the tangent cone of $b$ is a non-degenerate quadric, we also get that $\partial_z b \not \in \mathfrak m_x^2$.
 Since $\overline{u}$ is a unit, we see that the tangent cone of $R(V)$ at $x$ is the hyperplane cut out by $\partial_z b \in \mathfrak m_x/\mathfrak m_x^2$.
 So $R(V)$ is smooth at $x$.
\end{proof}

We now have all the tools for the proof of \autoref{thm:mainMain}. 
\begin{proof}[Proof of \autoref{thm:mainMain}]
  We induct on $\dim W$.
  The base case $\dim W = r+1$ is clear.

  We now do the induction step.
  Suppose $\dim W \geq r+2$.
  Choose a general point $x \in X$ such that the induced linear series $(\widetilde L, \widetilde W)$ on $\widetilde X = \Bl_x X$ is non-defective as in \autoref{prop:blowuppoint}.
  Choose a general $(r+1)$-dimensional subspace $V \subset W \otimes \mathfrak m_x = \widetilde W$ that satisfies the hypotheses of \autoref{lemma:tangentconeRam}.
  By the induction hypothesis, $V$ considered as a projection of $(\widetilde L, \widetilde W)$ is an isolated point in the projection-ramification map for $\widetilde X$.
  We now show that it is also an isolated point in the projection-ramification map for $X$.

  Let $(C, 0)$ be a pointed smooth curve and $V \subset W \otimes \O_C$ a sub-bundle of rank $(r+1)$ such that
  \begin{enumerate}
  \item $V_{0} = V$, and 
  \item $V_{c} \neq W_{0}$ for $c \in C \setminus \{0\}$.
  \end{enumerate}
  We must show that $R(V_c) \neq R(V)$ for a general $c \in C$.

  Suppose $V_c \subset W \otimes \mathfrak m_x = \widetilde W$ for all $c \in C$.
  Denote by $\widetilde R(V_c)$ the ramification divisor of $V_c$ considered as a projection of $\widetilde X$.
  Since $V = V_0$ is an isolated point in the projection-ramification map for $\widetilde X$, we know that $\widetilde R(V_c) \neq \widetilde R(V_0)$ for a general $c \in C$.
  Clearly, $R(V_c)$ and $\widetilde R(V_c)$ agree away from the exceptional divisor, and hence we conclude that $R(V_c) \neq R(V_0)$ for a general $c \in C$.

  On the other hand, suppose $V_c \not \subset W \otimes \mathfrak m_x = \widetilde W$ for a general $c \in C$.
  Consider the evaluation maps
  \[ e_c \from V_c \to L / \mathfrak m_x^2 L \]
  between an $(r+1)$-dimensional source and $(r+1)$-dimensional target.
  Since $V = V_0$ satisfies the hypotheses of \autoref{lemma:tangentconeRam}, $\rk e_0 = r$.
  Therefore, by semi-continuity, $\rk e_c \geq r$ for all $c \in C$.
  If $\rk e_c = (r+1)$ for a general $c \in C$, then $x \not \in R(V_c)$, and hence $R(V_c) \neq R(V)$.
  Otherwise, by shrinking $C$ if necessary, assume $\rk e_c = r$ for all $c \in C$.
  In other words, $\dim (V_c \otimes \mathfrak m_x^2) = 1$ for all $c \in C$.
  Let $b_c \in V_C \otimes \mathfrak m_x^2$ be a non-zero element.
  Since $v(b_0)$ has an ordinary double-point singularity at $x$, so does $v(b_c)$.
  Also, since $\rk (e_c) = r$ and $V_c \not \in W \otimes \mathfrak m_x$ for a general $c$, there exists $u_c \in V_c$ not vanishing at $x$, and a set of $(r-1)$ other elements that vanish at $x$ but reduce to linearly independent elements modulo $\mathfrak m_x^2$.
  That is, $V_c$ satisfies the hypotheses of \autoref{lemma:basepointfree} for a general $c \in C$.
  But \autoref{lemma:basepointfree} implies that $R(V_c)$ is smooth at $x$.
  Since $R(V_0)$ is singular at $x$, we conclude that $R(V_0) \neq R(V_c)$.
  The induction step is now complete.
\end{proof}

We immediately get part~\eqref{item:dual} of \autoref{thm:main}.
\begin{corollary}
  \label{cor:maintheorem} Let $X \subset \P^{n}$ be a non-degenerate projective variety such that the dual variety $X^{*} \subset \P^{n*}$ is a hypersurface. Then $\rho_{X}$ is generically finite onto its image.
\end{corollary}
\begin{proof}
  By \autoref{prop:non-deg-dual} the linear series on $X$ that gives the embedding $X \subset \P^n$ is non-defective.
  Now apply \autoref{thm:mainMain}.
\end{proof}


\section{Projection-ramification for varieties of minimal degree}\label{sec:minimaldegree}
In this section, we prove \autoref{theorem:minimaldegree}, which relates varieties of minimal degree and the projection-ramification map.
We then prove \autoref{Thm:Counterexamples} by constructing examples of rational scrolls where maximal variation fails.
Finally, we obtain an alternate description of the projection-ramification map for scrolls, which is used in \autoref{sec:proof_of_theorem:rationalnormalscrolls}.

The following is an easy application of the Kodaira vanishing theorem.
\begin{proposition}\label{lemma:KYmH}
  Let $X \subset \P^n$ be a non-degenerate, smooth, projective, variety of dimension $r \geq 1$ over a field of characteristic zero.
  For all $m \geq r$, we have the inequality
  \begin{equation}\label{eqn:KYmH}
    {m \choose r}(n-r) + {{m-1} \choose {r}}\leq h^0(X, K_X + mH).
  \end{equation}
  If equality holds for any $m \geq r$, then $X$ is a variety of minimal degree, that is $\deg X = n-r+1$.
  Conversely, for a variety of minimal degree, equality holds for all $m \geq r$.
\end{proposition}

\begin{proof}
  Without loss of generality, $X$ is embedded by the complete linear series.
  Indeed, passing to the complete linear series only increases the left side of the desired inequality, and does not change the right side.
  
  We first prove the inequality \eqref{eqn:KYmH}, using a double induction--first on $r$, and then on $m$.
  For the base case $r = 1$, Riemann--Roch gives
  \begin{equation}\label{eqn:r1}
    h^0(X, K_X + mH) = g_X - 1 + mn,
  \end{equation}
  from which \eqref{eqn:KYmH} follows for all $m$.

  Assume that \eqref{eqn:KYmH} holds for varieties of dimension $(r-1)$ and all $m \geq r-1$.
  Let $D \subset X$ be a general member of the linear series $|H|$.
  By Bertini's theorem, $D$ is a smooth variety.
  The adjunction formula $K_D = (K_X + H)|_D$ yields the exact sequence
  \begin{equation}\label{eqn:mainexact}
    0 \to \O_X(K_X + (m-1)H) \to \O_X(K_X + mH) \to \O_D(K_D + (m-1)H) \to 0.
  \end{equation}
  Note that, by the Kodaira vanishing theorem, we have $h^1(K_X + nH) = 0$ for all $n > 1$; we use this repeatedly, without further comment.
  For $m = r$, the long exact sequence in cohomology associated to \eqref{eqn:mainexact} gives
  \[ h^0(K_D + (r-1)H) \leq h^0(K_X + rH).\]
  By applying the induction hypothesis to $D$, we have
  \begin{equation}
    n-r \leq h^0(K_D + (r-1)H)
  \end{equation}
  Therefore, we conclude that
  \begin{equation}
    (n-r) \leq h^0(K_D + rH).
  \end{equation}
  Let $m > r$, and assume that \eqref{eqn:KYmH} holds for $X$ for $m-1$.
  The long exact sequence in cohomology associated to \eqref{eqn:mainexact} gives
  \begin{equation}\label{eqn:add}
    h^0(K_X + (m-1)H) + h^0(K_D + (m-1)H) = h^0(K_X + mH).
  \end{equation}
  By applying the induction hypothesis to $m-1$, we get
  \begin{align*}
    h^0(K_X + (m-1)H) &+ h^0(K_D + (m-1)H)\\
                      &\geq{{m-1} \choose r}(n-r) + {{m-2} \choose {r}} + {{m-1} \choose {r-1}}(n-r) + {{m-2} \choose r-1} \\
                      &={m \choose r} (n-r) + {{m-1} \choose r}.
  \end{align*}
  Together with \eqref{eqn:add}, we conclude 
  \begin{equation}
    {m \choose r} (n-r) + {{m-1} \choose r} \leq h^0(K_X + mH), 
  \end{equation}
  which is \eqref{eqn:KYmH} for $m$.
  The proof of the inequality is thus complete.

  We now examine when equality holds in \eqref{eqn:KYmH}.
  For $r = 1$, the equation \eqref{eqn:r1} shows that equality holds for some $m$ if and only if $g_X = 0$, that is $X \subset \P^n$ is a rational normal curve, and in this case, equality holds for all $m$.
  Furthermore, we observe in the inductive proof that if equality holds for an $X$ of dimension $r > 1$ and some $m$, then it must hold for the hyperplane slice $D$ and $(m-1)$.
  Again, by an induction on $r$, we conclude that $\deg X = n-r+1$, that is, $X \subset \P^n$ is a variety of minimal degree.

  Finally, for $X \subset \P^n$ of minimal degree, induction on $r$ shows that equality holds in \eqref{eqn:KYmH} for all $m$.
\end{proof}

As a consequence, we immediately deduce \autoref{theorem:minimaldegree}.
\begin{theorem}[\autoref{theorem:minimaldegree}]
  Let $X \subset \P^n$ be a smooth, non-degenerate projective variety of dimension $r \geq 1$ over a field of characteristic zero.
  We have the inequality
  \[ \dim \Gr(n-r, n+1) \leq \dim |K_X + (r+1)H|,\]
  where equality holds if and only if $X$ is a variety of minimal degree, that is $\deg X = n-r+1$.
\end{theorem}
\begin{proof}
  Apply \autoref{lemma:KYmH} with $m = r+1$.
\end{proof}

\subsection{Projection-ramification for scrolls}
\autoref{theorem:minimaldegree} motivates a deeper investigation of the projection-ramification map for varieties of minimal degree.
Indeed, for $X \subset \P^n$ of minimal degree, the projection-ramification map is potentially generically finite.
Recall that a large class of varieties of minimal degree are the rational normal scrolls, namely $X = \P E$ for an ample vector bundle $E$ on $\P^1$ embedded by the complete linear series $\O_X(1)$.
If $\dim X \geq 3$, then $X$ is neither incompressible nor does it have a divisorial dual variety.
Therefore, for such $X$, \autoref{thm:main} leaves the question of maximal variation unanswered.

We now examine the projection-ramification map for projectivizations of vector bundles on smooth curves in more detail.
Let $C$ be a smooth curve and $E$ an ample vector bundle on $C$ of rank $r$.
Set $X = \P E$, the space of one-dimensional quotients of $E$, and $L = \O_X(1)$.
Denote by $\pi \from X \to C$ the natural map.

Let $(L, V)$ be a projection of $X$.
Recall from \eqref{eqn:ramsection} that such a projection gives a map
\[ r_V \from \det V \to H^0(X, K_X \otimes L^{r+1}),\]
whose zero locus is the ramification divisor $R(V) \subset X$.
Note that we have an isomorphism $K_X \cong \pi^* (\det E \otimes K_C) \otimes L^{-r}$, and hence, we may view $r_v$ as a map
\[ r_V \from \det V \to H^0(C, E \otimes \det E \otimes K_C).\]

We now describe another construction of a section of $E \otimes \det E \otimes K_C$ from $V$, which we call the \emph{differential construction}.
The subspace $V \subset H^0(X, L) = H^0(C, E)$ gives the evaluation map
\[ e \from V \otimes \O_C \to E.\]
If $V$ is generic, then $e$ is a surjection, and its kernel is canonically isomorphic to $\det E^* \otimes \det V$.
Consider the diagram
\begin{equation}\label{eqn:differential_construction}
\begin{tikzcd}
  0 \arrow{r}& \det E^* \otimes \det V \arrow{r}\arrow{d}{d_V}& V \otimes \O_C \arrow{r}{e}\arrow{d}{e}& E \arrow{r}\arrow[equal]{d}& 0 \\
  0 \arrow{r}& K_C \otimes E \arrow{r}& P(E) \arrow{r}& E \arrow{r}& 0,
\end{tikzcd}
\end{equation}
where the bottom row is the standard sequence associated to $P(E)$, both maps labeled $e$ are evaluation maps, and the map $d_V$ is the map induced by them.
The map $d_V$ gives a map
\[ d_V \from \det V \to H^0(C, E \otimes \det E \otimes K_C).\]
\begin{proposition}\label{prop:rdv}
  In the setup above, the two maps $d_V$ and $r_V$ are equal.
\end{proposition}
\begin{proof}
  Recall that $r_V$ is induced by the determinant of the evalutation map
  \[ V \otimes \O_X \to P(L).\]
  Denote by $P_\pi(L)$ the bundle of principal parts of $L$ along the fibers of $\pi$.
  More explicitly,
  \[ P_\pi(L) = {\pi_1}_* \left(\pi_2^* L \otimes \left(\O_{X \times_\pi X} / I_{\Delta}^2\right)\right),\]
  where $\Delta \subset X \times_\pi X$ is the diagonal and $\pi_i$ for $i = 1,2$ are the two projections $X \times_\pi X \to X$.
  It is easy to check that the evaluation map $\pi^* E \to L$ induces an isomorphism $\pi^* E \to P_\pi(L)$.
  Furthermore, we have the sequence
  \[ 0 \to \pi^* K_C \otimes L \to P(L) \to P_\pi(L) \to 0.\]
  By combining this with the identification $\pi^* E = P_\pi(L)$, and the top row of \eqref{eqn:differential_construction}, we get the diagram
  \begin{equation}\label{eqn:pxpl}
    \begin{tikzcd}
      0 \arrow{r}& \pi^*(\det E^* \otimes \det V) \arrow{r}\arrow{d}{p}& V \otimes \O_X \arrow{r}\arrow{d}{e}& \pi^* E \arrow{r}\arrow{d}\arrow[equal]{d}& 0\\
      0 \arrow{r}& \pi^* K_C \otimes L \arrow{r}& P(L) \arrow{r}& P_\pi(L) \arrow{r}& 0.
    \end{tikzcd}
  \end{equation}
  From the diagram, we see that $\det e = p$, interpreted as elements of the appropriate $\Hom$ spaces.
  By definition, after taking global sections, $\det e$ gives the section $r_V$.
  Note that, applying $\pi_*$ to the bottom row of \eqref{eqn:pxpl} yields the bottom row of \eqref{eqn:differential_construction}.
  Hence, after applying $\pi_*$, twisting by $\det E$ and taking global sections, $p$ gives the section $d_V$.
  We conclude that $r_V = d_V$.
\end{proof}

Let $R = R(V) \subset X$ be the ramification divisor of the projection given by $V$.
Note that $R$ is a divisor of class $\pi^*(\det E \otimes K_C) \otimes \O_X(1)$.
Therefore, $R \subset X$ is a sub-scroll, or equivalently, the fibers of $R \to C$ are hyperplanes in the corresponding fiber of $X \to C$.
We can obtain an explicit description of these hyperplanes in two ways, one using the original definition, and one using the differential construction.
Fix a point $c \in C$, and a uniformizer $t$ of $C$ at $c$.
Let $X_c \subset X$ and $R_c \subset R$ be the fibers of $X \to C$ and $R \to C$ over $c$, respectively.

By definition $R \subset X$ is the set of points $x \in X$ for which there exists $s \in V$ such that $v(s)$ is singular at $x$.
Since $s$ is a section of $L = \O_X(1)$, the hypersurface $v(s)$ is singular at $x$ if and only if it contains the entire fiber of $\pi \from X \to C$ through $x$.
Suppose $\pi (x) = c$.
Then, in an open set of $X$ containing $X_c$, we have $s = t s_1$ for a section $s_1$ of $\O_X(1)$.
Observe that, we have $\sing(v(s)) \cap F = v(s_1) \cap F$, and therefore, $R_c \subset X_c$ is the hyperplane cut out by $s_1$.

To obtain the same description using the differential construction, consider the top row of \eqref{eqn:differential_construction}.
Let $v$ be a local section of $V \otimes \O_C$ around $c$ that generates the kernel of $e \from V \otimes \O_C \to E$ at $c$.
The fiber of the evaluation map $V \otimes \O_C \to P(L)$ over $c$ sends $v \in V$ to the image of $e(v)$ in $L / \mathfrak m_c^2 L$.
Since $v$ generates the kernel of $e \from V \otimes \O_C \to L$ at $c$, we know that image of $e(v)$ in $L/ \mathfrak m_cL$ is zero.
Writing $e(v) = ts_1$ for a section $s_1$ of $E$ around $c$, we see that $d_V(v) = s_1 \otimes t \in E \otimes \mathfrak m_c/\mathfrak m_c^2$.
Thus, the fiber of the sub-scroll defined by $d_V$ over $c$ is the hyperplane in $X_c$ cut out by $s_1$.

Finally, we write an equation of $R(V) \subset X$ over an open subset of $C$ containing $c$ explicitly in coordinates.
Choose a trivialization $X_1, \dots, X_r$ for $E$ over an open set $U \subset C$ containing $c$.
Then $X_U \cong \P^{r-1} \times U = \Proj \O_U[X_1, \dots, X_r]$.
We have a trivialization of $K_C$ over $U$ given by $dt$.
We then get a trivialization of $P(E)|_U$ by $X_1, \dots, X_r, dt \otimes X_1, \dots, dt \otimes X_r$.
Choose a basis $v_0, \dots, v_r$ of $V$, and suppose the map $e \from V \otimes \O_U \to E_U$ is given by
\[ e(v_i) = \sum m_{i,j} X_j,\]
for $m_{i,j} \in \O_U$, where $0 \leq i \leq r$ and $1 \leq j \leq r$.
Then the map $\det E^* \otimes \det V \to V \otimes \O_U$ defining the kernel of $e$ is given by the $r \times r$ minors of the matrix $(m_{i,j})$.
Denote the $\ell$-th minor by $M_\ell$; that is $M_\ell = (-1)^{\ell}\det (m_{i,j} \mid i \neq \ell)$.
Then the map $d_V$ sends the generator to the element of $E \otimes K_C$ given by
\[ \sum_{i,j} M_i \cdot \frac{\partial m_{i,j}}{\partial t} \cdot (dt \otimes X_j).\]
Note that the expression above is the determinant of the $(r+1) \times (r+1)$ matrix
\begin{equation}\label{eqn:Rmatrix}
  \begin{pmatrix}
  m_{0,1} & m_{0,2} & \dots & m_{0,r} & \sum_{i = 1}^r \frac {\partial m_{0,j}}{\partial t} \cdot dt \otimes X_j \\
  m_{1,1} & m_{1,2} & \dots & m_{1,r} & \sum_{i = 1}^r \frac {\partial m_{1,j}}{\partial t} \cdot dt \otimes X_j \\
  \vdots & \ddots & \dots & \vdots & \vdots \\
  m_{r,1} & m_{r,2} & \dots & m_{r,r} & \sum_{i = 1}^r \frac {\partial m_{r,j}}{\partial t} \cdot dt \otimes X_j \\
\end{pmatrix}.
\end{equation}
This gives an equation for $R_U \subset X_U = \Proj \O_U[X_1, \dots, X_r]$.

\subsection{Failure of maximal variation}
In this section, we show that there exists ample vector bundles $E$ of rank $r \geq 4$ on $\P^1$ such that the projection-ramification map for $X = \P E$ is not generically finite.
In other words, a generic projection of $X$ can be deformed in a one-parameter family so that the ramification divisor remains unchanged.

Recall that the projection-ramification map for $X = \P E$ and the complete linear series of $L = \O_X(1)$ is a map
\[ \rho \from \Gr(r+1, H^0(X, L)) \dashrightarrow |K_X \otimes L^{r+1}|,\]
or equivalently a map
\[ \rho \from \Gr(r+1, H^0(\P^1, E)) \dashrightarrow \P H^0(\P^1, E \otimes \det E \otimes K_{\P^1})^*.\]
By construction, $\rho$ is equivariant with respect to the action of $\Aut(X)$, and in particular, by the subgroup $\Aut(X/\P^1)$.

We engineer the failure of maximal variation using the following observation.
\begin{proposition}\label{prop:trivialStabilizer}
  A generic point of $\Gr(r+1, H^0(\P^1, E))$ has a trivial stabilizer under the action of $\Aut(\P E/\P^1)$.
\end{proposition}
\begin{proof}
  Fix $(r+1)$ distinct points $p_0, \dots, p_r \in \P^1$.
  Let $V \subset H^0(\P^1, E)$ be a generic $(r+1)$ dimensional subspace.
  Let $e \from V \otimes \O_{\P^1} \to E$ be the evaluation map.
  The points $p_0, \dots, p_r$ give vectors $v_0, \dots, v_r \in V$, unique up to scaling, defined by the property that $e(v_i) = 0$ in the fiber $E|_{p_i}$.
  Choose a generic point $t \in \P^1$.
  We get $(r+1)$ points $x_0, \dots, x_r \in \P E^*|_{t} \cong \P^{r-1}$ given by $e(v_0), \dots, e(v_r)$ evaluated at $t$.
  For generic $V$ and $t$, it is easy to check that these points are in linear general position.
  Any element of $\Aut(\P E/\P^1)$ that fixes $V$ must fix $x_0, \dots, x_r$.
  But then it must act as the identity on the projective space $\P E^*|_t$, and hence on the dual projective space $\P E|_t$.
  Since $t \in \P^1$ is general, it follows that it must be the identity.
\end{proof}


\begin{proposition}\label{prop:specialE}
  There exist ample vector bundles $E$ of every rank $\geq 4$ such that a general point of $\P H^0(\P^1, E \otimes \det E \otimes K_{\P^1})$ has a positive-dimensional stabilizer under $\Aut(\P E/\P^1)$.
  In particular, we may take $E = \O(1)^{r-1} \oplus \O(k+1)$ where $k \geq 1$ and $r \geq 4$.
\end{proposition}
\begin{proof}
  It suffices to exhibit an $E$ such that a generic element of $H^0(\P^1, E \otimes \det E \otimes K_{\P^1})$ has a positive dimensional stabilizer under the action of $\Aut(E/\P^1)$.
  Take
  \[ E = \O(a)^{r-1} \oplus \O(b),\]
  where $0 < a < b$ are to be determined.
  Elements of $\Aut (E/\P^1)$ can be represented by block lower triangular square matrices
  \[M = 
    \begin{pmatrix}
      A &  \\
      U & B
    \end{pmatrix},
  \]
  where $A \in \GL_a(\k)$, $B \in \k^\times$, and $U = (u_i)$ is an $(r-1)$ length row with entries in $H^0(\P^1, \O(b-a))$.
  Set $d = (r-1)a + b$ so that $\det E = \O(d)$.
  Suppose $a$, $b$, and $r$, are such that
  \begin{equation}\label{eqn:requirement}
    (r-1) (b-a+1) \geq b+d-1 = (r-1)a+2b-1.
  \end{equation}
  Take a general element of $H^0(\P^1, E \otimes \det E \otimes K_{\P^1})$; say it is given by the column vector
  \[ v = (p_1, \dots, p_{r-1}, q)^T,\]
  where the $p_i$ (resp $q$) are homogeneous polynomials in $X, Y$ of degree $a+d-2$ (resp $b+d-2$).
  We take $A = \id_{r-1}$ and $B = \lambda$ for some $\lambda \in \k^\times$, and show that there exists a $U = (u_{i})$ such that $Mv = v$.
  Indeed, we have $Mv = (p_1, \dots, p_r, q')$, where
  \[ q' = \lambda q + \sum u_{i}p_i. \]
  Let $W \subset H^0(\P^1, \O(a+d-1))$ be the vector space spanned by $p_1, \dots, p_{r-1}$.
  Consider the multiplication map
  \[ H^0(\P^1, \O(b-a)) \otimes W \to H^0(\P^1, \O(b+d-2)).\]
  Thanks to \eqref{eqn:requirement}, the dimension of the source is at least as much as the dimension of the target.
  It is easy to check that the map is in fact surjective for generic $p_1, \dots, p_{r-1}$.
  In particular, there exist $u_i \in H^0(\P^1, \O(b-a))$ for $i = 1, \dots, r-1$, such that
  \[ q(1-\lambda) = \sum u_i p_i.\]
  With this choice of $U = (u_i)$, we get $M$ such that $Mv = v$.

  Finally, note that the requirement \eqref{eqn:requirement} is satisfied for $a = 1$ and $b = k+1$ if $k \geq 1$ and $r \geq 4$.
\end{proof}

\begin{corollary}[\autoref{Thm:Counterexamples}]
  Let $r \geq 3$ and $d \geq r+1$.
  There exist ample vector bundles $E$ of rank $r$ and degree $d$ on $\P^1$ such that for $X = \P E$ and the complete linear series $L = \O_X(1)$, the projection-ramification map $\rho_X$ is not generically finite onto its image.
\end{corollary}
\begin{proof}
  Take $E$ such that the action of $\Aut(X/\P^1)$ on a generic point of $|K_X\otimes L^{r+1}|$ has a positive-dimensional stabilizer (see \autoref{prop:specialE}).
  Since $\rho_X \from \Gr(r+1, H^0(X, L)) \dashrightarrow |K_X \otimes L^{r+1}|$ is equivariant with respect to the action of $\Aut(X/\P^1)$, and a generic
  point of the source does not have a positive-dimensional stabilizer (see \autoref{prop:trivialStabilizer}), it follows that $\rho_X$ cannot be dominant.
  Since the dimension of the source and target of $\rho_X$ are the same, $\rho_X$ is not generically finite.
\end{proof}

\begin{remark}
In all the examples of scrolls where we know that maximal variation fails, the failure can be explained by the presence of generic stabilizers.
We do not know, however, if this is the only reason for the failure of maximal variation.
\end{remark}

\begin{remark}
  If $k = 1$ and $r \geq 4$, then $X$ is the most balanced scroll of its degree and rank, and hence, generic in moduli.
  Therefore, the non-dominance of projection-ramification is not directly connected to the eccentricity of the splitting type of a scroll. 
\end{remark}

\section{Maximal variation for generic scrolls}
In this section, we establish that the projection-ramification map is generically finite (equivalently, dominant) for most scrolls, notwithstanding the examples provided by \autoref{Thm:Counterexamples}.
We begin by treating the cases of some particular scrolls by hand.
We then bootstrap these to more general results using degeneration arguments.

\subsection{Maximal variation for some particular cases}

Given an ample vector bundle $E$ on $\P^1$, we say that \emph{maximal variation holds for $E$} if the projection-ramification map is generically finite (equivalently, dominant) for $X = \P E$ embedded by the complete linear series associated to $L = \O_X(1)$.

\begin{proposition}\label{prop:segre}
  Maximal variation holds for $E = \O(1)^r$.
  In fact, the degree of the projection-ramification map in this case is $1$.
\end{proposition}
\begin{proof}
  We know that the projection-ramification map
  \[ \rho \from \Gr(r+1, H^0(\P^1, \O(1)^r)) \dashrightarrow \P H^0(\P^1, \O(r-1)^r)^*\]
  is $\Aut \P E$ equivariant.
  In this case, it is easy to check that the action of $\Aut (\P E / \P^1) = \PGL_r$ has a unique open orbit and trivial generic stabilizers on both the source and the target of $\rho$.
  Hence, $\rho$ must be birational.  
\end{proof}


\begin{proposition}\label{prop:222}
  Maximal variation holds for $E = \O(2)^r$.
\end{proposition}
Compared to \autoref{prop:segre}, our proof of \autoref{prop:222} is significantly more involved, and does not yield the degree.
\begin{proof}
  We exhibit a point $\Gr(r+1, H^0(\P^1, E))$ at which $\rho$ is defined, and at which the induced map $d\rho$ on the tangent space is non-singular.
  It follows that $\rho$ is a local isomorphism at this point, and hence dominant overall.

  Our proof is by direct calculation.
  We calculate on $\A^1 = \spec \k[x] \subset \P^1$ and identify $\O(n)$ with $\O(n \cdot \infty)$.
  Then the global sections of $\O(n)$ are identified with polynomials in $x$ of degree at most $n$.
  Denote the generator of the $i$th summand of $E(-2)$ by $X_i$.
  Consider the point of $\Gr(r+1, H^0(\P^1, E))$ represented by the vector space $V \subset H^0(\P^1, E)$ spanned by the $(r+1)$ sections $v_1, \dots, v_{r+1}$ defined as follows.
  Set $v_i = (x-a_i)^2 X_i$ for $0 \leq i \leq r-1$, and $v_r = \sum p_i X_i$, where $a_i \in \k$, and $p_j \in H^0(\P^1, \O(2))$ are generic.
  By \eqref{eqn:Rmatrix}, the ramification divisor associated to $V$ is cut out by the determinant of the matrix
  \[
    M =
    \begin{pmatrix}
      (x-a_1)^2 & 0 &  \cdots & 0 & 2(x-a_1)X_1\\
      0 & (x-a_2)^2 & \cdots & 0 & 2(x-a_2)X_2\\
      0 & 0 & \ddots & 0 & \vdots\\
      0 & 0 & \cdots & (x-a_r)^2& 2(x-a_r)X_r\\
      p_1 & p_2 & \cdots & p_r & \sum p_i' X_i
    \end{pmatrix}.
  \]
  We leave it to the reader to check that $R = \det M$ is not identically zero.

  To do the tangent space computation, we choose elements $w_i \in H^0(\P^1, E)$, and change $v_i$ to $v_i + \epsilon w_i$, where $\epsilon^2 = 0$.
  Let $R_\epsilon$ be the equation of the discriminant of the projection given by $V_\epsilon \subset H^0(\P^1, E) \otimes \k[\epsilon]/\epsilon^2$, where $V_\epsilon$ is spanned by $v_1 + \epsilon w_1, \dots, v_{r+1} + \epsilon w_{r+1}$.
  Concretely, $R_\epsilon$ is the determinant of a matrix $M_\epsilon$ given by \eqref{eqn:Rmatrix}, which reduces to $M$ modulo $\epsilon$.
  Note that $R_\epsilon$ is an element of $H^0(\P^1, E \otimes \O(2r-2)) \otimes \k[\epsilon]/\epsilon^2$, and we have
  \[ R_\epsilon =  R + \epsilon S(w_1, \dots, w_{r+1}),\]
  for some $S(w_1, \dots, w_{r+1}) \in H^0(\P^1, E \otimes \O(2r-2))$.
  Furthermore, the map
  \begin{equation}\label{eqn:mainmap}
    S \from H^0(\P^1, E)^{r+1} \to H^0(\P^1, E \otimes \O(2r-2))
  \end{equation}
  is a linear map.
  To show that $d \rho$ is non-singular at $V$, it suffices to show that $S$ is surjective.
  For $1 \leq i \leq r$ and $1 \leq j \leq r+1$, let $E_{i,j} \in H^0(\P^1, E)^{r+1}$ be the element corresponding to $(w_1, \dots, w_{r+1})$ where $w_j = X_i$ and $w_\ell = 0$ for all $\ell \neq j$.
  For $i \neq j$ and $1 \leq j \leq r$ and $q \in H^0(\P^1, \O(2))$, by direct calculation we get
  \[ S\left(qE_{i,j}\right) = \frac{(x-a_1)^2 \cdots (x-a_r)^2p_j}{(x-a_i)^2(x-a_j)^2} \cdot [q, (x-a_i)^2] \cdot X_i,\]
  where the notation $[a,b]$ means $a'b-ab'$.
  Similarly, we get
  \[ S\left(qE_{i,r+1}\right) = - \frac{(x-a_1)^2 \cdots (x-a_r)^2}{(x-a_i)^2} \cdot [q, (x-a_i)^2] \cdot X_i,\]
  and
  \begin{equation}\label{eqn:diag}
    S\left(qE_{i,i} \right) = \det M_i,
  \end{equation}
  where $M_i$ is obtained from $M$ by changing the $(i,i)$-th entry from $(x-a_i)^2$ to $q$ and the $(i,r+1)$-th entry from $2(x-a_i)X_i$ to $q'X_i$.

  Fix an $i$ with $1 \leq i \leq r$, and consider the subspace $W_i \subset H^0(\P^1, E)^{r+1}$ spanned by $q E_{i,j}$ for $j \neq i$.
  By our calculations above, $S$ maps $W_i$ to the subspace of $H^0(\P^1, E \otimes \O(2r-2))$ spanned by $H^0(\P^1, \O(2r)) \otimes X_i$.
  We begin by identifying $S(W_i)$.

  For $1 \leq j \leq r$ and $j \neq i$, set
  \[
    Q_{i,j} = \frac{(x-a_1)^2 \cdots (x-a_r)^2p_j}{(x-a_i)^2(x-a_j)^2}, 
  \]
  and
  \[
    Q_{i,r+1} = - \frac{(x-a_1)^2 \cdots (x-a_r)^2}{(x-a_i)^2}.
  \]
  We claim that, there is no non-trivial linear relation among the $r$ polynomials $Q_{i,j}$ for $j \in \{1, \dots, r+1\} \setminus \{i\}$.
  Indeed, suppose we had a linear relation
  \[ \sum l_j Q_{i,j} = 0,\]
  then dividing throughout by $\frac{(x-a_1)^2\cdots (x-a_r)^2}{(x-a_i)^2}$ gives the relation
  \[ \sum_{j = 1}^r l_j \frac{p_j}{(x-a_j)^2} + l_{r+1} = 0.\]
  If $l_j \neq 0$ for some $j$ with $1 \leq j \leq r$, then we have a pole on the left side at $x = a_j$, but not on the right side (note that $(x-a_j)$ does not divide $p_j$ by the genericity of $p_j$).
  Therefore, we must have $l_j = 0$ for all $j$, and hence also $l_{r+1} = 0$.
  Consider the map
  \begin{equation}\label{eqn:big}
    H^0(\P^1, \O(1)) \otimes \langle  Q_{i,j} \mid j \in \{1, \dots, r+1\} \setminus \{i\}\rangle \to H^0(\P^1, \O(2r-1)).
  \end{equation}
  We just saw that this map is injective.
  But both sides have the same dimension, and hence the map must be surjective.
  Finally, it is easy to see that the image of the map
  \begin{equation}\label{eqn:q}
    H^0(\P^1, \O(2)) \to H^0(\P^1, \O(2)), \quad q \mapsto [q, (x-a_i)^2]
  \end{equation}
  is $(x-a_i)\cdot H^0(\P^1, \O(1))$.
  By \eqref{eqn:big} and \eqref{eqn:q}, we conclude that the image of the map
  \[ S \from W_i = \langle qE_{i,j} \mid j \in\{1, \dots, r+1\} \setminus \{i\} \to H^0(\P^1, \O(2r-1)) \otimes X_i\]
  is $(x-a_i)H^0(\P^1, \O(2r-2)) \otimes X_i$.
  In other words, the cokernel of the map is $X_i \otimes \k$ where the map
  \[H^0(\P^1, \O(2r)) \otimes X_i \to \k \otimes X_i \]
  is given by evaluation at $a_i$.
  Putting together the maps for various $i$, we see that the cokernel of the map
  \[ S \from \bigoplus_i W_i \to H^0(\P^1, E \otimes \O(2r-2)) = H^0(\P^1, \O(2r)) \otimes \langle  X_1, \dots, X_r \rangle\]
  is $\k \otimes \langle  X_1, \dots, X_r \rangle$, where the map
  \begin{equation}\label{eqn:partialsur}
    H^0(\P^1, E \otimes \O(2r-2)) = H^0(\P^1, \O(2r)) \otimes \langle  X_1, \dots, X_r \rangle \to \k \otimes \langle  X_1, \dots, X_r \rangle
  \end{equation}
  on $H^0(\P^1, \O(2r)) \otimes X_i$ is given by evaluation at $a_i$.

  To show that $S$ is surjective, it is now enough to show that the map
  \begin{equation}\label{eqn:remainsur}
    H^0(\P^1, \O(2)) \otimes \langle  E_{i,i} \mid i \in \{1, \dots, r+1\} \rangle \to \k \otimes \langle  X_1, \dots, X_r \rangle
  \end{equation}
  obtained by composing \eqref{eqn:mainmap} and \eqref{eqn:partialsur} is surjective.
  Recall from \eqref{eqn:diag} that we have $S(qE_{i,i}) = \det M_i$, where $M_i$ is obtained from $M$ by changing the $(i,i)$-th entry to $q$ and the $(i, r+1)$-th entry to $q'X_i$.
  Taking $q = (x-a_i)$ gives
  \[ S(qE_{i,i}) = \det M_i = \pm \prod_{j \neq i} (a_i-a_j)^2 p_i(a_i) X_i,\]
  which is a non-zero multiple of $X_i$.
  That is, the images of $(x-a_i)E_{i,i}$ under $S$ span $\k \otimes \langle  X_1, \dots, X_r \rangle$, and hence the map in \eqref{eqn:remainsur} is surjective.
  The proof is now complete.
\end{proof}

Our next goal is to bootstrap from \autoref{prop:segre} and \autoref{prop:222} to deduce maximal variation for generic scrolls of sufficiently high degree.
We do this by a degeneration argument.
We degenerate a vector bundle $E$ to a vector bundle $E_0$ on the nodal rational curve $P_0 = \P^1 \cup \P^1$, and show that the projection-ramification map for $E_0$ is dominant.
For this to work, we have to define the projection-ramification map for nodal curves.
It turns out that with the most n\"aive definition of linear series on scrolls on nodal curves, we do not get a dominant projection-ramification map.
We have to work with the limit linear series of higher rank as developed in \cite{tei:} and\cite{oss:14}.

\subsection{Limit linear series}\label{sec:lls}
We need limit linear series for the simplest singular curve, namely a (projective, connected) nodal curve $C$ which is the nodal union of two smooth (projective, connected) curves $C_1$ and $C_2$, but we need them for vector bundles of rank higher than $1$.
Let $B$ be the spectrum of a DVR with special point $0$, general point $\eta$.
Let $\pi \from X \to B$ be a smoothing of $C$ with non-singular total space $X$.
That is, $\pi$ is a flat, proper, family of connected curves, smooth over $\eta$, and isomorphic to $C$ over $0$.
Such a family is a particularly simple example of an almost local smoothing family \cite[\S~2.1--2.2]{oss:14}.
Let $g_i$ be the genus of $C_i$ for $i = 1, 2$, and $g = g_1+g_2$ the genus of $X_\eta$.

Let $E$ be a vector bundle of rank $r$ on $C$.
The \emph{multi-degree} of $E$ is the pair of integers $(\deg E|_{C_1}, \deg E|_{C_2})$.
The \emph{degree} or \emph{total degree} of $E$ is the sum $\deg E = \deg E|_{C_1} + \deg E|_{C_2}$.

Once and for all, fix a vector bundle $\mathcal E$ of rank $r$ on $X$, and set $E = \mathcal E|_C$.
Let $E$ have degree $d$ and multi-degree $(w_1, w_2)$.
Fix a positive integer $k$.
Our next task is to recall the definition of the space of limit linear series of dimension $k$.
It will be a $B$-scheme whose fiber over $\eta$ is the Grassmannian $\Gr(k, H^0(X_\eta, \mathcal E_\eta))$.
The key idea is to not only consider the sections of $\mathcal E$, but also of its various twists, namely the vector bundles obtained by tensoring with the powers of $\O_X(C_i)$.

Fix maps $\theta_1 \from \O_X \to \O_X(C_1)$ and $\theta_2 \from \O_X \to \O_X(C_2)$.
The choice of these maps is auxilliary, and each one is unique up to multiplication by an element of $\O_B^*$.
For $n \in \Z$, set
\[ \mathcal E_n =
  \begin{cases}
    \mathcal E \otimes \O_X(C_1)^{\otimes n} & \text{if $n \geq 0$},\\
    \mathcal E \otimes \O_X(C_2)^{\otimes (-n)}  & \text{if $n < 0$}.
  \end{cases}
\]
The maps $\theta_1$ and $\theta_2$ induces maps
\[ \theta_n \from \mathcal E_m \to \mathcal E_{m+n}\]
given by
\[
  \theta_n = 
  \begin{cases}
    \theta_1^n & \text{if $n \geq 0$,} \\
    \theta_2^{-n} & \text{if $n < 0$.}
  \end{cases}
\]
Note that the multi-degree of $\mathcal E_n$ is $(w_1 - nr, w_2 + nr)$.
In particular, for sufficiently negative $n$, say for $n \leq n_1$, we have $H^0(C_2, \mathcal E_n|_{C_2}) = 0$, and similarly, for sufficiently positive $n$, say $n \geq n_2$, we have $H^0(C_1, \mathcal E_n|_{C_1}) = 0$.
Assume, without loss of generality, that $n_2 \geq n_1$.
Set
\[ d_1 = w_1 - n_1 r, \text{ and } d_2 = w_2 + n_2 r, \text{ and } b = n_2 - n_1.\]
Observe that
\[ d_1 + d_2 - rb = d.\]
We say that $\mathcal E$ has multi-degree $w$ if for every $s \in S$ mapping to $0 \in B$, the degree of $\mathcal E|_s$ on $C_v$ is $w_v$ for $v = 1, 2$.
Note that, if $\mathcal E$ has multi-degree $(w_1, w_2)$, then $\mathcal E_n$ has multi-degree $(w_1-rn, w_2+rn)$.

\begin{definition}[Limit linear series]
  \label{def:lls}
  Let $S$ be a $B$-scheme.
  A \emph{$k$-dimensional limit linear series} on $\mathcal E_S$ consists of sub-bundles $V_n \to \pi_* (\mathcal E_n)_S$ of rank $k$ for every $n \in \Z$ satisfying the following compatibility condition.
  For every $m, n \in \Z$, the map
  \begin{equation}\label{lls:compatibility}
    \pi_* \theta_n \from \pi_* (\mathcal E_m)_S \to \pi_* (\mathcal E_{m+n})_S \text{ maps } V_m \to V_{m+n}.
  \end{equation}
\end{definition}
\autoref{def:lls} is a special case of \cite[Definition~3.3.2]{oss:14}.
From now on, we will talk about the image of an element in $V_m$ in $V_{m+n}$; this should be understood as the image under the map $\pi_* \theta_n$.

\begin{remark}
The notion of a sub-bundle of a push-forward is a bit subtle; it is treated in depth in \cite[Definition~B.2.1]{oss:14}.
We recall the main points.
For a flat proper morphism $X \to S$ and a vector bundle $\mathcal E$ on $S$, a \emph{sub-bundle} of $\pi_* \mathcal E$ is a vector bundle $V$ on $S$ along with a map $i \from V \to \pi_* \mathcal E$ such that for every $T \to S$, the pull-back $i_T \from V_T \to \pi_* (\mathcal E_T)$ is injective.
Note that this is a local condition on $S$.
For Noetherian schemes such as ours, it is enough to check this condition for the $T \to S$ that are inclusions of closed points.
Alternatively, if $F_0 \to F_1 \to \cdots $ is a complex of vector bundles on $S$ quasi-isomorphic to $R\pi_* \mathcal E$, then a sub-bundle of $\pi_* \mathcal E$ is a vector bundle $V$ along with a map $i \from V \to \pi_* \mathcal E$ such that the composite $V \to F_0$ is an injection of vector bundles (that is, the dual map is surjective).
\end{remark}


\begin{remark}
\autoref{def:lls} defines limit linear series on a particular vector bundle $\mathcal E$.
We can also vary the choice of the vector bundle, as is done in \cite{oss:14}; in that case, one imposes an additional vanishing condition on the vector bundles to ensure boundedness of the moduli space of limit linear series.
\end{remark}


\begin{definition}\label{def:simple_lls}
Let $S = \spec K$, where $K$ is a field, and let $V = (V_n \mid n \in \Z)$ be a limit linear series on $S$.
We say  $V$ is \emph{simple} if there exist integers $w_1, \dots, w_k$, not necessarily distinct, and elements $v_i \in V_{w_i}$ such that for every $w \in \Z$, the images of $v_1, \dots, v_k$ in $V_w$ form a basis of $V_w$.
\end{definition}

Note that if $S \to B$ maps to the generic point $\eta$, then the data of a limit linear series $V = (V_n)$ is equivalent to the data of an individual $V_n$ for any $n \in \Z$, and in particular, for $n = 0$.
As a result, the functor that associates to $S \to \eta$ the set of $k$-dimensional limit linear series of $\mathcal E_S$ is represented by the Grassmannian $\Gr(k, H^0(X_\eta, \mathcal E_\eta))$.
The main theorem of \cite{oss:14} is the following representability theorem.
\begin{theorem}[{\cite[Theorem~3.4.7]{oss:14}}]
  \label{thm:lls}
  The functor that associates to a $B$-scheme $S \to B$ the set of limit linear series on $\mathcal E_S$ is representable by a projective $B$-scheme $\mathcal G(k, \mathcal E)$ isomorphic to the Grassmannian $\Gr(k, H^0(X_\eta, \mathcal E_\eta))$ over $\eta$.
  The locus of simple linear series ${\mathcal G}^{\rm simple}(k, \mathcal E) \subset {\mathcal G}(k, \mathcal E)$ is an open subscheme, and the map ${\mathcal G}^{\rm simple}(k, \mathcal E) \to B$ has universal relative dimension at least $k(d-k-r(g-1))$.
\end{theorem}
The last stamement implies that if $v \in {\mathcal G}^{\rm simple}$ is such that ${\mathcal G}^{\rm simple}$ has relative dimension at most $k(d-k-r(g-1))$ at $v$, then it has relative dimension exactly $k(d-k-r(g-1))$ at $v$ and, futhermore, it is an open map near $v$.
In particular, $v$ is in the closure of $\Gr(k, H^0(X_\eta, \mathcal E_\eta))$.
\begin{remark}
  Osserman proves a stronger theorem, namely a relative version of the statement above, over the stack of vector bundles on $X$.
  But the statement above is enough for our purposes.  
\end{remark}

Although the definition of a limit linear series demands that we specify infinitely many vector bundles $V_n$, one for each $n \in \Z$, this is neither practical nor necessary.
In the best case, only specifying the extremal ones, namely $V_{n_1}$ and $V_{n_2}$, suffices, provided that they satisfy some compatibility conditions.
The original definition of limit linear series due to Eisenbud--Harris \cite{eis.har:86} in the rank 1 case and Teixidor i Bigas \cite{tei-i-big:91} in the general case, took this minimalist approach.

Let $E_n$ be the restriction of $\mathcal E_n$ to the central fiber $C = X_0$, and set $p = C_1 \cap C_2$.
\begin{definition}
  \label{def:eht}
  A \emph{$k$-dimensional EHT limit linear series} on $E$ consists of $k$-dimensional subspaces $W_i \subset H^0(C_i, E_{n_i}|_{C_i})$ for $i = 1, 2$ that satisfy the following two conditions.
  \begin{enumerate}
  \item
    \label{ieq:eht}
    If $a^i_1 \leq \cdots \leq a^i_k$ is the vanishing sequence for $(\mathcal E_{n_i}|_{C_i}, W_i)$ at $p$ for $i = 1, 2$, then for every $v = 1, \dots, k$ we have
    \[ a^1_v + a^2_{k+1-v} \geq b.\]
  \item\label{gluing:eht}
    There exist bases $s^i_1, \dots, s^i_k$ for $W_i$ for $i = 1, 2$, such that $s^i_v$ has order of vanishing $a^i_v$ at $p$, and if we have $a^1_v + a^2_{k+1-v} = b$ for some $v$, then
    \[ \widetilde \phi (s^1_v) = s^2_{k+1-v},\]
    where $\widetilde \phi \from E_{n_1}(-a^1_{v}\cdot p)|_p \to E_{n_2}(-a^2_{k+1-v} \cdot p) |_p$ is the isomorphism obtained by taking the appropriate twist of the identity map.
  \end{enumerate}
  We say that $(W_1, W_2)$ is a \emph{refined EHT limit linear series} if all equality holds in \eqref{ieq:eht} for all $v = 1, \dots, k$.
\end{definition}
This definition is adapted from \cite[Definition~4.1.2]{oss:14}.
Note that, due to the vanishing condition on the twists of $E$, the restriction map
\[ H^0(C, E_{n_i}) \to H^0(C_i, E_{n_i}|_{C_i})\]
is an isomorphism.
Via this isomorphism, we sometimes treat $W_i$ as a subspace of $H^0(C_i, \mathcal E_{n_i}|_{C_i})$.

Although the notions of a limit linear series and an EHT limit linear series differ in general, they essentially agree when we restrict to the simple limit linear series and the refined EHT limit linear series.
More precisely, we have the following statement.
\begin{proposition}\label{prop:llseht}
  Let $S$ be a $B$-scheme, and $V = (V_n \mid n \in \Z)$ a limit linear series on $\mathcal E_S$.
  For every $s \in S$ over $0 \in B$, taking $W_i = V_{n_i}|_s$ for $i = 1, 2$ gives an EHT limit linear series.
  Conversely, assume that $S$ reduced, and let $\mathcal W_i \subset \pi_*(\mathcal E_{n_i})_S$ be sub-bundles whose restrictions to every $s \in S$ over $\eta \in B$ agree under the isomorphism $(\mathcal E_{n_1})_\eta \cong (\mathcal E_{n_2})_\eta$, and to eveny $s \in S$ over $0 \in B$ define a refined EHT limit linear series.
  Then there exists a unique limit linear series $V = (V_n \mid n \in \Z)$ on $\mathcal E_S$ such that $\mathcal W_i = V_{n_i}$.
  Furthermore, for every $s \in S$ over $0$, the series $V|_s$ is simple.
\end{proposition}
\begin{proof}
  Proving that $(W_1, W_2)$ is an EHT limit linear series is straightforward, and left to the reader.
  It is a special case of \cite[Theorem~4.3.4]{oss:14} and the equivalence of type I and type II series in the two component case (\cite[Remark~3.4.15]{oss:14}.

  The converse also follows from the proof of \cite[Theorem~4.3.4]{oss:14}, but it is not explicitly stated there.
  So we offer a proof.
  
  First, suppose that $S$ lies over $\eta \in B$.
  Then $V_n \subset \pi_* (\mathcal E_n)_S$ is determined uniquely as the image of $V_{n_i} = \mathcal W_{n_i} \subset \pi_* (\mathcal E_{n_i})_S$ for either $i = 1$ or $i = 2$.

  Next, suppose that $S = \spec K$, and it lies over $0 \in B$.
  Denoting $(\mathcal E_n)_S$ by $E_n$, we must construct $V_n \subset H^0(C, E_n)$.
  By composing $\theta_{n_i-n} \from E_n \to E_{n_i}$ and the restriction $E_{n_i} \to E_{n_i}|_{C_i}$, we get a map
  \[ \iota \from H^0(C, E_n) \to H^0(C_1, E_{n_1}|_{C_1}) \oplus H^0(C_2, E_{n_2}|_{C_2}). \]
  The vanishing condition on the twists of $E$ mean that $\iota$ is injective.
  The compatibility condition in \autoref{def:lls} implies that we must choose $V_n$ so that $\iota (V_n) \subset W_1 \oplus W_2$.
  We claim that $\dim \iota^{-1}(W_1 \oplus W_2) = k$, so that there is a unique choice of $V_n$, namely $V_n = \iota^{-1}(W_1 \oplus W_2)$.

  Suppose $s \in i^{-1}(W_1 \oplus W_2)$.
  Then $\iota(s)$ is a linear combination of $(s^1_1,0), \dots, (s^1_k, 0)$, and $(0, s^2_1), \dots, (0,s^2_k)$.
  Write $\iota(s) = (s_1, s_2)$.
  Since $s_i$ is obtained by applying $\theta_{n-n_i}$, and $\theta$ on $C_i$ at $p$ corresponds to multiplication by the uniformizer, we see that
  \begin{equation}\label{eqn:vanishing}
    \ord_p(s_1) \geq n - n_1, \text{ and likewise, } \ord_p(s_2) \geq n_2 - n.
  \end{equation}
  Let $v_1 \in \{1, \dots, k\}$ be the smallest such that $a^1_v \geq n-n_1$, and $v_1+c$ the smallest such that $a^1_{v_1+c} > n-n_1$.
  Since $(W_1, W_2)$ is refined, and $n_2 - n_1 = b$, we see that $v_2 = k+1-v_1$ is the largest such that $a^2_{v_2} \leq n_2 - n$, and $v_2 - c$ the smallest such that $a^2_{v_2 + c} < n_2 - n$.
  The vanishing conditions \eqref{eqn:vanishing} imply that $\iota(s)$ bust be a linear combination of $(s^1_{v_1},0), \dots, (s^1_k,0)$ and $(0, s^2_{v_2-c}), \dots,  (0,s^2_k)$.
  Suppose
  \[ \iota(s) = \sum_{\ell = i}^k \alpha_{\ell} \cdot (s^1_\ell,0) + \sum_{\ell = j}^{k} \beta_\ell \cdot (0,s^2_\ell),\]
  where $\alpha_{\ell}$ and $\beta_{\ell}$ are elements of the field $K$.
  Since $s$ is a section on the entire nodal curve $C$, its two restrictions to $C_1$ and $C_2$ are equal at $p$.
  In terms of the two components of $\iota(s)$, and in light of the gluing condition \eqref{gluing:eht} in \autoref{def:eht}, this equality is equivalent to $\alpha_{\ell} = \beta_{k+1-\ell}$.
  That is, $\iota(s)$ is a linear combination of the $k$ elements 
  \[ (s^1_{v_1} , s^2_{v_2}), \dots, (s^1_{v_1+c-1} , s^2_{v_2-c+1}), (s^1_{v_1+c},0), \dots, (s^1_k,0), (0,s^2_{v_2+1}), \dots, (s^2_{k},0).\]  
  Conversely, it is easy to see that any such linear combination lies in $W_1 \oplus W_2$.
  Hence the claim that $\dim \iota^{-1}(W_1 \oplus W_2) = k$.

  Set $V_n = \iota^{-1}(W_1 \oplus W_2)$.
  To see that $V$ is simple, we must exhibit appropriate $w_i$ and $v_i \in V_{w_i}$ for $i = 1, \dots, k$.
  Take $w_i = n-n_1-a^1_i$, and let $v_i \in V_{w_i} \subset H^0(C, E_{w_i})$ be such that $\iota(v_i) = (s^1_i, s^2_{k+1-i})$.
  Then the images of $v_1, \dots, v_k$ form a basis of $V_n$ for all $n \in \Z$.

  For more general $S$, consider the map
  \[ \overline \iota \from \pi_* (\mathcal E_n)_S \to \pi_*(\mathcal E_{n_1})_S / \mathcal W_1 \oplus \pi_*(\mathcal E_{n_2})_S / \mathcal W_2,\]
  obtained by composing $\iota = \pi_*(\theta_{n_1-n} \oplus \theta_{n_2-n})$ and the projections $\pi_*(\mathcal E_{n_i})_S \to \pi_*(\mathcal E_{n_i})_S / \mathcal W_i$.
  We proved that, for every $\spec K \to S$, the kernel of $\overline \iota \otimes_{\O_S} K$ is $k$-dimensional.
  Since $S$ is reduced, it is easy to prove that $V_n = \ker \iota$ is a sub-bundle of $\pi_*(\mathcal E_n)$.
  It is also easy to check that $V = (V_n \mid n \in \Z)$ a limit linear series, the only one that satisfies $V_{n_i} = \mathcal W_i$.
  The proof is now complete.
\end{proof}

\autoref{prop:llseht} allows us to combine the economy of specifying an EHT limit linear series with the convenient functorial definition of a limit linear series.
We use this in the definition of the projection-ramification map in terms of limit linear series.

\subsection{Projection-ramification with non-generic vanishing sequence}
\label{sec:prnodal}
We consider the projection-ramification map for linear series with a non-generic vanishing sequence.
The analysis of such series plays a key role in defining the projection-ramification map for limit linear series.

Let $C$ be a smooth curve and $p \in C$ a point.
Let $E$ be a vector bundle on $C$ of rank $r$.
The projective spaces associated to the vector spaces $E(np)|_p$, for $n \in \Z$, are canonically isomorphic to each other, so we identify them.
The vanishing sequences considered are at the point $p$.
Choose a uniformizer $t$ of $C$ at $p$.

Suppose $V \subset H^0(C, E)$ is an $(r+1)$-dimensional subspace with the vanishing sequence 
\begin{equation}\label{eqn:specialvs}
  (\underbrace{a, \dots, a}_{i}, \underbrace{a+1, \dots, a+1}_{r+1-i}),
\end{equation}
for some $i$ with $1 \leq i \leq r$, and $a \geq 0$.
Let $v_1, \dots, v_{r+1}$ be a basis of $V$ adapted to the vanishing sequence, namely a basis $v_1, \dots, v_{r+1}$ such that in the stalk $E_p$, we can write
\begin{equation}\label{eqn:basis}
  v_1 = t^a \widetilde v_1, \dots, v_{i} = t^a \widetilde v_i,\quad v_{i+1} = t^{a+1} \widetilde v_{i+1}, \dots, v_{r+1} = t^{a+1} \widetilde v_{r+1},
\end{equation}
for some $\widetilde v_1, \dots, \widetilde v_{r+1} \in E_p$ such that the images of $\widetilde v_1, \dots, \widetilde v_i$ in the fiber $E|_p$ are linearly independent, and the same holds for the images of $\widetilde v_{i+1}, \dots, \widetilde v_{r+1}$.
Here we are slightly abusing the notation by denoting $v_i$ and its image in $E_p$ under the natural evaluation map by the same letter.
Let $V^0 \subset E|_p$ be spanned by the images of $\widetilde v_1, \dots, \widetilde v_i$, and $V^1 \subset E|_p$ by the images of $\widetilde v_{i+1}, \dots, \widetilde v_{r+1}$.
It is easy to check that a different choice of basis adapted to the vanishing sequence gives the same $V^0$ and $V^1$.
By construction, $\dim V_0 = i$ and $\dim V^1 = r+1-i$, and therefore, $\dim (V^0 \cap V^1) \geq 1$.
We say that $V$ has \emph{transverse vanishing} at $p$ if 
\begin{equation}\label{eq:genericity}
  \dim (V^0 \cap V^1) = 1.
\end{equation}
Note that if $V$ is base-point free at $p$, then $\dim V^0 = r$ and $\dim V^1 = 1$, so $V$ automatically has transverse vanishing.

\begin{proposition}\label{prop:agreement}
  Suppose $V \subset H^0(C, E)$ is an $(r+1)$-dimensional subspace with vanishing sequence \eqref{eqn:specialvs} and transverse vanishing at $p$.
  Then the ramification section $r_V$ of $V$ vanishes to order $(r+1)a + (r-i)$ at $p$.
  Furthermore, writing $r_V = t^{(r+1)a+r-i} \cdot \widetilde r$, the one-dimensional subspace of $E|_p$ spanned by $\widetilde r|_p$ is $V^0 \cap V^1$.
\end{proposition}
\begin{proof}
  Thanks to transverse vanishing, there exists a basis $\{\overline s_1, \dots, \overline s_{r} \}$ of $E|_p$ such that
  \[ V^0 = \langle  \overline s_1, \dots, \overline s_i \rangle \text{ and } V^1 = \langle  \overline s_{i+1}, \dots, \overline s_r, \overline s_1 \rangle.\]
  Let $v_1, \dots, v_{r+1}$ be a basis of $V$ adapted to the vanishing sequence such that if $\widetilde v_i$ are defined as in \eqref{eqn:basis} then the images of $\widetilde v_1, \dots, \widetilde v_r$ in $E|_p$ are $\overline s_1, \dots, \overline s_r$, respectively, and the image of $\widetilde v_{r+1}$ is $\overline s_1$.
  In particular, the $r$ elements $\widetilde v_1, \dots, \widetilde v_r \in E_p$ give a trivialization of $E$ around $p$.
  Write
  \[ \widetilde v_{r+1} = b_1 \widetilde v_1 + \dots + b_r \widetilde v_r\]
  in $E_p$, where $b_1, \dots, b_r \in \O_{C,p}$.
  Since the image of $\widetilde v_{r+1}$ in $E|_p$ is $\overline s_1$, we get that $b_1 \equiv 1 \pmod {\mathfrak m_p}$, and $b_2, \dots, b_r \in \mathfrak m_p$.  
  Using the basis $v_1, \dots, v_{r+1}$ of $V$ and the local trivialization $\widetilde v_1, \dots, \widetilde v_r$ of $E$, we can write $r_V$ as the determinant (see \eqref{eqn:Rmatrix}) as follows
  \begin{align*}
    r_V &= \det
    \begin{pmatrix}
      t^a & & & & & &  at^{a-1}\widetilde v_1\\
       & \ddots & & & & &\vdots\\
       & & t^a  & & & &a t^{a-1}\widetilde v_i\\
       & & & t^{a+1}  & & &(a+1)t^a \widetilde v_{i+1} \\
       & & & & \ddots & & \vdots\\
       & & & & &t^{a+1} &(a+1)t^{a}\widetilde v_r\\
      b_1t^{a+1}& b_2 t^{a+1} & \cdots & b_{r-1}t^{a+1} & & b_rt^{a+1} & (a+1)t^a\widetilde v_1 + t^{a+1}(\cdots)
    \end{pmatrix}\\
        &= t^{(r+1)a+r-i} \widetilde v_1  + t^{(r+1)a+r-i+1} (\cdots).
  \end{align*}
  Thus the order of vanishing of $r_V$ is as claimed.
  Furthermore, $\widetilde r$ is given by
  \[ \widetilde r = \widetilde v_1 + t (\cdots).\]
  Since the image of $\widetilde v_1$, namely $\overline s_1$, spans $V^0 \cap V^1$, the proof is complete.
\end{proof}

We are primarily interested in generic $(r+1)$-dimensional subspaces $V \subset H^0(C, E)$.
A generic such $V$ has the vanishing sequence
\[ (0, \dots, 0, 1).\]
For limit linear series, it is important to also study the $V$ with complementary vanishing sequence, namely
\[ (0,1, \dots, 1),\]
which we now do.
For simplicity, we restrict to $C = \P^1$.

Let $E$ be an ample vector bundle on $\P^1$ of rank $r$.
Fix a point $p \in \P^1$; all the vanishing sequences are at $p$.
Consider  the locally closed subset $U \subset \Gr(r+1, H^0(\P^1, E))$ parametrizing $V \subset H^0(\P^1, E)$ with vanishing sequence
\[ (0,\underbrace{1,\dots, 1}_{r}).\]
Given such a $V$, let $\widetilde r_V \in \P H^0(E \otimes \det E \otimes K_{\P^1} \otimes \O(-(r-1)p)^*$ be the reduced ramification section, namely the section obtained by dividing the usual ramification section $r_V$ by the $(r-1)$-th power of a uniformizer at $t$ (see \autoref{prop:agreement}).
The assignment $V \mapsto \widetilde r_V$ gives a variant of the projection-ramification map, which we call the \emph{reduced projection-ramification map}
\begin{equation}\label{eqn:rrd}
  \widetilde \rho \from U \to \P H^0(\P^1, E \otimes \det E \otimes K_{\P^1} \otimes \O(-(r-1)p))^*.
\end{equation}
Note that, just as in the case of the usual projection-ramification map, the source and the target of the reduced projection-ramification map are of the same dimension.

Having defined the reduced projection-ramification map, we now relate it back to the usual projection-ramification map, but on a different vector bundle.
Given a one-dimensional subspace $\ell \subset E|_p$, define $E'_\ell$ by the exact sequence
\[ 0 \to E_\ell' \to E \to E|_p/\ell\to 0.\]
There exists a Zariski open subset of the projective space of lines in $E|_p$ such that for all $\ell$ in this set, the isomorphism class of $E'_{\ell}$ remains constant.
Denote this isomorphism class by $E'_{\rm gen}$.
\begin{proposition}\label{prop:domred}
  If the usual projection-ramification map
  \[ \rho \from \Gr(r+1, H^0(\P^1, E'_{\rm gen})) \dashrightarrow \P H^0(\P^1, E'_{\rm gen} \otimes \det E'_{\rm gen} \otimes K_{\P^1})^*\]
  is dominant, then so is the reduced projection-ramification map
  \[\widetilde \rho \from U \to \P H^0(\P^1, E \otimes \det E \otimes K_{\P^1} \otimes \O(-(r-1)p))^*.\]
\end{proposition}
\begin{proof}
  Let $D \in \P H^0(E \otimes \det E \otimes K_{\P^1} \otimes \O(-(r-1)p))^*$ be a generic section.
  Let $\ell \subset E|_p$ be the one-dimensional subspace defined by $D|_p$, and set $E' = E'_{\ell}$.
  Since $D$ is generic, we may assume $E' \cong E'_{\rm gen}$.
  The inclusion of sheaves $E' \to E$ induces an inclusion of sheaves
  \[
    E' \otimes \det E' \otimes K_{\P^1} \to E \otimes \det E \otimes \O(-(r-1)p) \otimes K_{\P^1},
  \]
  and by construction, $D$ is the image of a section $D' \in \P H^0(E' \otimes \det E' \otimes K_{\P^1})^*$.
  Since $\rho$ is dominant for $E'$, there exists a sequence of subspaces $V_n' \in \Gr(r+1, H^0(\P^1, E'))$ such that the limit of $\rho(V_n')$ is $D'$.
  Let $V_n \subset \Gr(r+1, H^0(\P^1, E))$ be the image of $V'_n$.
  Then the limit of $\widetilde \rho(V_n)$ is $D$.
  Since $D$ was generic, we get that $\widetilde \rho$ is dominant.
\end{proof}

\begin{corollary}\label{prop:domredexamples}
  The reduced projection-ramification map is dominant for the bundles $E = \O(1) \oplus \O(2)^{r-1}$ and $E = \O(2) \oplus \O(3)^{r-1}$.
\end{corollary}
\begin{proof}
  Follows from \autoref{prop:domred} and that the projection-ramification map is dominant for $E' = \O(1)^r$ and $E' = \O(2)^r$.
\end{proof}

\subsection{Projection-ramification for limit linear series}
Recall the setup from \autoref{sec:lls}: $C = C_1 \cup C_2$ is a nodal union of two smooth projective curves of genus $g_1$ and $g_2$, and $\pi \from X \to B$ be a smoothing of $C$.
Let $\mathcal E$ be a vector bundle of rank $r$ on $X$ whose restriction $E$ to $C$ has multi-degree $(w_1, w_2)$.
The integers $n_2 \geq n_1$ are such that we have vanishing $H^0(C_2, E_n|_{C_2}) = 0$ for all $n \leq n_1$ and $H^0(C_1, E_n|_{C_1}) = 0$ for $n \geq n_2$.
For convenience, we decrease $n_1$ and increase $n_2$ so that the vanishing on $C_2$ holds for all $n \leq n_1 - (w_1-2g_1)$ and on $C_1$ for all $n \geq n_2 + (w_2-2g_2)$.
Define
\[ d_1 = w_1 - n_1r, \quad d_2 = w_2 + n_2r,\text{ and } b = n_2 - n_1,\]
as before.

Set $\mathcal E' = \mathcal E \otimes \det \mathcal E \otimes \omega_{X/B}$.
Then $\mathcal E'$ is a vector bundle of rank $r$ on $X$ whose restriction $E'$ to $C$ has multi-degree $(w_1', w_2')$ where
\[ w_1' = w_1 + r(w_1-2g_1+1) \text{ and } w'_2 = w_2 + r(w_2-2g_2+1).\]
We set
\[ n_1' = n_1(1+r) \text{ and } n_2' = n_2(1+r),\]
and observe that we have vanishings $H^0(C_2, E'_{n}|_{C_2}) = 0$ for $n \leq n_1'$ and $H^0(C_1, E'_{n}|_{C_1}) = 0$ for $n \geq n_2'$.
We also set
\[ b' = n_2' - n_1' = b(1+r).\]

Our next goal is to define a rational map
\begin{equation}\label{eq:Rtilde}
  \rho \from {\mathcal G}(r+1, \mathcal E) \dashrightarrow {\mathcal G}(1, \mathcal E')
\end{equation}
that extends the projection-ramification map
\[
  \rho \from \Gr(r+1, H^0(X_\eta, \mathcal E_\eta)) \dashrightarrow \Gr(1, H^0(X_\eta, \mathcal E'_\eta))
\]
on $X_\eta$.
For technical reasons, we define the map in \eqref{eq:Rtilde} only on the reduced scheme underlying ${\mathcal G}(r+1, \mathcal E)$.

Before defining the map, we identify three conditions on limit linear series on the central fiber that are required for the map to be defined.
To do this, consider a limit linear series $(V_n \mid n \in \Z)$ on $C$, and let $(W_1, W_2)$ be the associated EHT limit linear series namely $W_1 = V_{n_1}$ and $W_2 = V_{n_2}$ (see \autoref{prop:llseht}).
The first condition we want to impose is that $(W_1, W_2)$ be a refined EHT limit linear series; this is an open condition (see \cite[Proposition~4.1.5]{oss:14}).
The second condition we want to impose is that the vanishing sequence of $W_1 \subset H^0(C_1, E_{n_1}|_{C_1})$ at $p$ is of the form
\begin{equation}\label{eqn:llsvs}
  (\underbrace{a, \dots, a}_i, \underbrace{a+1, \dots, a+1}_{r+1-i})
\end{equation}
as in \eqref{eqn:specialvs}; imposing a particular vanishing sequence is again an open condition (see \cite[Proposition~4.2.5]{oss:14}).
Since $(W_1, W_2)$ is refined, it follows that the vanishing sequence of $W_2 \subset H^0(C_2, E_{n_2}|_{C_2})$ at $p$ is
\[ (\underbrace{b-a-1, \dots, b-a-1}_{r+1-i}, \underbrace{b-a, \dots, b-a}_{i}).\]
Recall from \autoref{sec:prnodal} that $W_1$ yields two vector spaces $V^0$ and $V^1$ in the fiber $E_{n_1}|_p$, which we may identify canonically (up to scaling) with the fiber $E|_p$.
Likewise, $W_2$ yields two analogous vector spaces, call them $\Lambda^0$ and $\Lambda^1$, in $E|_p$.
The gluing condition in the definition of EHT limit linear series (\autoref{def:eht}) and the definition of these vector spaces immediately shows that
\begin{equation}\label{eqn:vlambdaswitch}
  V^0 = \Lambda^1 \text{ and } V^1 = \Lambda^0.
\end{equation}
The third condition we want to impose is that these two vector spaces be transverse, namely $\dim (V^0 \cap V^1) = 1$.

Let $\mathcal U \subset {\mathcal G}(r+1, \mathcal E)$ be the complement of the union of the following closed sets:
\begin{enumerate}
\item the closure of the subset of $\Gr(r+1, H^0(X_\eta, \mathcal E_\eta))$ corresponding to $V \subset H^0(X_\eta, \mathcal E_\eta)$ for which the evaluation map $V\otimes\O_{X_\eta} \to \mathcal E_\eta$ has generic rank less than $r$.
\item the set of limit linear series $(V_n \mid n \in \Z)$ on $C$ such that the associated EHT limit linear series $(W_1, W_2)$ is not refined, or does not have the vanishing sequence as in \eqref{eqn:llsvs}, or does not satisfy the transversality condition $\dim (V^0 \cap V^1) = 1$.
\end{enumerate}
Give $\mathcal U$ the reduced scheme structure.

Let $S$ be a reduced $B$-scheme with a map to $\mathcal U$ given by the limit linear series $(V_n \mid n \in \Z)$.
On $X_S$, we have a diagram analogous to \eqref{eqn:differential_construction}, namely
\begin{equation}
  \label{eq:llspr}
  \begin{tikzcd}
    &\det \mathcal E_n^* \otimes \det V_n\ar{r}{j}\ar{d}{d} & V_n \otimes \O_{X_S}\ar{r}{e}\ar{d}{e} & \mathcal E_n\ar[equal]{d}&\\
    0\ar{r} & \Omega_{X_S/S} \otimes \mathcal E_n\ar{r} & P(\mathcal E_n)\ar{r} &\ar{r} \mathcal E_n \ar{r}& 0.
  \end{tikzcd}
\end{equation}
Here $P(\mathcal E_n)$ is the sheaf of principal parts of $\mathcal E_n$ relative to $X_S \to S$, and the bottom row is the natural exact sequence coming from its definition.
The top row is a complex, but it may not be exact.
The maps labeled $e$ are the evaluation maps.
The map $j$ is defined by the maximal minors of $e \from V_n \otimes \O_{X_S} \to \mathcal E_n$.
The map $d$ is the unique map induced by the other maps in the diagram.
By composing $d$ through the inclusion $\Omega_{X_S/S} \to \omega_{X_S/S}$, and doing some rearrangement, we obtain a map
\begin{equation}\label{eqn:Rn}
r_n \from \det V_n \to \mathcal \pi_*(\mathcal E_n \otimes \det \mathcal E_n \otimes \omega^*_{X_S/S}) = \pi_*(\mathcal E'_{(r+1)n}).
\end{equation}
Consider the two extremal sections, namely those corresponding to $n = n_1$ and $n = n_2$.
\begin{lemma}\label{lem:rameht}
  Over every $s \in S$ over $0 \in \Delta$, the restrictions $r_{n_1}|_s$ and $r_{n_2}|_s$ define a one-dimensional refined EHT limit linear series for $E'$.
\end{lemma}
\begin{proof}
  Without further comment, we identify $r_{n_i}|_s \in H^0(C, E'_{(r+1)n_i})$ with its image in $H^0(C_i, E'_{(r+1)n_i}|_{C_i})$.
  We have
  \[E'_{(r+1)n_2}|_{C_i} = E_{n_2} \otimes \det E_{n_2} \otimes \omega_C|_{C_1} = E_{n_2} \otimes \det E_{n_2} \otimes \Omega_C|_{C_1} \otimes \O_{C_1}(p),\]
  and by construction $r_{n_1}|_s$ is the image of the ramification section of $V_{n_1} \subset H^0(C_1, E_{n_1}|_{C_1})$ under the inclusion map
  \[ E_{n_1} \otimes \det E_{n_1} \otimes \Omega_C|_{C_1} \to E_{n_1} \otimes \det E_{n_1} \otimes \omega_C|_{C_1} = E'_{(r+1)n_1}|_{C_1}.\]
  By \autoref{prop:agreement}, the ramification section of $V_{n_1}$ has order of vanishing $(r+1)a+(r-i)$ at $p$, and hence $r_{n_1}|_s$ on $C_1$ has order of vanishing $(r+1)a+(r-i+1)$ at $p$.
  Likewise, $r_{n_2}|_s$ on $C_2$ has order of vanishing $(r+1)(b-a-1)+i$ at $p$.
  Since
  \[ (r+1)a+(r-i+1) + (r+1)(b-a-1) + i = (r+1)b = b',\]
  we see that $r_{n_1}|_s$ and $r_{n_2}|_s$ have complementary orders of vanishing, leading to an equality in condition~\eqref{ieq:eht} of \autoref{def:eht}.

  We must next ensure that condition~\eqref{gluing:eht} of \autoref{def:eht} holds, that is, the images of $r_{n_i}|_s$ in the appropriate twists of $E_{n_i}|_p$ are equal, at least up to scaling.
  By \autoref{prop:agreement}, the image of $r_{n_1}|_s$ in the appropriate twist of $E_{n_1}|_p$ spans the line $(V^0 \cap V^1)$, and the image of $r_{n_2}|_s$ spans the line $\Lambda^0 \cap \Lambda^1$.
  But by \eqref{eqn:vlambdaswitch}, we have $V^1 = \Lambda^0$ and $V^0 = \Lambda^1$, so the two lines are equal.
\end{proof}

Thanks to \autoref{lem:rameht}, we apply \autoref{prop:llseht}, and conclude that there exists a unique (1-dimensional) limit linear series $(R_n \mid n \in \Z)$ of $\mathcal E'$ on $X_S$ for which $R_{n_1'} = \det V_{n_1}$ and $R_{n_2'} = \det V_{n_2}$, at least if $S$ is reduced.
The transformation
\[ (V_n \mid n \in Z) \mapsto (R_n \mid n \in \Z)\]
defines a morphism
\begin{equation}\label{prop:mapreduced}
  \rho \from \mathcal U \to \mathcal G(1, \mathcal E'),
\end{equation}
as desired in \eqref{eq:Rtilde}.
Note that $\mathcal U$ has the reduced scheme structure.

The fruit of our labor is the following corollary.
\begin{corollary}\label{prop:degeneration}
  Suppose $v \in \mathcal U_0$ is such that $\dim_v \mathcal U_0 = (r+1)(d-rg-1)$ and $v$ is isolated in the fiber of $\rho$, then the projection-ramification map $\Gr(r+1, H^0(X_\eta, \mathcal E_\eta)) \dashrightarrow \P H^0(X_\eta, \mathcal E_\eta \otimes \det E_\eta \otimes K_{X_\eta})$ is generically finite.
\end{corollary}
\begin{proof}
  If $\dim_v U_0 = (r+1)(d-rg-1)$, then $v$ is in the closure of $\Gr(r+1, H^0(X_\eta, \mathcal E_\eta))$ by \autoref{thm:lls}.
  The statement now follows from the upper semi-continuity of fiber dimension.
\end{proof}


\subsection{Maximal variation for generic scrolls of high degree}
We now have all the tools to prove \autoref{thm:rationalnormalscrolls}
\begin{theorem}[\autoref{thm:rationalnormalscrolls}]
  Let $E$ be a generic vector bundle on $\P^1$ of rank $r$ and degree $d = a(r-1) + b(2r-1)+1$, where $a, b$ are positive integers.
  Then the projection-ramification map is generically finite, and hence dominant, for $E$.
  In particular, the projection-ramification map is dominant for generic $E$ of degree $\geq (r-1)(2r-1)+1$.
\end{theorem}
\begin{proof}
  We say that generic dominance holds for rank $r$ and degree $d$ if the projection-ramification map is dominant (equivalently, generically finite) for the generic vector bundle of rank $r$ and degree $d$.
  The rank will be fixed througout, so let us drop it from the discussion.
  Let us prove that if generic dominance holds for degrees $d_1$ and $d_2$, then it also holds for degree $d = d_1 + d_2 - 1 $.
  With the base cases $d_1 = r$ (\autoref{prop:segre}) and $d_2 = 2r$ (\autoref{prop:222}), this proves the theorem.

  Take $C_1 = C_2 = \P^1$, and let $C = C_1 \cup C_2$ be their nodal union at one point, which we take to be the point labeled $0$ on both $\P^1$s.
  Let $X \to B$ be a smoothing of $C$.
  Note that any vector bundle on $C$ is the restriction of a vector bundle on $X$.
  Therefore, by \autoref{prop:degeneration}, it suffices to construct a vector bundle $E$ of degree $d$ on $C$ and a limit linear series $(V_n \mid n \in \Z)$ on $E$ such that the following conditions hold for the point $v$ of $\mathcal G (r+1, E')$ represented by $(V_n \mid n \in \Z)$:
  \begin{enumerate}
  \item $\dim_v \mathcal G(r+1, E) = (r+1)(d-1)$,
  \item $\rho$ is defined at $v$, and
  \item $v$ is an isolated point in the fiber of $\rho$.
  \end{enumerate}

  We construct $E$ as follows.
  Let $E_1$ be a generic vector bundle of degree $d_1$ on $C_1$, and $E_2'$ a generic vector bundle of degree $d_2 - 1$ on $C_2$.
  Choose a generic isomorphism $E_1|_0 \cong E_2'|_0$, and construct the vector bundle $E$ on $C$ by gluing $E_1$ and $E_2'$ along this isomorphism.
  Choose $n_1 = a$ and $n_2 = b+a$ for sufficiently negative $a$ and sufficiently positive $b$.
  The isomorphism $E_1 |_0 \cong E_2'|_0$ yields isomorphisms, canonical up to scaling, of $E_1(m)|_0$ and $E_2'(n)|_0$ for any $m, n \in \Z$.

  Having constructed $E$, we must now construct $(V_n \mid n \in \Z)$.
  By \autoref{prop:llseht}, it is enough to construct $V_{n_1} \subset H^0(C_1, E_1 \otimes \O(a))$ and $V_{n_2} \subset H^0(C_2, E_2'(b-a))$, provided they define a refined EHT limit linear series.
  Let $V \subset H^0(C_1, E_1)$ be a generic $(r+1)$-dimensional vector space.
  Then it will have the vanishing sequence $(0, \dots, 0, 1)$.
  Hence, we have $V^0 = E|_0$ and $V^1 \subset E|_0$ is $1$-dimensional (see \autoref{sec:prnodal} for the definition of these two subspaces).
  Furthermore, the genericity of $V$ implies that $V^1$ is a general $1$-dimensional subspace.
  Let $\Lambda \subset H^0(C_2, E_2'(1))$ be the image of a general $(r+1)$ dimensional subspace of $H^0(C_2, E_2)$, where $E_2$ is the vector bundle of degree $d_2$ defined by the sequence
  \[ 0 \to E_2 \to E_2'(1) \to E'_2(1)|_0 / V^1 \to 0.\]
  Then $\Lambda \subset H^0(C_2, E_2'(1))$ has the vanishing sequence $(0, 1, \dots, 1)$, with $\Lambda^0 = V^1$ and $\Lambda^1 = V^0$.
  Let $V_{n_1} \subset H^0(C_1, E_1 \otimes \O(a))$ be the image of $V$ and $V_{n_2} \subset H^0(C_2,E_2 \otimes \O(b-a))$ the image of $\Lambda$.
  Then $V_{n_1}$ has the vanishing sequence $(a, \dots, a, a+1)$, and $\Lambda$ the complementary vanishing sequence $(b-a-1, b-a, \dots, b-a)$.
  By the construction of $\Lambda$, there exist bases of $V_{n_1}$ and $V_{n_2}$ that satisfy the gluing condition at $0$.
  In conclusion, $V_{n_1}$ and $V_{n_2}$ form a refined EHT limit linear series, and hence define a limit linear series $v = (V_n \mid n \in \Z)$.

  It is easy to check that $\dim_v \G(r+1, E) = (r+1)(d-1)$.
  Indeed, for every limit linear series $w = (W_n \mid n \in \Z)$ in an open subset around $v$, the EHT limit linear series associated to $w$ determines $w$ and has the same vanishing sequence as $v$.
  In particular, $W_{n_1} \subset H^0(C_1, E_1(a))$ is the image of an $(r+1)$-dimensional subspace $W \subset H^0(C_1, E_1)$ with vanishing sequence $(0, \dots, 1)$, and $W_{n_2} \subset H^0(C_2, E_2(b-a))$ is the image of an $(r+1)$-dimensional subspace $M \subset H^0(C_2, E_2(1))$ with vanishing sequence $(0,1,\dots,1)$.
  Furthermore, the gluing condition implies that $M$ is in fact the image of an $(r+1)$-dimensional subspace of the kernel of the map
  \[ E_2'(1) \to E_2'(1)/W^1.\]
  By the genericity of $V$, the isomorphism type of the kernel of this map is constant around $v$; that is, the kernel is isomorphic to $E_2$.
  So, a dimension count for $\mathcal G(r+1, E)$ around $v$ gives
  \begin{align*}
    \dim_v \mathcal G(r+1, E) &= \dim \Gr(r+1, H^0(C_1, E_1)) + \dim \Gr(r+1, H^0(C_2, E_2))\\
                              &= (r+1)(d_1-1) + (r+1)(d_2-1)\\
                              &= (r+1)(d_1+d_2-2)\\
                              &= (r+1)(d-1).\\
  \end{align*}

  Finally, we must check that $v$ is an isolated point in the fiber of
  \[ \rho \from \mathcal G(r+1, E) \dashrightarrow \mathcal G(1, E \otimes \det E \otimes \omega_C).\]
  For any $w \in \mathcal G(r+1, E)$ in an open set around $v$, either $V \neq W$ or $\Lambda \neq M$, where $V, \Lambda, W, M$ are as above.
  By construction, $V \subset H^0(r+1, H^0(C_1, E_1))$ and $\Lambda \subset H^0(r+1, H^0(C_2, E_2'))$ are isolated in their respective projection-ramification maps.
  Therefore, either $\rho_{C_1} (V) \neq \rho_{C_1}(W)$ or $\rho_{C_2}(\Lambda) \neq \rho_{C_2}(M)$.
  In either case, we obtain that $\rho(v) \neq \rho(w)$, and hence conclude that $v$ is an isolated point in the fiber of $\rho$.
\end{proof}


\section{The Projection-Ramification enumerative problem} % (fold)
\label{sec:enumerativeproblems}

Our objectives in this section are to prove \autoref{Thm:Examples}, and also to relate special instances of the projection-ramification enumerative problem with constructions in classical algebraic geometry.


\subsection{Quadric hypersurfaces} % (fold)
\label{sub:a_quadric_surface}
A smooth quadric hypersurface $X \subset \P^{n}$ defined by an equation $F(x_{0},x_{1},x_{2}, ...) = 0$ induces the classical {\sl polarity isomorphism} $\P^{n} \longleftrightarrow (\P^{n})^{\smvee}$ given by 
\begin{align*}
  p = [p_{0}:p_{1}:p_{2}: ...] \mapsto [\partial_{0}F(p):\partial_{1}F(p):\partial_{2}F(p): ...]
\end{align*}
where $\partial_{i}$ denotes derivative with respect to the $i$-th variable $x_{i}$. The duality morphism is equal to $\rho_{X}$, and hence $\deg \rho_{X} = 1$.  


% subsection a_quadric_surface (end)

\subsection{The Veronese surface} % (fold)

Let $\P^{2} \simeq X \subset \P^{5}$ be the Veronese surface. Then the projection-ramification morphism 
\begin{align*}
  \rho_{X}: \Gr(3,6) \dashrightarrow \P^{9}
\end{align*}
assigns to a general net of conics $N$ the cubic curve $C \subset \P^{2}$ consisting of the nodes of the singular members of $N$.  We will outline the classical algebraic geometry (the relationship between cubic curves and their Hessians) underlying the claim that the degree of $\rho_{X}$ is $3$. 

Suppose $N = \langle Q_{1}, Q_{2}, Q_{3} \rangle$ is a general net of conics in $\P^{2}$, with $Q_{i}$ general ternary quadratic forms. For each line $L \subset \P^{2}$, the net $N$ restricts either to the complete linear series of two points on $L$ or it restricts to a pencil.   The latter type of line a {\sl Reye} line. 

\begin{lemma}
   \label{lem:specialLines}
   The set of Reye lines $C' \subset \P^{2*}$ is a smooth  cubic equipped with a fixed-point free involution $\tau$ with quotient isomorphic to $C$.
 \end{lemma}  

\begin{proof}
  Each Reye line $L$ arises from a unique singular conic of the net $N$, and hence possesses a conjugate line $L'$, which defines the fixed point free involution $\tau$. 

  Let $S \to \P^{2*}$ denote the rank $2$ tautological subbundle. The forms $Q_{1},Q_{2},Q_{3}$ define a map of vector bundles 
  \begin{align*}
    \O^{3} \to \Sym^{2}S^{*}
  \end{align*}
  The determinant of this map defines the locus of Reye lines, and a simple Chern class calculation reveals the locus is a cubic $C' \subset \P^{2*}$.  The quotient of $C'$ by the involution sending $L$ to $L'$ is clearly identifiable with $C \subset \P^{2}$.
\end{proof}


If $L$ is a Reye line, then $L$ is a component of a unique singular member of the net $N$, and therefore has a conjugate line $L'$.  On $L$, there are now three points of significance: $x = L \cap L'$, which is clearly a point on $C$, and the residual pair of points $a_{L}, b_{L} \in L \cap C$. Similarly for $L'$.  We may view $C'$ as a $2:1$ unramified cover of $C$. Lying above the points $a_{L},b_{L} \in C$ are points $a_{L}',a_{L}'', b_{L}', b_{L}'' \in C'$. 

\begin{lemma}
  \label{lem:lineEq}
  Maintain the setting above. Then on $C$ the following linear equivalences hold: $2a_{L} \sim 2b_{L} \sim 2a_{L'} \sim 2b_{L'}$.
\end{lemma}

\begin{proof}
  Attached to $a_{L}$ and $b_{L}$ are the dual lines $a_{L}^{*}, b_{L}^{*} \subset \P^{2*}$. The intersections $a_{L}^{*} \cap C'$ and $b_{L}^{*} \cap C'$ both contain the point $[L] \in \P^{2*}$. The residual intersections of $C'$ with $a_{L}^{*}$ and $b_{L}^{*}$ are the points $a_{L}', a_{L}''$ and $b_{L}', b_{L}''$ in $C'$. Hence, on $C'$ we get a linear equivalence $a_{L}'+ a_{L}'' \sim b_{L}' +  b_{L}''$. Pushing this linear equivalence forward under the quotient map $C' \to C$ gives $2a_{L} \sim 2b_{L}$. 

  Proceeding in a similar way, note that the points $[L], [L'] \in C'$ constitute the two points in $C'$ lying above $x \in C$. Further, on $C'$ we get the equivalence $[L] + a_{L}'+ a_{L}'' \sim [L'] + a_{L'}'+ a_{L'}''$ since both triads are collinear in $\P^{2*}$.  Pushing this equivalence forward to $C$ yields the equivalence $2a_{L} \sim 2a_{L'}.$ 
\end{proof}




\begin{lemma}
  \label{lem:2torsionclass}
  The class $\eta = a_{L}-b_{L} \in Jac(C)[2] \setminus \{0\}$ is independent of the point $x$ and the choice of Reye line $L$.
\end{lemma}
\begin{proof}
  For each $x \in C$ there are two Reye lines $L,L'$ containing $x$, and two pairs of points $a_{L},b_{L}$ and $a_{L'},b_{L'}$ respectively. 

  Now, if four points $p,q,r,s \in C$ satisfy $2p \sim 2q \sim 2r \sim 2s$, then it is always true that $p-q 
  \sim r-s$, as a straightforward divisor calculation shows. 

  The lemma now follows by the fact that there are only finitely many $2$-torsion divisor classes on $C$, combined with the fact that $C'$ is irreducible.
\end{proof}


The $2$-torsion class $\eta$ defines a translation on $C$ which takes a point $x \in C$ to the unique point denoted $\eta(x) \in C$ which is linearly equivalent to $x + \eta$. Therefore, \autoref{lem:2torsionclass} allows us to describe the set of Reye lines as the lines joining $p$ with $\eta(p)$ for all points $p \in C$.  




Thanks to \autoref{lem:2torsionclass}, we see the projection-ramification map $\rho_{X}$ factors as: 
\begin{align}\label{eq:compose}
  \rho_{X}: \Gr(3,6) \dashrightarrow Jac[2] \dashrightarrow \P^{9}
\end{align}
where $Jac[2]$ is the variety parametrizing pairs $(C,\eta)$ with $C$ a smooth plane cubic and $\eta \in Jac(C)[2]$ a non-trivial $2$-torsion element. 


To conclude, we argue the first map in \eqref{eq:compose} is birational by constructing its inverse.  To this end, suppose $C$ is a smooth plane cubic, and $\eta \in Jac(C)[2]$ a chosen non-trivial $2$-torsion element. We will create from this data a net of conics $N$ whose set of nodes is $C$. Again, we think of $\eta$ as a translation $C \to C$ in the usual way.

  For every point $p \in C$, we get a line $L_{p} \subset \P^{2}$ joining $p$ and $\eta(p)$. In this way, we obtain a map $f: C \to \P^{2*}$ which is $2:1$ onto its image, since $L_{p} = L_{q}$ if and only if $p=q$ or $\eta(p)=q$.  Since  $\eta:C \to C$ is fixed-point free, it is easy to see that $f$ is also unramified. Hence, the image of $f$ must be a smooth cubic. 

  If $\beta \neq \eta \in Jac(C)[2]$ is any other non-trivial 2-torsion element, the pair of points $\beta(p), \beta(\eta(p))$ span a well-defined second line $L'_{p}$ containing the point $p$.  

  The collection of singular conics $L_{p} \cup L_{p}'$ parametrized by $p \in C$ induces a map $C \to \P^{5}$, whose degree is $3$, since  through a general point in $\P^{2}$ there pass $3$ of the lines $L_{p}$.  Furthermore, a divisor class computation shows that the node point $L_{p} \cap L_{p'}$ is again on $C$. Hence the image of $C \to \P^{5}$ spans a plane which by construction is the desired net of conics $N$ whose locus of nodes is $C$.


\subsection{Quartic surface scroll} Our next objective is to prove that $ \deg \rho_{X} = 2$ for a generic quartic surface scroll $X \subset \P^{5}$.  Our proof uncovers a  rich geometric picture similar to the case of the Veronese surface in the previous subsection.  


We begin with the following seemingly unrelated geometric figure: $C \subset \P^{2}$ is a smooth cubic curve, $a \in \P^{2} \setminus X$ a point, and $Q \subset \P^{2}$ the {\sl polar conic} of $a$ with respect to $C$ -- the unique conic which passes through the six points of ramification on $C$ of the projection from $a$.   We assume $a$ is chosen so that $Q$ is a smooth conic. 

To set notation moving forward, if $x \in \P^{2}$ is any point, we let $P_{x}(C)$ denote the polar conic of $x$ with respect to $C$.  Similarly, we let $P_{x}(Q)$ denote the polar line of $x$ with respect to the conic $Q$.

\begin{lemma}\label{lemma:basicsaboutHessian}
The Hessian $Hess(C) \subset \P^{2}$ consists of the points $x$ such that $P_{x}(C)$ is singular, and if $C$ is not a Fermat cubic $P_{x}(C)$ is the union of two distinct lines for every $x \in Hess(C)$.  Furthermore, if $x \in Hess(C)$, then the unique singularity $s(x)$ of $P_{x}(C)$ lies on $Hess(C)$, and the map $x \mapsto s(x)$ is translation by a $2$-torsion point $\eta \in \Jac( Hess(C))$.
\end{lemma} 

\begin{proof}
    Standard. \todo{Reference, probably Dolgachev}
\end{proof} 

\begin{proposition}\label{proposition:polarline}
Suppose $x \in Hess(C)$. Then the line $P_{x}(C)$ passes through the point $x + \eta \in Hess(C)$.
\end{proposition}

\begin{proof}
    We have: 
    \begin{align*}
        P_{x}(Q) = P_{x}P_{a}(C) = P_{a}P_{x}(C).
    \end{align*}

Since $x \in Hess(C)$, $P_{x}(C)$ is a singular conic. Hence $P_{x}(Q)$ must pass through the singularity $\sing P_{x}(C)$, which by \autoref{lemma:basicsaboutHessian} is the point $x + \eta$.
\end{proof}

Next suppose $\ell_{1},m_{1}, \ell_{2}, m_{2}, \ell_{3}, m_{3}$ are six distinct lines in $\P^{2}$ with the  properties: 
\begin{enumerate}
    \item The three singular conics $\ell_{i} \cup m_{i}$ are polars of $C$.
    \item The triangle $\ell_{1}\ell_{2} \ell_{3}$ is {\sl conjugate} to the triangle $m_{1}  m_{2}  m_{3}$ with respect to $Q$.
\end{enumerate}
The second condition above simply means that the vertices of one triangle are polar to the lines of the other triangle.  By basic projective geometry, the two triangles are then in {\sl linear perspective}, i.e. the three points $x_{1} := \ell_{1} \cap m_{1}, x_{2} := \ell_{2} \cap m_{2}, x_{3} := \ell_{3} \cap m_{3}$ are collinear.

\begin{proposition}\label{proposition:importantReyeLineFact}
Maintain the notation above, and recall the definition of Reye line from the previous subsection. The lines $P_{x_{i}}(Q)$ are Reye lines of the net of polar conics of $C$.
\end{proposition}

\begin{proof}
    The triangles $\ell_{1}\ell_{2}\ell_{3}$ and $m_{1}m_{2}m_{3}$ are conjugate with respect to $Q$.  Hence it follows that the polar line $P_{x_{3}}(Q)$ equals $\ell := \overline{\ell_{12}m_{12}}$, where $\ell_{ij} = \ell_{i} \cap \ell_{j}$ and $m_{ij} = m_{i} \cap m_{j}$.
    
    We will prove that $\ell$ is one of the three Reye lines of the net of polars of $C$ which pass through the point $\ell_{12}$, the other two Reye lines being $\ell_{1}$ and $\ell_{2}$. From \autoref{lemma:basicsaboutHessian}, we can find points $y,z \in C$ and write: 
    \begin{align*}
        \ell_{1} = \overline{y, y + \eta}\\
        \ell_{2} = \overline{z, z + \eta}
    \end{align*}
Then a divisor class computation shows that the third Reye line through $\ell_{12}$ must be $\overline{w,w + \eta}$, with $w$ satisfying 
\begin{align*}
    y+z+w \sim H + \epsilon,
\end{align*}
where $\epsilon$ is any one of the two non-trivial $2$-torsion elements on $Hess(C)$ differing from $\eta$.  We let $s \in Hess(C)$ denote the third point of intersection of the line $\overline{w,w + \eta}$ with $C$. (Notice that $w$ depends on the choice of $\epsilon$, but the line $\overline{w,w + \eta}$ is independent of this choice.)

From this setup, we get: 
\begin{align*}
    x_{1} &\sim H-2y-\eta\\
    x_{2} &\sim H-2z - \eta\\
    s &\sim H-2w - \eta
\end{align*}
from which we get:
\begin{align*}
    s &\sim H-2w - \eta\\
    &\sim H - 2[H+\epsilon -y-z] - \eta\\
    &\sim 2y+2z-H-\eta\\
    &\sim H-x_{1}-\eta + H-x_{2}-\eta-H-\eta\\
    &\sim H-x_{1}-x_{2} - \eta.
\end{align*}

Therefore, to prove that $\ell$ is a Reye line, it suffices to show that the points $\ell_{12}, m_{12}$, and $s \sim H-x_{1}-x_{2}-\eta$ are collinear.  But, this is true if and only if their respective polar lines $P_{\ell_{12}}(Q), P_{m_{12}}(Q), P_{s}(Q)$ are concurrent.  The latter is true if and only if the lines $m_{3}, \ell_{3}, P_{s}(Q)$ are concurrent, which in turn  translates to the condition that $x_{3} \in P_{s}(Q)$.  But, $x_{3} \sim H-x_{1}-x_{2}$, and $s \sim H-x_{1}-x_{2}+\eta$, and so by \autoref{proposition:importantReyeLineFact}, we conclude that indeed $x_{3} \in P_{s}(Q)$, which is what we needed to show.
\end{proof}

\subsubsection{Returning to the projection ramification problem} Our next objective is to relate the geometry in the previous subsection to 



\subsection{Rational curves, the differential construction, and the case of Segre varieties} % (fold)
\label{sec:rational_curves_in_projective_space}
\autoref{problem:degree} connects with an old story involving rational curves in projective space.  

Let $\gamma: \P^{1} \to \P^{n}$ be a degree $d$ morphism. Its derivative 
\begin{align*}
    d\gamma:T_{\P^{1}} \to \gamma^{*}(T_{\P^{n}})
  \end{align*}  
may be viewed as a global section of the rank $r$ vector bundle $\gamma^{*}(T_{\P^{n}}) \otimes T_{\P^{1}}^{\smvee}$.  The splitting of $\gamma^{*}(T_{\P^{n}})$ is known to be balanced for a general morphism $\gamma$. In particular, if the divisibility
\begin{align*}
  n \mid d
\end{align*}
holds, and if we set $\ell := d+d/n-2$, then a general $\gamma$ satisfies: 
\begin{align*}
  (\gamma^{*}T_{\P^{n}}) \otimes T_{\P^{1}}^{\smvee} \simeq \bigoplus_{i=1}^{n} \O_{\P^{1}}(\ell).
\end{align*} 
The direct sum decomposition is not canonical, it is only defined up to the
action of $GL_{n}(k)$.  

Assuming $\gamma$ is an immersion, the element $d \gamma \in
H^{0}(\P^{1},\bigoplus_{i=1}^{n} \O_{\P^{1}}(\ell))$ does not vanish anywhere,
and hence defines a degree $\ell$ map $$D(\gamma) : \P^{1} \to \P^{n-1},$$
only well-defined up to the action of post-composition by $PGL_{n}(k)$. 

\begin{definition}
  Let $M^{n}_{d}$ denote the moduli stack parametrizing $PGL_{n+1}(k)$
  equivalence classes of degree $d$ maps $\gamma : \P^{1} \to \P^{n}$, and let
  $U^{n}_{d} \subset M^{n}_{d}$ denote the open substack parametrizing local immersions with $\gamma^{*}(T_{\P^{n}})$ balanced.
\end{definition}

\begin{remark}
Notice:  $\dim M^{n}_d= (k+1)(n+1) - (n+1)^{2} = (n+1)(k-n) = \dim \G(n, k).$  Furthermore, notice  $PGL_{2}(k)$ acts on $U^{n}_{d}$ and $M^{n}_{d}$ by pre-composition. 
\end{remark}  

\begin{remark}
  Though $M^{n}_{d}$ is an Artin stack, the open substack $U^{n}_{d}$ is a scheme, provided $n \leq d$, represented by an open subset of $\Gr(n+1,d+1)$. 
\end{remark}




When $n \mid d$, and $\ell := d+d/n-2$, we get the morphism of stacks: 
\begin{align*}
  D^{n}_{d} &: U^{n}_{d} \to M^{n-1}_{\ell}\\
  \gamma &\longmapsto D(\gamma)
\end{align*}
which we call the {\sl differential construction}. Interestingly, the dimensions of the domain and codomain of the differential construction are equal, and this leads to another collection of enumerative problems: 
\begin{problem}\label{problem:differential}
   Compute the degrees of the differential constructions $D^{n}_{d} : U^{n}_{d} \to M^{n-1}_{\ell}$. 
 \end{problem} 

\begin{remark}
  The maps $D^{n}_{d}$ are clearly $PGL_{2}(k)$ equivariant.  The image of the differential construction $D^{n}_{d}$ need not be the open set $U^{n-1}_{\ell}$. \todo{Sure?}
\end{remark}


 The $n=d$ instances of \autoref{problem:differential} are immediate: 

 \begin{proposition}\label{proposition:trivialdegree}
   The degree of the differential construction $D^{d}_{d}$ is $1$.
 \end{proposition}
\begin{proof}
  The space $U^{d}_{d}$ is a single $PGL_{2}(k)$ orbit.
\end{proof}
 
\begin{definition}\label{definition:pointlinescroll}
  Let $\gamma: \P^{1} \to \P^{n}$ be any map. We define the {\sl point-hyperplane scroll} of $\gamma$ to be 
  \begin{align*}
        X_{\gamma} := \big\{(t, \Lambda) \mid \gamma(t) \in \Lambda \big\} \subset \P^{1} \times (\P^{n})^{\smvee}
      \end{align*}
      We denote by $\pi_{1}, \pi_{2}$ the projections of $X_{\gamma}$ to $\P^{1}$ and $(\P^{n})^{\smvee}$ respectively. Finally, we set $X_{\gamma}^{\smvee} := \P(\gamma^{*}T_{\P^{n}})$.   
\end{definition}

\begin{remark}
  The $\P^{n-1}$-bundle $X_{\gamma}$ is isomorphic to $\P(\gamma^{*}T^{\smvee}_{\P^{n}})$. Hence, for a general map $\gamma: \P^{1} \to \P^{n}$, $X_{\gamma}$ and $X_{\gamma}^{\smvee}$ are balanced scrolls.
\end{remark}

\begin{proposition}\label{proposition:transfer}
 Let $\gamma: \P^{1} \to \P^{n}$ be a non constant map.

 \begin{enumerate} 
  \item The image of $\gamma: \P^{1} \to \P^{n}$ is non-degenerate if and only if $\pi_{2} : X_{\gamma} \to (\P^{n})^{\smvee}$ is finite; in any case,  $\deg \pi_{2} = \deg \gamma.$ 
  \item The ramification divisor $R(\pi_{2}) \subset X_{\gamma}$ is a smooth, codimension $1$ subscroll of $X_{\gamma}$ if and only if $\gamma$ is an immersion. 
  \item Assuming $\gamma$ is an immersion, the dual section $R^{\smvee}(\pi_{2}) \subset X^{\smvee}_{\gamma}$ is induced by the inclusion $d \gamma: T_{\P^{1}} \hookrightarrow \gamma^{*}T_{\P^{n}}$.
\end{enumerate}
\end{proposition}

\begin{proof}
  \todo{PROVE}
\end{proof}

Let $X = \P^{1} \times \P^{n-1}$, and denote by $h$ and $f$ the divisor classes of the pullback of a hyperplane in $\P^{n-1}$ and a point in $\P^{1}$, respectively.  When $n \mid k$,  \autoref{proposition:transfer} sets up a commuting diagram: 

\begin{center}
\begin{tikzcd}
  & U^{n}_{k} \ar[rr, leftrightarrow, "\text{duality}"] \ar[dd, "D^{n}_{k}"] & &PGL_{n+1}\Big\backslash \left \{ \begin{tabular}{c} \text{Deg. $k$ maps} \\$X \to (\P^{n})^{\smvee}$ \\ \text{induced by $|h+\frac{k}{n}f|$} \end{tabular} \right \}\Big/PGL_{n} \ar[dd, "\rho_{X}"] \\
  &  &  & \\
  & M^{n-1}_{\ell}  \ar[rr, leftrightarrow, "\text{duality}"]  && \left \{ \begin{tabular}{c} \text{Smooth divisors $R \subset X$} \\\text{with div. class $|h + \ell f|$} \end{tabular} \right \} \big/ PGL_{n}
\end{tikzcd}
\end{center}

From this, we conclude: 
\begin{proposition}\label{proposition:equivalence}
  Let $k = nm$, and let $X \subset \P^{n(m+1)-1}$ be the variety $\P^{1} \times \P^{n-1}$ embedded by the linear series $|h+mf|$. Then 
  \begin{align*}
    \deg \rho_{X} = \deg D^{n}_{k}.
  \end{align*}
\end{proposition}

\begin{corollary}
  If $X \subset \P^{2n-1}$ is a Segre embedding of $\P^{1} \times \P^{n-1}$, then $\deg \rho_{X} = 1$.
\end{corollary} 

\begin{proof}
  The corollary follows at once from \autoref{proposition:equivalence} and \autoref{proposition:trivialdegree}.
\end{proof}

\subsection{Quartic surface scrolls}





\subsubsection{The explicit differential construction for trinodal quartics}

A trinodal quartic $R$ can be obtained as an abstract curve by identifying three pairs of points $\{a',a''\}, \{b',b''\}, \{c',c''\}$ on $\P^{1}$. These pairs can be encoded by the three binary quadratic forms (up to scale)  defining them.  In terms of these three quadratic forms, we will now describe the differential construction $D^{2}_{4}$. 


In what follows, we let $\{ q_{1},q_{2},q_{3} \}$ denote a point in $\Sym^{3} \P H^{0}(\O_{\P^{1}}(2))$.

\begin{definition}
  \label{definition:nodemap}
  Let
  \begin{align*}
    \nu: \Sym^{3}\P H^{0}(\O_{\P^{1}}(2)) \dashrightarrow \Gr(3,5)
  \end{align*}
  denote the map given by the formula:
  \begin{align*}
    \nu \left(\{q_{1},q_{2},q_{3}\}\right ) =    \left\{ \begin{tabular}{c}
      v. space of meromorphic $1$-forms $\omega$ on $\P^{1}$ with at worst \\
      simple poles at the zeros of $q_{i}$ and with {\sl opposite} residues\\
      at the pairs of zeros of $q_{i}$, for all $i = 1,2,3$
    \end{tabular}\right\}
  \end{align*}
 
\end{definition}



\begin{proposition}
  \label{proposition:symtwoptwo}
The map $\nu$ is birational.
\end{proposition}
\begin{proof}
  Suppose a general three dimensional space $W \subset H^{0}(\O_{\P^{1}}(4))$ is given. Then the induced degree four map $\P^{1} \to \P W^{\smvee}$ is the normalization of a trinodal quartic $R$. The vector space $W$ is naturally identified with the sections of the dualizing sheaf of $R$, which consist of meromorphic $1$-forms on $\P^{1}$ with the properties stated in the proposition.
\end{proof}

\begin{definition}
  \label{definition:pi}
  Let
  \begin{align*}
    \pi: \Sym^{3}\P^{2} \dashrightarrow \Gr(2,5)
  \end{align*}
  be given by the formula
  \begin{align*}
\pi\left (\{q_{1},q_{2},q_{3}\}\right ) =    \left\{ \begin{tabular}{l}
      v. space of meromorphic $1$-forms $\omega$ with at worst simple\\
      poles at the zeros of $q_{i}$ and with {\sl equal} residues\\
      at the pairs of zeros of $q_{i}$, for all $i = 1,2,3$
    \end{tabular}\right\}
  \end{align*}
 
\end{definition}

\begin{proposition}
  \label{proposition:reinterpretTangent}
  The rational map $\pi \circ \nu^{-1} : \Gr(3,5) \dashrightarrow \Gr(2,5)$ is the differential construction $D^{2}_{4}$.
\end{proposition}
\begin{proof}
  Let $\gamma : \P^{1} \to \P^{2}$ be a general map induced by a three dimensional vector space $W \subset H^{0}(\O_{\P^{1}}(4))$ having image $R$, and let $(q_{1},q_{2},q_{3})$ be $\nu^{-1}(\varphi)$. The pencil $D(\gamma)$ is cut out by the perspective conics. \todo{Why?} According to \autoref{theorem:perspectiveconics}, the linear series on $R$ cut out by perspective conics is $\O_{R}(1) \otimes \eta$, where $\eta$ is the distinguished element $(-1,-1,-1) \in \Pic(R)[2]$.
  If the space of sections of the line bundle $\O_{R}(1)$ is identified with $\nu(q_{1},q_{2},q_{3})$, then it follows that the space of sections of the twist $\O_{R}(1)\otimes \eta$ equals $\pi(q_{1},q_{2},q_{3})$. 
\end{proof}

\begin{definition}
  \label{definition:jacobiantwoquadrics}
  Let $\{a(x,y),b(x,y)\}$ be two homogeneous quadratic polynomials with no common zeros. Their {\sl Jacobian} is
  \begin{align*}
    J(a,b):= a_{x}b_{y}-a_{y}b_{x}.
  \end{align*}

\end{definition}

Note that the Jacobian vanishes precisely at the two branch points of the map $[x : y] \mapsto [a(x,y):b(x,y)]$.

\begin{theorem}
  \label{theorem:onlyapencil}
  Let $\left\{ q_{1},q_{2},q_{3} \right\} \in \Sym^{3}\P^{2}$ have six distinct roots. Then the vector space
  \begin{align*}
    \langle q_{1}J(q_{2},q_{3}), q_{2}J(q_{1},q_{3}), q_{3}J(q_{1},q_{2}) \rangle 
  \end{align*}
  is equal to  $\pi(q_{1},q_{2},q_{3}) \in \Gr(2,5)$.
\end{theorem}
\begin{proof}
  By $SL_2(k)$-equivariance, it suffices to prove the theorem for three quadratic functions $\left\{ xy, q_{2}, q_{3} \right\}$ where $q_{2}$ and $q_{3}$ are general.
 
  Let $\alpha_{1}, \alpha_{2}$, and $\beta_{1}, \beta_{2}$ denote the roots of $q_{2}, q_{3}$. Note that these roots are assumed to be in ${\bf A}^{1} \subset \P^{1}$.  We let $t = x/y$ denote the affine coordinate.

  The vector space $\Pi := \pi(t, q_{2}(t), q_{3}(t))$ is equal to the vector space of forms
  \begin{align*}
    \omega = \frac{f(t)dt}{tq_{2}(t)q_{3}(t)},
  \end{align*}
  with $\deg(f) \leq 4$, and with the additional constraints

 \begin{align*}
    \res_{\alpha_{1}}\omega = \res_{ \alpha_{2}} \omega\\
    \res_{\beta_{1}}\omega = \res_{ \beta_{2}} \omega\\
    \res_{0}\omega = \res_{\infty} \omega
  \end{align*}

  Since we know a priori that the space of such forms is two dimensional, we conclude in particular that there exists a nonzero $\omega \in \Pi$ which is nonzero and vanishing at $\alpha_{1}$.  However, the first residue condition then forces $\omega$ to vanish at $\alpha_{2}$ as well. (This is clear from the geometry: an element of the pencil of perspective conics is cut out by a (possibly singular) conic in $\P^{2}$. If it contains a node, then its pullback to $\P^1$ must vanish at both points above the node.)

  Therefore, there exists an $\omega \in \Pi$ of the form $$ \omega = \frac{(t-\alpha_{1})(t-\alpha_{2})g(t)dt}{tq_{2}q_{3}} = \frac{g(t)dt}{tq_{3}}.$$
  The residue conditions at $\beta_{i}$, and $0, \infty$ together imply, up to nonzero scaling,
  \begin{align*}
    g(t) = t^{2} - \beta_{1}\beta_{2}.
  \end{align*}
  The roots $\pm \sqrt{\beta_{1}\beta_{2}}$ are precisely the branch points of the map $[x:y] \to [xy : q_{3}]$.  Therefore $\omega$ vanishes at the roots of the quartic polynomial $q_{1}j(xy,q_{3})$. The theorem follows by arguing in the same manner for the two other pairs of roots.

\end{proof}

Given a general triple $\{a,b,c\}$ of binary quadratic forms, we can create the three quartic binary forms $a[b,c], b[c,a], c[a,b]$, where $[p,q]$ denotes $p_{x}q_{y} - p_{y}q_{x}$.  As we know, these three forms are actually linearly dependent, yielding a pencil of binary quartics. 

In this way, we obtain an {\sl a priori} rational map
\begin{align*}
 	D: \Hilb^{3}(\P^{2}) \dashrightarrow \Gr(2,5)
 \end{align*} 
 where the domain is the Hilbert scheme of $3$ points on $\P^{2}$. 

 The main observation is: 
 \begin{proposition}
 	\label{proposition:Dregular}
 	The rational map $D$ extends to a regular map.
 \end{proposition}

 \begin{proof}
 	This is best seen by describing $D$ geometrically, and noting that the geometric construction makes sense at every point of $H$.

 	If $\{a,b,c\}$ is a general subset of $\P^{2}$, then the quartic pencil $D(\{a,b,c\})$ is obtained as follows.  Recall that in $\P^{2}$ we have the canonical discriminant conic $C$ parametrizing square forms. A point $a \in \P^{2}$ defines a line $Pol(a) \subset \P^{2}$ spanned by the two points of $C$ which correspond to the roots of $a$. Furthermore, a pair of points $b,c \in \P^{2}$ defines the line $\overline{b,c} \subset \P^{2}$. 

 	To the triple $\{a,b,c\}$ we attach the triple of pairs of lines $Pol(a) \cup \overline{b,c}$ (and permutations), which cut the conic $C$ at $3$ members of a degree $4$ pencil. 

 	This geometric construction works even for non-reduced schemes.  For example, if $Z \subset \P^{2}$ is a fat point concentrated at a point $a \in \P^{2}$, we assign the degree $4$ pencil on $C$ as: The degree $2$ pencil corresponding to $Pol(a)$ with two base points at $Pol(a) \cap C$.
 \end{proof}

 The map $D$ is only generically finite; the locus of collinear triples is contracted, and has the same image as the locus of fat schemes.   However, it is easy to exhibit a point in $G$ over which there are exactly two preimages. 

 \begin{lemma}
 	\label{lemma:TwoPreimages}
 	Let $\Lambda \in \Gr(2,5)$ denote the unique pencil of binary quartics with simple base points at $0,1,\infty$ in $\P^{1}$. Then the preimage $D^{-1}(\Lambda)$ consists of two non-reduced points.
 \end{lemma}

 \begin{proof}
 	The two configurations are described as follows: View the three points $0,1,\infty$ on the diagonal conic $C$. Then the triple $\{0,1,\infty\}$ clearly maps to $\Lambda$, as does the triangle created by $Pol(0), Pol(1), Pol(\infty)$. 

 	A simple infinitesimal calculation shows any non-trivial first-order deformation of either of these configurations will have the effect of either removing the base-points, or moving their location. 

 	Furthermore, it is clear that these are the only two possible configurations giving rise to the pencil $\Lambda$. 
 \end{proof}

The previous lemma immediately gives:

\begin{theorem}
  	Let $X \subset \P^{5}$ be a balanced quartic surface scroll. Then $\deg \rho_{X} = 2$.
  \end{theorem}  


\subsection{Eccentric surface scrolls} % (fold)
\label{sub:surfaces}

% subsection surfaces (end)
Let $E = \O(1) \oplus \O(k+1)$, $X = \P E$. Choose an affine coordinate $t$ on $\P^{1}$, and consider the projection-ramification enumerative problem for $X \subset \P^{k+3}$.  We claim: 
\begin{proposition}\label{proposition:rhobirationalsurfaces}
Maintaining the setting above,	$\rho_{X}$ is birational.
\end{proposition}


Let $A= H^{0}(\O_{X}(1))$. This vector space will be identified with the space of expressions of the form $\ell(t)x_{1} + q_{k+1}(t)x_{2}$, where $\ell, q_{k+1}$ are polynomials of degrees at most $1$ and $k+1$ respectively. In what follows, subscripts of polynomials in $t$ represent the degree.

If $W \subset A$ is a general three dimensional vector space, then there will be a unique triple of elements in $W$ of the form
\begin{align*}
	w_{0} &= t(x_{1} + q_{k}(t)x_{2})\\
	w_{\infty} &= (x_{1} + r_{k}(t)x_{2})\\
	w_{*} &= s_{k+1}(t)x_{2}
\end{align*}

The Wronski determinant of this triple is: 

\begin{align}\label{equation:jacobsurface}
	sx_{1} + \left[s(qt)' - s'(qt) - t(r's-s'r)\right]x_{2}
\end{align}

\begin{proof}[Proof of \autoref{proposition:rhobirationalsurfaces}]
	Let $r := \sigma x_{1} + \tau x_{2} \in H^{0}(X,\O(R))$ be a general element, we can extract the unique vector space $W$ obeying $\rho_{X}(W) = [r] \in |R|$ as follows: First, we set $s := \sigma$. Secondly, given $s$, the equation $\left[s(qt)' - s'(qt) - t(r's-s'r)\right] = \tau$ is a system of $2k+2$ linear equations involving the $2k+2$ coefficients of the pair $(q,r)$. We know (from \autoref{thm:main})  this system has a finite, positive number of solutions. Hence it must have a unique solution, proving the proposition.
\end{proof}

\subsection{Eccentric threefold scrolls} % (fold)
\label{sub:eccentric_threefolds}
Now let $E = \O(1) \oplus \O(1) \oplus \O(k+1)$, $k \geq 0$, and set $X := \P E$.  Embed $X \subset \P^{k+5}$ via the natural $\O(1)$ on $X$. Again, we choose affine coordinate $t \in \P^{1}$ and relative coordinates $x_{1},x_{2},x_{3}$ on $X$ corresponding to the three factors of the splitting of $E$. 

\begin{proposition}\label{proposition:threefold}
Maintain the setting above. Then	$\rho_{X}$ is birational.
\end{proposition}
% subsection eccentric_threefolds (end)

Suppose $W \subset H^{0}(E)$ is a general $4$ dimensional vector space. Then the projection $W \to H^{0}(\O(1)\oplus \O(1))$ will be an isomorphism.  Hence, there will be $4$ uniquely defined elements of $W$ of the form: 
\begin{align*}
	x_{1} + ax_{3}\\
	x_{2} + bx_{3}\\
	tx_{1} + cx_{3}\\
	tx_{2} + d x_{3}
\end{align*}
where $a,b,c,d$ are degree $\leq k+1$ polynomials in $t$.
The Wronski determinant for this tuple of equations is: 
\begin{align}\label{eq:jacobianthreefold}
	\alpha x_{1} + \beta x_{2} + \gamma x_{3} = (d-bt)x_{1} + (at-c)x_{2} + \left[a't(bt-d) + b't(c-at) + c'(d-bt)+ d'(at-c) \right]x_{3}.
\end{align}

\begin{proof}[Proof of \autoref{proposition:threefold}]
	We replace the Grassmannian $\Gr(4,H^{0}(E))$ with the affine open subset $\A^{4k+8}$ parametrizing quadruples $(a,b,c,d)$. Then the ramification divisor equation \eqref{eq:jacobianthreefold} defines a map 
	\begin{align*}
	 	\rho^{*}: \A^{4k+8} \to \A^{4k+9}
	 \end{align*} 
	 where the latter $\A^{4k+9}$ is the vector space of triples $(\alpha,\beta,\gamma)$ with $\deg \alpha, \beta \leq k+2$ and $\deg \gamma \leq 2k+2$. The projection-ramification $\rho_{X}$ map $\rho$ is recovered by composing $\rho^{*}$ with the projection $\A^{4k+9} \dashrightarrow \P^{4k+8}$. 

	 First, if $(a,b,c,d)$ are general, then one can directly use the relative primeness of $d-bt$ and $at-c$ (we omit this simple calculation) to conclude that $\rho^{*}$ is generically injective on tangent spaces, and hence the generic fiber of $\rho^{*}$ is finite.


	 We next show  $\rho_{X}$ is dominant. In light of the previous paragraph, it suffices to prove: If $(\alpha, \beta, \gamma)$ is a general point in the image of $\rho^{*}$, and $\lambda \neq 0,1$ is a constant, then $\lambda(\alpha, \beta, \gamma)$ is not in the image of $\rho^{*}$. 

	 To this end, suppose $(a,b,c,d)$ is a general point in $\A^{4k+8}$. Then  $\alpha := d-bt$ and $\beta := at-c$ will be degree $k+2$ polynomials which are relatively prime.  


	 For any polynomial $p(t)$, let $p^{+}$ denote the highest degree coefficient of $p$. Observe that $\beta^{+} = a^{+}$.  Furthermore, the expression for $\gamma$ is easily seen to be 
	 \begin{align}\label{gammaEq}
	  	\gamma = (\alpha'\beta - \beta' \alpha) + \alpha a + \beta b
	  \end{align} 
	  where $'$ denotes $d/dt$. 

	  If we scale by $\lambda$, we get: 
	  \begin{align}
	      \label{firstEquations}
	  	\lambda \alpha &= \lambda (d-bt)\\
	  	\lambda \beta &= \lambda (at-c) \nonumber\\
	  	\lambda \gamma &= \lambda(\alpha'\beta - \beta' \alpha) + \lambda \alpha a + \lambda \beta b \nonumber
	  \end{align}

	  At the same time, if $\lambda(\alpha, \beta, \gamma)$ is also realized by some quadruple $(\tilde{a}, \tilde{b}, \tilde{c}, \tilde{d})$ then we get the equations: 
	  \begin{align}\label{secondEquation}
	  	\lambda \alpha &= \tilde{d} - \tilde{b}t\\
	  	\lambda \beta &= \tilde{a}t - \tilde{c} \nonumber\\
	  	\lambda \gamma &= \lambda^{2}(\alpha'\beta - \beta' \alpha) + \lambda \alpha \tilde{a} + \lambda \beta \tilde{b}\nonumber
	  \end{align}
	  The second equation gives $\tilde{a}^{+} = \lambda \beta^{+}$.  The last equation gives: $\gamma = \lambda(\alpha'\beta - \beta' \alpha) + \alpha \tilde{a} + \beta \tilde{b}$.  Combining with \eqref{gammaEq}, we get 
	  \begin{align*}\label{alphasbetas}
	    	\alpha (a - \beta') + \beta (b + \alpha') &= \alpha(\tilde{a} - \lambda\beta') + \beta(\tilde{b} + \lambda \alpha').
	    \end{align*} 
	    Since $\alpha$ and $\beta$ are relatively prime and have degree greater than $a,b,\tilde{a},\tilde{b}$, we deduce:
	    \begin{align*}
	     	a-\beta' &= \tilde{a} - \lambda \beta'\\
	     	b+\alpha' &= \tilde{b} + \lambda \alpha'
	     \end{align*} 
	     By examining top coefficients, and using $a^{+} = \beta^{+}$, $\tilde{a}^{+} = \lambda \beta^{+}$ we get: 
	     \begin{align*}
	     	\beta^{+} - (k+2)\beta^{+} &= \lambda\beta^{+} - \lambda(k+2)\beta^{+}
	     \end{align*}
	     or 
	     \begin{align*}
	     	(1-\lambda)\beta^{+} &= (1-\lambda)(k+2)\beta^{+}
	     \end{align*}
	     Given our assumption on $\lambda$, this is only possible if $\beta^{+} = 0$.  However, since $(a,b,c,d)$ were chosen generically, $\beta^{+} = a^{+}$ would not be zero, providing our desired contradiction.

	     Finally, we argue $\deg \rho_{X} = 1$. It suffices to show that a general ramification equation $\alpha x_{1} + \beta x_{2} + \gamma x_{3}$ of the form \eqref{eq:jacobianthreefold} arises from a unique choice of polynomials $(a,b,c,d)$.  The conditions $d-bt = \alpha$ and $at-c=\beta$ produce an affine linear subspace $\Lambda$ in the vector space of choices $(a,b,c,d)$. With respect to linear coordinates on $\Lambda,$ the expression for $\gamma$ is also linear, and hence the available choices of $(a,b,c,d)$ producing \autoref{eq:jacobianthreefold} is an intersection of affine linear spaces.  Since we already know $\rho^{*}$ is generically finite, it follows that $\deg \rho_{X} = 1$ as desired.

\end{proof} 

Since every smooth three dimensional rational normal scroll specializes isotrivially to the scroll $X$ in \autoref{proposition:threefold}, we immediately get: 

\begin{corollary}\label{corollary:maxVariation3Scrolls}
	The projection-ramification map $\rho_{X}$ is dominant for every smooth three dimensional rational normal scroll $X \subset \P^{n}$.
\end{corollary}

\subsection{Recasting the projection-ramification map for scrolls} Let $E$ be a rank $r$ ample vector bundle on $\P^{1}$, and set $X = \P E$.  Then a general $r+1$-dimensional subspace 
\begin{align*}
    W \subset H^{0}(X, \O(1)) = H^{0}(\P^{1},E)
\end{align*}
yields a short exact sequence 
\begin{align*}
    0 \to (\det E)^{-1} \to W \otimes \O_{\P^{1}} \to E \to 0
\end{align*}
which corresponds to an element $w$ (up to scalar) of the extension space $\Ext^{1}(E,(\det E)^{-1})$. The assignment $W \mapsto [w] \in \P(\Ext^{1}(E,(\det E)^{-1}))$ is easily seen to be a birational map between $\G := \Gr(n+1,H^{0}(E))$ and $\P(\Ext^{1}(E,(\det E)^{-1}))$.

The ramification linear series $|R|$ is the projectivization of the vector space $V = H^{0}(E \otimes \det E \otimes K_{\P^{1}})$.  By Serre duality, $V$ is dual to $\Ext^{1}(E,(\det E)^{-1})$.  Therefore, the projection-ramification map $\rho_{X}$ may be recast as a map
\begin{align*}
    \delta_{X}: \P(V^{*}) \dashrightarrow \P(V)
\end{align*}



\section{Further Questions}

\begin{enumerate}
    \item How many of our theorems are valid in characteristic $p > 0$?  
    \item When $\dm \Gr < \dm |R|$ and $\rho_{X}$ is generically finite onto its image, then is $\rho_X$ birational onto its image?
    \item Is $\Gr(2,4)$ the only incompressible Grassmannian?
    \item Is it possible to classify the scrolls for which $\deg \rho_{X} = 1$?
    \item Is there an analogous characterization of varieties of minimal degree using ``higher codimension" ramification loci?
\end{enumerate}









 \bibliographystyle{amsalpha}
   \bibliography{CommonMath}

\end{document}
