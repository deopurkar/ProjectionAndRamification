\documentclass[11pt,reqno]{amsart}
\usepackage[margin = 1.3 in]{geometry}
%\usepackage[frenchmath,defaultmathsizes]{mathastext}
\usepackage{
  hyperref,
  amsmath,
  amssymb,
  amsthm,
  thmtools,
  microtype,
  mathrsfs,
  enumitem,
  stmaryrd,
  diagbox
}
\usepackage[draft]{showlabels}
\usepackage[table]{xcolor}
\usepackage{tikz, tikz-cd}

\setlength{\parskip}{.25em}

\newcommand*\justify{%
  \fontdimen2\font=0.4em% interword space
  \fontdimen3\font=0.2em% interword stretch
  \fontdimen4\font=0.1em% interword shrink
  \fontdimen7\font=0.1em% extra space
  \hyphenchar\font=`\-% allowing hyphenation
}


\usepackage{graphicx}

\linespread{1.15}

\usepackage{eucal}

%\usepackage[all]{xy}

%\usepackage[draft]{showlabels}

\theoremstyle{plain}
\newtheorem{theorem}{Theorem}[section]
\newtheorem{proposition}[theorem]{Proposition}
\newtheorem{lemma}[theorem]{Lemma}
\newtheorem{conjecture}[theorem]{Conjecture} 
\newtheorem{corollary}[theorem]{Corollary}
\theoremstyle{definition}
\newtheorem{definition}[theorem]{Definition}
\newtheorem{observation}{Observation}
\newtheorem{example}[theorem]{Example}
\newtheorem{exercise}[theorem]{Exercise}
\newtheorem{counterexample}[theorem]{Counterexample}
\newtheorem{convention}[theorem]{Convention}
\newtheorem{question}[theorem]{Question}
\theoremstyle{remark}
\newtheorem{notation}[theorem]{Notation}
\numberwithin{equation}{section}



\title{Projection and Ramification}
\author{Anand Deopurkar, Eduard Duryev, \& Anand Patel}



%-----------------------------------------------
\def\labelitemi{--}
\newcommand{\todo}[1]{\fbox{ToDo: #1}}
\renewcommand{\k}{k}
\DeclareMathOperator{\id}{id}
\DeclareMathOperator{\Bl}{Bl}
\DeclareMathOperator{\Br}{Br}
\DeclareMathOperator{\ProjBun}{ProjBun}
\renewcommand{\Vec}{\operatorname{Vec}}
\DeclareMathOperator{\Def}{Def}
\DeclareMathOperator{\res}{Res}
\DeclareMathOperator{\Quot}{Quot}
\DeclareMathOperator{\Hilb}{Hilb}
\DeclareMathOperator{\sing}{Sing}
\DeclareMathOperator{\dm}{dim}
\DeclareMathOperator{\F}{\mathbf F}

\newcommand{\cO}{{\mathcal O}}
\renewcommand{\to}{{\longrightarrow}}

\renewcommand{\sectionautorefname}{\S}
\renewcommand{\subsectionautorefname}{\S}
% Let us keep this minimial
% Let us also define things only if they are previously undefined.

% Common theorem-like environments
\ifcsname theorem\endcsname{}\else\declaretheorem[parent=section]{theorem}\fi
\ifcsname corollary\endcsname{}\else\declaretheorem[sibling=theorem]{corollary}\fi
\ifcsname lemma\endcsname{}\else\declaretheorem[sibling=theorem]{lemma}\fi
\ifcsname proposition\endcsname{}\else\declaretheorem[sibling=theorem]{proposition}\fi
\ifcsname conjecture\endcsname{}\else\declaretheorem[sibling=theorem]{conjecture}\fi
\ifcsname problem\endcsname{}\else\declaretheorem[sibling=theorem]{problem}\fi
\ifcsname question\endcsname{}\else\declaretheorem[sibling=theorem]{question}\fi
\ifcsname definition\endcsname{}\else\declaretheorem[sibling=theorem, style=definition]{definition}\fi
\ifcsname exercise\endcsname{}\else\declaretheorem[sibling=theorem, style=definition]{exercise}\fi
\ifcsname example\endcsname{}\else\declaretheorem[sibling=theorem, style=definition]{example}\fi
\ifcsname remark\endcsname{}\declaretheorem[sibling=theorem, style=remark]{remark}\fi

% Common abbreviations

% Absolutely standard rings and fields
\providecommand {\N}{{\bf N}}
\providecommand {\Z}{{\bf Z}}
\providecommand {\Q}{{\bf Q}}
\providecommand {\R}{{\bf R}}
\providecommand {\C}{{\bf C}}

% Common spaces grassmannian
\renewcommand {\P}{{\bf P}}
\providecommand {\Gr}{{\bf Gr}}
\providecommand {\A}{{\bf A}}

% Groups
\providecommand{\SL}{\operatorname{SL}}
\providecommand{\GL}{\operatorname{GL}}
\providecommand{\PGL}{\operatorname{PGL}}
\providecommand{\Gm}{{\bf G}_m}

% f \from G \to H reads much better than f \colon G \to H
\providecommand {\from}{{\colon}}

% Absolutely standard notation
\providecommand{\spec}{\operatorname{Spec}}
\providecommand{\proj}{\operatorname{Proj}}
\providecommand{\coker}{\operatorname{coker}}
% Kernel is already defined
\providecommand{\Blowup}{\operatorname{Bl}}
\providecommand{\Hom}{\operatorname{Hom}}
\providecommand{\Ext}{\operatorname{Ext}}
\providecommand{\Tor}{\operatorname{Tor}}
\providecommand{\End}{\operatorname{End}}
\providecommand{\Aut}{\operatorname{Aut}}
\providecommand{\codim}{\operatorname{codim}}
% Dim is already defined
\providecommand{\Pic}{\operatorname{Pic}}
\providecommand{\Sym}{\operatorname{Sym}}
\providecommand{\rk}{\operatorname{rk}}
\declaretheorem[sibling=theorem,style=remark]{remark}
\numberwithin{equation}{section}
\declaretheorem[title=Theorem, style=plain]{maintheorem}
\declaretheorem[title=Theorem-Example, style=plain]{exampletheorem}
\renewcommand{\themaintheorem}{\Alph{maintheorem}}
\renewcommand{\theexampletheorem}{\Alph{exampletheorem}}

\renewcommand{\O}{\mathcal O}
\newcommand{\G}{\mathbf G}
\newcommand{\td}{\widetilde}
\newcommand{\Frac}{{\mathrm{Frac}}\,}
\newcommand{\Jac}{{\textrm{Jac}}}
\DeclareMathOperator{\Ram}{Ram}
\newcommand{\fm}{\mathfrak m}
\newcommand{\smvee}{\raise0.5ex\hbox{$\scriptscriptstyle\vee$}}
\DeclareMathOperator{\ord}{ord}

\newcommand{\compl}[1]{\widehat{#1}}
\newcommand{\Spec}{{\text{\rm Spec}\,}}
\newcommand{\Spf}{{\text{\rm Spf}\,}}
\renewcommand {\o}[1]{\overline{#1}}
\newcommand{\Proj}{{\text{\rm Proj}\,}}
\newcommand{\git}{\sslash}
% -----------------------------------------------

%%% BEGIN DOCUMENT

\begin{document}

\begin{abstract}
    When a projective variety is linearly projected to a projective space of the same dimension, a ramification divisor forms. We study basic properties of this projection-ramification assignment, and uncover enumerative phenomena extending the classical appearance of Catalan numbers in the geometry of rational normal curves.
\end{abstract}


\maketitle
\tableofcontents


\section{Introduction}\label{sec:intro}
Let $X \subset \P^n$ be a smooth projective variety of dimension $r$, not contained in any hyperplane.
Projection from a general $(n-r-1)$-dimensional linear subspace $L \subset \P^n$ defines a finite surjective map
\begin{align*}
  p_{L}\from X \to \P^{r}.
\end{align*}
Associated to $p_L$ is its ramification divisor $R_L \subset X$.
A simple Riemann--Hurwitz calculation shows that $R_L$ lies in the linear series $|K_X + (r+1)H|$, where $K_X$ is the canonical class, and $H$ is the hyperplane class on $X$.
The association $L \mapsto R_L$ defines a rational map
\[ \rho_X \from \Gr(n-r, n+1) \dashrightarrow |K_X + (r+1)H|, \]
which we call the \emph{projection-ramification} map.
The goal of this paper is to explore the relationship between the geometry of $X$ and the properties of $\rho_X$.

A simple argument shows that $\rho_X$ is itself a linear projection of $\Gr(n-r, n+1)$ in its Pl\"ucker embedding.
When $X$ is a smooth curve over a field of characteristic $0$, the map $\rho_X$ is regular everywhere on $\Gr(n-r, n+1)$.
When $X$ is a rational normal curve, the map $\rho_X$ is dominant.
In this case, the ramification divisor of a map $\P^1 \to \P^1$ of degree $n$ represented by the rational function $f/g$ is cut out by the \emph{Wronskian} expression, namely the degree $(2n-2)$ polynomial $f'g - g'f$.
Since $\rho_X$ is regular, its degree is the degree of the Grassmannian, which in this case is the Catalan number $\frac{(2n-2)!}{n!(n-1)!}$.
When $X$ has dimension 2 or more, $\rho_X$ may not be regular on the entire Grassmannian, which makes it difficult to understand.
Nevertheless, it appears that the geometry of $\rho_X$ is related to some fascinating areas of classical projective geometry, and the enumerative questions surrounding $\rho_X$ hint at a rich underlying structure. 

\subsection{Maximal variation}
Our focus is the following question.
\begin{question}\label{q:maxvar}
  Is $\rho_{X}$ generically finite onto its image? In other words, does the image of $\rho_X$ have maximal possible dimension?
\end{question}

To our knowledge, this question first appeared in the work of Flenner and Manaresi \cite{MANAFlenn:}.
Our first result answers this question affirmatively for a large class of varieties.
We say that $X \subset \P^n$ is \emph{incompressible} if for every $(n-r-1)$-dimensional linear subspace $L \subset \P^n$, the projection map $p_L \from X \dashrightarrow \P^r$ is dominant.
Recall that the dual variety $X^* \subset {\P^n}^*$ is the closure of the locus of hyperplanes in $\P^n$ whose intersection with the smooth part of $X$ is singular.
\begin{maintheorem}\label{theorem:Main}
  Let $X \subset \P^n$ be a non-degenerate, normal, projective variety over a field of characteristic zero.
  Suppose at least one of the following holds:
  \begin{enumerate}
  \item\label{item:incomp} $X$ is incompressible, 
  \item\label{item:dual} the dual variety $X^* \subset {\P^n}^*$ is a hypersurface.
  \end{enumerate}
  Then $\rho_{X}$ is generically finite onto its image.
\end{maintheorem} 
We do not assume that $X$ is smooth in the statement of \autoref{theorem:Main}.
This requires defining $\rho_X$ more carefully.
To state the conclusion informally, if we move a generic $L \subset \P^n$ of complementary dimension, then the ramification locus $R_L \subset X$ also moves.

The hypotheses in \autoref{theorem:Main} are sufficient, but not necessary.
Indeed, consider $X = \P^{r-1} \times \P^1 \subset \P^{2r-1}$, embedded by the Segre embedding, for $r \geq 3$.
Then $X$ is neither incompressible nor is $X^*$ a hypersurface, and yet $\rho_X$ is dominant (see \autoref{Thm:Examples}).

To our knowledge, the known results about maximal variation operate under condition \eqref{item:incomp} in \autoref{theorem:Main}.
For example, in \cite{MANAFlenn:}, the authors deduce maximal variation under the condition that for every $(n-r-1)$-dimensional linear subspace $L \subset \P^n$, the join $J(L, X)$ equals $\P^n$, or under the condition that $X$ is smooth and the twisted normal bundle $N_{X/\P^n}(-1)$ is ample.
Either condition implies that $X$ is incompressible, and hence falls under condition \eqref{item:incomp}.
If $X$ is a curve or a smooth complete intersection, then $X$ is incompressible, and covered by condition \eqref{item:incomp}.

\autoref{theorem:Main} substantially increases the class of varieties where we now know maximal variation.
For example, it is easy to see that if $X$ is a smooth surface over a field of characteristic $0$, then $X^*$ is a hypersurface.
Therefore, maximal variation holds for all surfaces, although incompressibility may not (The cubic surface scroll $X \subset \P^4$ is the smallest counter example, as the projection from the directrix line of $X$ is not dominant).
As another source of new examples, take a sufficiently high degree Veronese re-embedding $X \subset \P^N$ of any smooth $X$.
Then $X^*$ is divisorial, and hence $X$ is covered under \autoref{theorem:Main}.
But $X \subset \P^N$ will be compressible.

Given that maximal variation holds in such a large class of varieties, it is natural to wonder if it always holds.
This is not the case.
\begin{maintheorem}
  \label{Thm:Counterexamples}
  There exist smooth, non-degenerate, rational normal scrolls $X^{r} \subset \P^{n}$ of every dimension $r \geq 4$ such that the projection-ramification map $\rho_{X}$ is not generically finite onto its image.
\end{maintheorem}
\autoref{Thm:Counterexamples} provides the first known examples of varieties with non-maximal variation of ramification divisors.
We describe the rational normal scrolls in \autoref{Thm:Counterexamples} in \autoref{sec:proof_of_second_result}; they include some of general moduli.

We now turn our attention to cases where the projection-ramification map $\rho_X$ may be dominant.
The next result classifies $X \subset \P^n$ for which the source and the target of $\rho_X$ are of the same dimension.
\begin{maintheorem}\label{theorem:minimaldegree}
  Let $X \subset \P^{n}$ be a smooth, non-degenerate projective variety of dimension $r$ over a field of characteristic zero.
  We have the inequality
  \[ \dim \Gr(n-r, n+1) \leq \dim |K_X + (r+1)H|,\]
  where equality holds if and only if $X$ is a variety of minimal degree, namely $\deg X = n-r+1$.
\end{maintheorem}
Recall the list of smooth varieties of minimal degree: rational normal curves, quadric hypersurfaces, the Veronese surface in $\P^5$, and rational normal scrolls.
By \autoref{theorem:Main}, $\rho_X$ is dominant for the first three, so we are led to investigate the scrolls.
It came to us as a surprise that $\rho_X$ is \emph{not} dominant for all scrolls (see \autoref{Thm:Counterexamples}).
Nevertheless, it is dominant for most scrolls, which we now make precise.

Recall that if $X \subset \P^n$ is a smooth rational normal scroll, then $X$ is isomorphic to the projectivization of an ample vector bundle $E$ on $\P^1$, and the embedding is given by the complete linear series $|\O_{\P E}(1)|$.
\begin{maintheorem}
  \label{theorem:rationalnormalscrolls}
  Let $X = \P E \subset \P^n$ be a rational normal scroll, where $E$ is a ample vector bundle of rank $r$ on $\P^1$, general in its moduli.
  If $\deg E = a \cdot (r-1) + b \cdot (2r-1) + 1$ for non-negative integers $a, b$, then the projection-ramification map $\rho_X$ is dominant for $X$.
  In particular, the conclusion holds if $E$ is general of degree at least $(r-1)(2r-1) + 1$.
\end{maintheorem}
Thus, at least among the general scrolls, the projection-ramification map is dominant except possibly in small degrees.
We prove \autoref{theorem:rationalnormalscrolls} by degeneration, using the theory of limit linear series of higher rank developed by Osserman \cite{oss:14}.

\subsection{Enumerative problems}
\autoref{theorem:minimaldegree} and \autoref{theorem:rationalnormalscrolls} motivate a gamut of enumerative questions.
\begin{question}\label{q:degree}
 When $X \subset \P^n$ is a variety of minimal degree, what is the degree of $\rho_X$?
\end{question}


The following result summarizes our knowledge of the answers to \autoref{q:degree}.
\begin{maintheorem}\label{Thm:Examples}\mbox{}
\begin{enumerate}
  \item If $X \subset \P^{n}$ is a rational normal curve, then $\rho_X$ is regular and $\deg \rho_{X} = \frac{(2n-2)!}{n!(n-1)!}$.
  \item  If $X \subset \P^{n}$ is a quadric hypersurface, then $\rho_{X}$ is an isomorphism.

  \item  If  $X = \P^{r-1} \times \P^{1} \hookrightarrow \P^{2r-1}$ is the Segre embedding, then $ \rho_{X}$ is birational.

  \item  If $X \subset \P^{5}$ is the Veronese surface, then $ \deg \rho_{X} = 3$.
  \item If $X \subset \P^{5}$ is a general quartic surface scroll, then $\deg \rho_{X} = 2$.
  \item If $X = \P(\O_{\P^{1}}(1) \oplus \O_{\P^{1}}(k+1)) \subset \P^{k+3}$ is the surface scroll with most imbalanced splitting type, then $\rho_{X}$ is birational.
  \item If $X = \P(\O_{\P^{1}}(1) \oplus \O_{\P^{1}}(1) \oplus \O_{\P^{1}}(k+1)) \subset \P^{k+5}$ is the threefold scroll with most imbalanced splitting type, then $\rho_{X}$ is birational.
\end{enumerate} 
\end{maintheorem}

For $X$ of dimension $1$, namely a rational normal curve, the projection-ramification map
\[ \rho_X \from \Gr(2, n+1) \to \P^{2n-2}\]
is regular, and defined by the Pl\"ucker line bundle on the Grassmannian.
Therefore, its degree is the top self-intersection of the Pl\"ucker line bundle, which in this case is the Catalan number $\frac{(2n-2)!}{n!(n-1)!}$.

For $X$ of codimension $1$, namely a quadric hypersurface, the projection-ramification map
\[ \rho_X \from \Gr(n, n+1) = \P^{n} \to {\P^n}^* \]
is again regular, and is in fact the duality isomorphism induced by the (non-degenerate) quadric $X$.

The case of the Veronese surface and of the quartic surface scroll in \autoref{Thm:Examples} are particularly delightful; these are treated in \autoref{sec:enumerativeproblems}.
They involve intricate classical projective geometry that intertwines cubic plane curves, Steinerians and Cayleyans, and apolarity.

The cases of the most unbalanced surface and threefold scrolls follow from direct calculation.
Note, however, that for the most unbalanced scroll in dimension 4 and higher, the projection-ramification map is not dominant.
For scrolls, $\rho_X$ is not regular.
Furthermore, the complexity of the base locus of $\rho_X$ effectively blocks any straightforward application of the excess intersection formula.

A smooth rational normal scroll $X \subset \P^n$ of degree $d$ and dimension $r$ is isomorphic to the projectivization of an ample vector bundle $E$ on $\P^1$, which in turn is isomorphic to a direct sum $\O(a_1) \oplus \dots \oplus \O(a_r)$ for positive integers $a_1, \dots, a_r$ satisfying $d = a_1 + \dots + a_r$.
Let $\Sigma_{r,d}$ be the set of $r$-term partitions of $d$.
We get a function $\rho \from \Sigma_{r,d} \to \Z_{\geq 0}$ defined by
\[ \rho(a_1, \dots, a_r) = \deg \rho_X,\]
for $X = \P (\O(a_1) \oplus \dots \oplus \O(a_r))$.
The set $\Sigma_{r,d}$ has a partial ordering $\prec$ given by dominance.
If $(a_1, \dots, a_r) \prec (b_1, \dots, b_r)$, then the scroll $\P(\O(b_1) \oplus \dots \oplus \O(b_r))$ isotrivially specializes to the scroll $\P(\O(a_1) \oplus \dots \oplus \O(a_r))$.
By the lower semi-continuity of degrees of rational maps, we get
\[ \rho(a_1, \dots, a_r) \leq \rho(b_1,\dots, b_r).\]
\autoref{theorem:rationalnormalscrolls} implies that, at least if $d$ is sufficiently large compared to $r$, then $\rho$ is not identically zero.
\autoref{Thm:Examples} determines the value of $\rho$ for the partitions $(n)$, $(1, \dots, 1)$, $(1,k+1)$, $(1,1,k+1)$, and $(2,2)$.
The following table lists some more values of $\rho$ computed using randomized calculations over finite fields using the computer algebra systems \texttt{Macaulay2} and \texttt{MAGMA}.
We plan to return to a more complete enumerative investigation of $\rho$ in a future paper.
\begin{table}
  \centering
  \rowcolors{2}{gray!10}{white}

  \begin{tabular}{l| r r r r}
    \rowcolor{gray!25}
    \diagbox{$a_1$}{$a_2$} & 1 & 2 & 3 & 4\\
    \hline
    1 & 1 & & &\\
    2 & 1 & 2 & &\\
    3 & 1 & 6 & 22 &\\
    4 & 1 & 17 & 92 & 422\\
  \end{tabular}
  
  \caption{Degree of $\rho_X$ for $X = \P(\O(a_1) \oplus \O(a_2))$} \label{tab:computation}
\end{table}


\subsection{Further remarks and questions}
One of the  central enumerative problems concerning branch divisors, originating in the work of Hurwitz, is to compute the number of branched covers of the projective line with specified branch set in $\P^1$.
This number is called the Hurwitz number.
As is well known, the Hurwitz numbers are difficult to compute, but they exhibit remarkable structure.
There is a related question of computing the number of rational functions on $\P^1$ with a prescribed ramification set.
This question is much more elementary, and yields the Catalan numbers, as we have seen.

In higher dimensions, however, the analogue of the Hurwitz problem is expected to be much less interesting, thanks to Chisini's conjecture (now Kulikov's theorem \cite{1064-5632-63-6-A03}).
Kulikov's theorem asserts that a branched cover $S \to P^2$ with generic branching is uniquely determined by its branch divisor $B \subset \P^{2}$, with finitely many well-understood exceptions.
In contrast, the enumerative problem regarding ramification divisors persists in all dimensions, thanks to \autoref{theorem:minimaldegree}, and poses a significant challenge.
In some sense, the ``branch'' and ``ramification'' enumerative stories trade places, at least in terms of difficulty, but perhaps also in terms of structure.

The projection-ramification map generalizes the Wronski map
\[ \rho \from \Gr(2, n+1) \to \P^{2n-2}.\]
The geometry surrounding the Wronski map has received a lot of attention, thanks to the B. and M. Shapiro conjecture.
This conjecture states that the pre-image of any point in $\P^{2n-2}$ defined by a set of $(2n-2)$ real points on $\P^1$ consists entirely of real points in $\Gr(2, n+1)$ \cite{sottile2000} (the conjecture has been proved by Eremenko and Gabrielov \cite{Erem/Gabr1}).
\autoref{theorem:minimaldegree} potentially sets the stage for a higher-dimensional generalization of the body of work around the Shapiro conjecture.

The study of $\rho_X$ in positive characteristic is likely to bring new surprises and require different techniques.
We do not know if \autoref{theorem:Main} or \autoref{theorem:minimaldegree} holds in positive characteristic; our proof certainly does not.
The answers to the enumerative questions \autoref{q:degree} do depend on the characteristic, even in the simplest case of rational normal curves, due to the presence of inseparable covers \cite{MR2218904}.

\subsection{Notation and conventions} We work over an algebraically closed field
$k$ of characteristic $0$ (We use Bertini's theorem and generic smoothness. We
also appeal to the Kodaira Vanishing theorem.) By a {\sl proper variety}, we mean a proper, integral, finite-type $k$-scheme. For any scheme $X$, we let $X^{\rm sm}$ denote its smooth locus. If $F$ is a coherent sheaf, we let $P(F)$ denote its sheaf of principal parts. We will let $e\from H^{0}(X,F) \to P(F)$ denote the natural evaluation morphism -- we suppress the dependence on $F$. If $s$ is a global section of a locally free sheaf, we let $v(s)$ denote the vanishing scheme of $s$.  If $L$ is a line bundle, we let $|L|$ denote the projective space $\P(H^{0}(L))$.  If $L$ is a line bundle on a smooth variety $Y$, and $s \in H^{0}(Y,L)$ is a section, then the {\sl singular scheme} $\sing(v(s))$ of $s$ is the vanishing scheme of $e(s) \in H^{0}(Y,P(L))$; if $K$, the kernel sheaf of $e \from H^{0}(Y,L) \otimes \cO_{Y} \to P(L)$, is locally free, then $\sing(v(s))$ is the largest closed subscheme $T \subset Y$  such that $s\from \cO_{T} \to H^{0}(Y,L) \otimes \cO_{T}$ factors through $K|_{T}$.  

\section{The projection-ramification map}\label{sec:prmap}
In this section, we define a projection-ramification map for a pair $(X, L)$ consisting of a proper, normal, variety $X$ and a sufficiently positive line bundle $L$ on $X$.
For $X \subset \P^n$, taking $L = \O(1)$ recovers the projection-ramification map introduced in \autoref{sec:intro}.
Working with abstract pairs, however, offers more flexibility that is helpful in inductive proofs.

Let $X$ be a proper variety of dimension $r$ over an algebraically closed field $k$ of characteristic zero.
A \emph{linear series} on $X$ is a pair $(L, W)$ consisting of a line bundle $L$ on $X$ and a subspace $W \subset H^0(X, L)$.
The \emph{complete linear series} associated to $L$ is $(L, W)$ with $W = H^0(X, L)$.
A \emph{projection} is a linear series $(L, V)$ with $\dim V = r+1$.
A \emph{projection of $(L, W)$} is a projection $(L, V)$ with $V \subset W$.
As a convention, we use $V$ for projections and $W$ for more general linear series.

\begin{definition}  \label{definition:properlyramified}
We say that a projection $(L,V)$ is \emph{properly ramified} if the evaluation homomorphism
\[e \from V \otimes \O_{X} \to P(L)\]
is an isomorphism over a general point in $X$.  If $(L,V)$ is properly ramified, its \emph{ramification divisor}
\[R(L,V) \subset X\]
is the closure of the scheme defined by the determinant of $e \from V \otimes \O_{X^{\rm sm}} \to P(L)|_{X^{\rm sm}}$.
\end{definition}
In most cases, $L$ is clear from context, so we drop it from the notation and denote the ramification divisor simply by $R(V)$.

A projection $(L, V)$ gives the evaluation map
\[e \from V \otimes \O_X \to L.\]
The evaluation map yields a map $p_{V,L} \from X \dashrightarrow \P V$, regular on the non-empty open set of $X$ where $e$ is surjective.
The following is an easy observation, whose proof we skip.
\begin{proposition}\label{prop:proj}
  The projection $(L, V)$ is properly ramified if and only if the map on tangent spaces induced by $p_{V,L}$ is generically an isomorphism.
  In characteristic zero, this is equivalent to the condition that $p_{V,L}$ is dominant.
\end{proposition}

For a fixed $(L, W)$, the set of all projections of $(L, W)$ are parametrized by the Grassmannian $\Gr(r+1, W)$.
The property of being properly ramified is a Zariski open condition on the Grassmannian.

We now define a map that assigns to a projection its ramification divisor.
To do so, we interpret the ramification divisor as an element of a linear series.

Assume, furthermore, that $X$ is normal.
Let $K_X$ be the canonical sheaf of $X$.
Denoting by $i \from X^{\rm sm} \to X$ the inclusion, $K_X$ is given by the push-forward
\[ K_X = i_* K_{X^{\rm sm}}.\]
Note that, since $X$ is normal, the complement of $X^{\rm sm} \subset X$ has codimension at least 2.
The sheaf $K_X$ is coherent, reflexive, and satisfies Serre's S2 condition.

Let $L$ be a line bundle on $X$.
The sheaf $P(L)$ is locally free of rank $(r+1)$ on $X^{\rm sm}$, and we have a canonical isomorphism
\[ \bigwedge^{r+1} P(L) |_{X^{\rm sm}} \cong K_{X^{\rm sm}} \otimes L^{r+1}.\]
Given a subspace $V \subset H^0(X, L)$, we apply $\bigwedge^{r+1}$ to the evaluation map
\[ e \from V \otimes \O_{X^{\rm sm}} \to P(L)|_{X^{\rm sm}},\]
to get
\[ \det e \from \det V \otimes \O_{X^{\rm sm}} \to K_{X^{\rm sm}} \otimes L^{r+1}. \]
By applying $i_*$ and taking global sections, we get
\[ \det V \to H^0(X, K_X \otimes L^{r+1}).\]
If $(L, V)$ is properly ramified, then this map is non-zero, and hence gives a point of the projective space $\P H^0(X, K_X \otimes L^{r+1})^*$.
Doing the same construction universally over the Grassmannian $\Gr = \Gr(r+1, W)$ yields a map
\begin{equation}\label{eqn:rammap}
  \det \mathcal V \to H^0(X, K_X \otimes L^{r+1}) \otimes \O_{\Gr},
\end{equation}
where $\mathcal V \subset W \otimes \O_{\Gr}$ is the universal sub-bundle of rank $(r+1)$.
Let $U \subset \Gr$ be the open subset of properly ramified projections.
Then the map in \eqref{eqn:rammap} is non-zero at every point of $U$, and defines a map $U \to \P H^0(X, K_X \otimes L^{r+1})^*$ given by the surjection
\begin{equation}\label{eqn:rammapfamily}
  H^0(X, K_X \otimes L^{r+1})^* \otimes \O_{U} \to \det \mathcal V|_U^*.
\end{equation}
Note that $U$ is non-empty if and only if $W$ separates tangent vectors at a general point of $X$.
\begin{definition}
  \label{def:ProjectionRamification}
  Let $(L, W)$ be a linear series that separates tangent vectors at a general point of $X$.
  The \emph{projection-ramification} map for $(L,W)$ is the rational map
  \[
    \rho_{(X,L,W)} \from \Gr(r+1, W) \dashrightarrow \P H^0(X, K_X \otimes L^{r+1})^*
  \]
  defined on the non-empty open subset of properly ramified maps by \eqref{eqn:rammapfamily}.
\end{definition}
If any of $X$, $L$, or $W$ are clear from context, we drop them from the notation.
In particular, for a non-degenerate $X \subset \P^n$, we denote by $\rho_X$ the map $\rho_{X,L,W}$  with $L = \O_X(1)$ and $W$ the image in $H^0(X, L)$ of $H^0(\P^n, \O(1))$.

Note that the map \eqref{eqn:rammapfamily} factors as
\[ \det \mathcal V \xrightarrow{a} \bigwedge^{r+1} W \otimes \O_{\Gr} \xrightarrow{b} H^0(X, K_X \otimes L^{r+1}) \otimes \O_{\Gr},\]
where $a$ is $\wedge^{r+1}$ applied to the universal inclusion $\mathcal V \subset W \otimes \O_{\Gr}$, and $b$ is induced by $\wedge^{r+1}$ applied to the evaluation map $e \from W \otimes \O_{X} \to P(L)$.
The map $a$ defines the Pl\"ucker embedding
\[ i \from \Gr(r+1, W) \to \P \left(\bigwedge^{r+1}W^*\right),\]
and the map $b$ defines a linear projection
\[ p \from \P \left(\bigwedge^{r+1}W^*\right) \dashrightarrow \P H^0(X, K_X \otimes L^{r+1}).\]
Thus, $\rho_{X,L,W}$ factors as the Pl\"ucker embedding followed by a linear projection.

\section{Maximal variation for incompressible and non-defective $X$}
\label{sec:proof_of_theorem:main}
The goal of this section is to prove \autoref{theorem:Main}.
We begin by proving part \eqref{item:incomp}, which is substantially easier.
\begin{proposition}[\autoref{theorem:Main}~\eqref{item:incomp}]
  \label{prop:incompress}
  Let $X \subset \P^n$ be a non-degenerate, normal, incompressible projective variety over a field of characteristic zero.
  Then $\rho_X$ is a finite map.
\end{proposition}
\begin{proof}
  Set $L = \O(1)$ and let $W \subset H^0(X, L)$ be the image of $H^0(\P^n, \O(1))$.
  Let $V \subset W$ be an $(r+1)$-dimensional subspace.
  Since $X$ is incompressible, the projection map $p_{V,L} \from X \dashrightarrow \P V$ induced by $(L, V)$ is dominant.
  By \autoref{prop:proj}, this implies that $(L, V)$ is properly ramified.
  Since $V$ was arbitrary, the projection-ramification map 
  \[ \rho \from \Gr(r+1, W) \to |K_X + (r+1) H|\]
  is regular.
  Since the Picard rank of a Grassmannian is $1$, a regular map from a Grassmannian is either constant or finite.
  It is easy to check that $\rho$ is not constant; so it must be finite.
\end{proof}

For the proof of part \eqref{item:dual} of \autoref{theorem:Main}, we proceed inductively by showing that a general $(n-r-1)$-dimensional linear subspace which is incident to $X$ is an isolated point in its fiber under $\rho$.
Again, it is more convenient to work with the more abstract set-up of a linear series, allowing for series that are not very ample.

Let $X$ be a proper variety of dimension $r$, and let $(L, W)$ be a linear series on $X$.
For an ideal sheaf $I \subset \O_X$ we denote by $W \otimes I$ the subspace of $W$ consisting of the sections that vanish modulo $I$. %he
More precisely, if $K$ is the kernel of the evaluation map
\[ W \otimes \O_X \to L \otimes \O_X/I,\]
then $W \otimes I = H^0(X, K)$.
In particular, for $W = H^0(X, L)$, we have $W \otimes I = H^0(X, L \otimes I)$.
For $s \in W \otimes I$, the vanishing locus $v(s)$ refers to the vanishing locus of $s$ as a section of $L$.
We set $|W| = \P W^*$, the space of one-dimensional subspaces of $W$, and likewise $|W \otimes I| = \P (W \otimes I)^*$.
For a complete linear series, we write $|L|$ for $|W|$.
Note that $v(s) = v(\lambda s)$ for a non-zero scalar $\lambda$, so it causes no ambiguity to talk about $v(s)$ for $s \in |W|$.

\subsection{Non-defective linear series}\label{sec:non-defectivity}
We study a positivity property of linear series that generalizes the property of having a divisorial dual.
\begin{definition}
  \label{definition:Genericallynon-defective} 
  We say that a linear series $(L, W)$ is \emph{non-defective} if,  for a general point $x \in X$ either $W \otimes \mathfrak m_x^2 = 0$, or there exists $s \in W \otimes \mathfrak m_x^2$ such that $v(s)$ has an isolated singularity at $x$.
\end{definition}
Note that for $s \in |W|$, the condition that $v(s)$ have an isolated singularity at $x$ is a Zariski open condition on $|W|$.
Therefore, if there exists an $s \in |W \otimes \mathfrak m_x^2|$ such that $v(s)$ has an isolated singularity at $x$, then a general $s \in |W \otimes \mathfrak m_x^2|$ has the same property.
\begin{remark}
  Let $x$ be a point of $X$.
  Suppose there exists $s \in |W|$ with an isolated singularity at $x$.
  It may be tempting to conclude from this that $(L, W)$ is non-defective.
  This is not necessarily true!
  For example, take $X = \F_3$.
  Denote by $E$ the section of self-intersection $-3$ and $F$ the fiber of the projection $\F_3 \to \P^1$.
  Let $L = \O_X(E + 2F)$ and $W = H^0(X, L)$.
  For $x \in E$, the general member of $|W \otimes \mathfrak m_x^2|$ has an isolated singularity at $x$, but the same is not true for a general $x \in X$.
\end{remark}

\begin{remark}
  Suppose $(L, W)$ is non-defective.
  Let $x \in X$ be general, and let $s \in |W|$ be such that $v(s)$ has an isolated singularity at $x$.
  For all such $s$, it may be the case $v(s)$ has singularities away from $x$, even along a positive dimensional locus.
  For example, let $\pi \from X \to \P^2$  be the blow-up at a point, and $E$ the exceptional divisor.
  The complete linear series associated to $L = \pi^* \O(2) \otimes \O(2E)$ is non-defective, but for every global section of $L$, the singular locus of $v(s)$ contains $E$.
\end{remark}


We now define the conormal variety of a linear series, which plays an important role in our analysis of non-defectivity.
Let $K$ be the kernel of the evaluation map
\[ e \from W \otimes \O_X \to P(L).\]
Let $U \subset X$ be an open subset such that $K|_U$ is locally free and the dual of the inclusion 
\[W^* \otimes \O_U \to K|_U^*\]
is a surjection.
This surjection defines a closed embedding $\P(K|_U) \subset U \times |W|$.
The \emph{conormal variety of $(L,W)$}, denoted by $P_{L,W}$, is the closure of $\P(K|_U)$ in $X \times |W|$.

\begin{proposition}\label{prop:dimension}
  \label{prop:dimP}
  Suppose $(L, W)$ is non-defective.
  If $\dim W \geq r+2$, then $P_{L,W}$ is irreducible of dimension $\dim W - 2$.
  If $\dim W \leq r+1$, then $P_{L,W}$ is empty.
\end{proposition} 

\begin{proof}
  Set $n = \dim |W| = \dim W - 1$.
  Let $k$ be the (generic) rank of $K$, namely the rank of the locally free sheaf $K|_U$.
  Then $k \geq n-r$.
  The statement of the proposition is equivalent to showing that if $k > 0$, then $k = n-r$.

  For brevity, set $P = P_{L,W}$.
  Consider the projection $\sigma \from P \to |W|$, obtained by restricting the second projection $X \times |W| \to |W|$.
  For $s \in |W|$, we view $\sigma^{-1}(s)$ as a subscheme of $X$.
  We then have
  \begin{align*}
    \sigma^{-1}(s) \cap U = \sing(v(s)) \cap U.
  \end{align*}

  Suppose $r>0$.
  Then $P$ is non-empty and irreducible, since it is the closure of a non-empty and irreducible variety.
  Since $(L,W)$ is non-defective, a general point $(x,s) \in P$ is such that $x$ is an isolated point of $\sing(v(s))$.
  Therefore, $\sigma \from P \to |W|$ is generically finite onto its image.
  We conclude that $\dim P \leq \dim |W|$, and hence $k \leq n-r+1$.

  To show that $k = n-r$, it suffices to show that $\sigma \from P \to |W|$ is not surjective.
  We do so using Bertini's theorem.
  Let $B \subset X$ denote the union of the base locus of $|W|$ and the singular locus of $X$.
  Then $B$ is a proper closed subset of $X$.
  Let $P^B \subset P$ be the pre-image of $B$ under the projection $\pi \from P \to X$.
  By the definition of $P$, the map $\pi \from P \to X$ is surjective, and hence $P^B$ is a proper closed subset of $P$.
  Since $P$ is irreducible, we have $\dim P^B < \dim P \leq \dim |W|$, so the projection $P^B \to |W|$ cannot be dominant.
  Let $s \in |W|$ be general, in particular, not in the image of $P^B \to |W|$.
  By Bertini's theorem $v(s)$ is non-singular away from $B$.
  Thus, for any $x \in X$, the point $(x, s) \in X \times |W|$ does not lie in $P$.
  For $x \in B$, this is because $s$ is not in the image of $P^B$, and for $x \not \in B$, this is because $v(s)$ is non-singular at $x$.
  We conclude that $s$ does not lie in the image of $P \to |W|$.
  Hence $P \to |W|$ is not surjective.
\end{proof} 

\begin{proposition}
  \label{prop:dimensionCriterion}
  Let $(L, W)$ be a linear series with $\dim W \geq r+2$, and let $P = P_L$ be its conormal variety.
  The projection $\sigma\from P \to |W|$ is generically finite onto its image if and only if $(L, W)$ is non-defective. 
\end{proposition}

\begin{proof}
  Since $\dim W \geq r+2$, the conormal variety $P = P_{L,W}$ is non-empty.
  Let $(x,s) \in P$ be a general point.
  We may assume that $x \in U$.
  Then $x$ is a singular point of $v(s)$, and it is an isolated singularity of $v(s)$ if and only if $(x,s)$ is an isolated point in the fiber of $\sigma \from P \to |W|$ over $s$.
  The conclusion follows.
\end{proof}

The following observation relates non-defectivity with the non-degeneracy of the dual.
\begin{proposition}\label{prop:non-deg-dual}
  Let $X \subset \P^n$ be a non-degenerate projective variety.  Let $L = \O_X(1)$ and $W \subset H^0(X, L)$ the image of $H^0(\P^n, \O(1))$.
  Then $(L, W)$ is non-defective if and only if the dual variety $X^* \subset {\P^n}^*$ is a hypersurface.
\end{proposition}
\begin{proof}
  Since $X \subset \P^n$ is not contained in a hyperplane, we have $\dim W = n+1 \geq r+1$.
  Since $(L, W)$ is very ample, it separates tangent vectors on $X$, so the evaluation map
  \[ e \from W \otimes \O_X \to P(L)  \]
  is surjective.
  It follows that the rank of the kernel is $n-r$, and hence
  \[ \dim P_{L,W} = (n-r - 1) + r = n-1.\]
  By definition, the dual variety $X^* \subset {\P^n}^* = |W|$ is the image of the conormal variety under the projection $P_{L,W} \to |W|$.
  By \autoref{prop:dimensionCriterion}, $(L, W)$ is non-defective if and only if $\dim X^* = n-1$.
\end{proof}

\begin{proposition}\label{prop:ordinarydoublepoint}
  Let $(L, W)$ be a non-defective linear series on $X$ with $\dim W \geq r+2$.
  Let $x \in X$ be a general point.
  Then there exists $s \in |W|$ such that $v(s)$ has an ordinary double point singularity at $x$.
\end{proposition}
\begin{proof}
  By \autoref{prop:dimensionCriterion}, the projection $\sigma\from P \to |W|$ is generically finite onto its image. 
  Let $(x,s) \in P$ be a general point.
  Since our ground field is of characteristic zero, we may assume that $P$ is smooth at $(x,s)$, that $x \in U \cap X^{\rm sm}$, and $\sigma \from P \to |W|$ is a local immersion at $(x,s)$.
  This implies that $x \in \sing(v(s))$ is isolated, and also that $x$ is a reduced point of the scheme $\sing(v(s))$.
  These two properties show that $v(s)$ possesses an ordinary double point at $x$.
  To see this, choose local coordinates $(x_{1}, ..., x_{n})$ so that the complete local ring ${\widehat{\O}_{X,x}}$ is isomorphic to $k\llbracket x_{1},\dots, x_{r}\rrbracket$.
  After choosing a local trivialization for $L$ around $x$, the section $s$ corresponds to a power series $s(x_1,\dots,x_r)$ contained in $\mathfrak m_x^2 \widehat \O_{X,x}$.
  The germ of $\sing(v(s))$ at $x$ is cut out by the power series $\frac{\partial s}{\partial x_1}, \dots, \frac{\partial s}{\partial x_r}$.
  Since the germ of $\sing(v(s))$ at $x$ is the reduced point $x$, we get that $\frac{\partial s}{\partial x_1}, \dots, \frac{\partial s}{\partial x_r}$ are linearly independent as elements of $\mathfrak m_x / \mathfrak m_x^2$.
  From this, it is easy to check that the tangent cone of $s(x_1, \dots, x_r)$ at $x$ is a non-degenerate quadric cone.
\end{proof}


\begin{proposition}
  \label{prop:genericSeparateTangents}
  If $(L, W)$ is a non-defective linear series with $\dim W \geq r+1$, then $W$ separates tangent vectors at a general point $x \in X$.
  That is, the evaluation map
  \[e_{x}\from W \otimes \O_X \to L/\mathfrak m_x^2 L\]
  is surjective for general $x \in X$.
\end{proposition}
\begin{proof}
  By the definition of $P(L)$, we have a natural isomorphism
  \[ P(L)|_x = L/\mathfrak m_x^2 L,\]
  so it suffices to show that the evaluation map
  \[ e \from W \otimes \O_X \to P(L)\]
  is surjective at $x$.
  Let $k$ be the generic rank of $K$, the kernel of $e$.
  From the proof of \autoref{prop:dimP}, we get
  \[  k = \dim W - r - 1.\]
  Since $(r+1)$ is the generic rank of $P(L)$, we conclude that $e$ is generically surjective.
\end{proof}
\begin{corollary}\label{cor:properlyramified}
  Suppose $(L, W)$ is a non-defective linear series on $X$ with $\dim W \geq r+1$.
  Then there exists a properly ramified projection $(L,V)$ of $(L, W)$.
\end{corollary}
\begin{proof}
  This follows immediately from \autoref{prop:genericSeparateTangents}.
\end{proof}
As a consequence of \autoref{cor:properlyramified}, the projection-ramification rational map $\rho_{X,L, W}$ is defined for a non-defective linear series $(L, W)$ with $\dim W \geq r+1$.

Let $\pi \from \widetilde X \to X$ be the blow-up at a point $x \in X$, and $E \subset \widetilde X$ the exceptional divisor.
A linear series $(L, W)$ on $X$ gives a linear series $(\widetilde L, \widetilde W)$ as follows.
Take $\widetilde L = \pi^* L \otimes \O_{\widetilde X}(-E)$.
Note that $H^0(X, L) = H^0(\widetilde X, \pi^*L)$, so we may think of $W$ as a subspace of $H^0(\widetilde X, \pi^*L)$.
Take $\widetilde W = W \otimes \O_{\widetilde X}(-E)$ with its natural inclusion $\widetilde W \subset H^0(\widetilde X, \widetilde L)$.
\begin{proposition}
  \label{prop:blowuppoint}
  In the setup above, if $(L, W)$ is non-defective, $\dim W \geq r+2$, and $x \in X$ is general, then $(\widetilde L, \widetilde W)$ is also non-defective.
\end{proposition}

\begin{proof}
  Let $y$ be a general point of $\widetilde X$.
  We have the equality
  \[ \widetilde W \otimes \mathfrak m_y^2 = W \otimes \mathfrak m_x \cdot \mathfrak m_y^2. \]
  By \autoref{prop:genericSeparateTangents}, for a general $y \in X$, we have
  \[ \dim (W \otimes \mathfrak m_y^2) = \dim W - (r+1).\]
  Since $x \in X$ is general, we get
  \[ \dim (W \otimes \mathfrak m_x \cdot \mathfrak m_y^2) = \dim W - (r+2).\]
  If $\dim W = r+2$, then we get $\widetilde W \otimes \mathfrak m_y^2 = 0$, so we are done.
  Assume that $\dim W \geq r+3$.
  Then $\dim (W \otimes \mathfrak m_y^2) \geq 2$.
  Since $(L, W)$ is non-defective, a general $s \in W \otimes \mathfrak m_y^2$ is such that $v(s)$ has an isolated singularity at $y$.
  Moreover, since $\dim (W \otimes \mathfrak m_y^2) \geq 2$, for every $x \in X$, there exists $s \in V$ such that $v(s)$ passes through $x$.
  Hence, as $x \in X$ is general, there exists $s \in W \otimes \mathfrak m_y^2$ such that $v(s)$ has an isolated singularity at $y$ and passes through $x$.
  That is, there exists $s \in \widetilde W \otimes \mathfrak m_y^2 $ that has an isolated singularity at $y$.
  We conclude that $(\widetilde L, \widetilde W)$ is non-defective.
\end{proof}



\subsection{Maximal variation for non-defective pairs}

In this section, we prove part~\eqref{item:dual} of \autoref{theorem:Main}.
In fact, we prove a more general result (\autoref{theorem:MainMain}).

As before, $X$ is a proper, normal variety of dimension $r$ over an algebraically closed field of characteristic zero.
\begin{theorem}
  \label{theorem:MainMain}
  Let $(L, W)$ be a non-defective linear series on $X$ with $\dim W \geq r+2$.
  Then the projection-ramification map $\rho_{X,L,W}$ is generically finite onto its image.
\end{theorem}

For the proof, we need two lemmas, which are essentially local computations.
Throughout, $X$, $L$, and $W$ are as in the statement of \autoref{theorem:MainMain}.

\begin{lemma}\label{lemma:tangentconeRam}
  Let $x \in X$ be a general point and $V \subset W \otimes \mathfrak m_x$ a general $(r+1)$-dimensional subspace.
  Then $V$ is properly ramified, and the ramification divisor $R(V)$ has an ordinary double point singularity at $x$.
\end{lemma}

\begin{proof}
  Using \autoref{prop:ordinarydoublepoint} and \autoref{prop:genericSeparateTangents}, we get a basis $(s_{1}, ..., s_{n}, t)$ of $V$ satisfying the following two conditions:
  \begin{enumerate}
      \item $s_{1}, \dots, s_{n}$ generate $L \otimes ({\mathfrak m}_{x}/{\mathfrak m}^{2}_{x})$, and
      \item $v(t)$ has an ordinary double point singularity at $x$.
    \end{enumerate}  

    Let $\widehat{\O}_{X,x}$ denote the completion of the local ring at $x \in X$ along its maximal ideal.  Upon trivializing $L$, we may regard $s_{i}$ and $t$ as elements of $\widehat{\O}_{X,x}$, and can also assume  $\widehat{\O}_{X,x} = k\llbracket s_{1}, \dots s_{n}\rrbracket$.
    In the bases $(s_1, \dots, s_n, t)$ for $V$ and $(1, s_1, \dots, s_n)$ for $P(L)$, the evaluation map 
\begin{align*}
  e\from V \otimes \widehat{\O}_{X,x} \to P(L) \otimes \widehat{\O}_{X,x}
\end{align*}
has the matrix
\begin{align}\label{matrix}
\begin{pmatrix}
  s_{1} & s_{2} & \dots & t \\
  1 & 0 & \dots & \partial_{1}t \\
  0 & 1 & \dots & \partial_{2}t \\
  \vdots & \vdots & \vdots & \vdots \\
  0 & 0 & \dots & \partial_{n}t
\end{pmatrix},
\end{align}
where $\partial_{i}$ denotes $\frac{\partial}{\partial s_{i}}$.
The determinant of the matrix \eqref{matrix}
\begin{align*}
  t - \sum_{i}s_{i}\partial_{i}t
\end{align*}
is an analytic local equation for the ramification divisor $R(V)$ near $x$.
Evidently, $R(V)$ shares the same tangent cone as $v(t)$ at $x$.
The proposition follows.
\end{proof}

\begin{lemma}\label{lemma:basepointfree}
  Let $x \in X$ be a general point and $V \subset W$ an $(r+1)$-dimensional subspace with a basis $(u, a_1,\dots, a_{r-1},b)$ where
  \begin{enumerate}
    \item $u$ does not vanish at $x$,
    \item $a_1, \dots , a_{r-1}$ vanish at $x$ and reduce to linearly independent elements of $L \otimes ({\mathfrak m}_{x}/{\mathfrak m}^{2}_{x})$, and
    \item $v(b)$ has an ordinary double point at $x$. 
  \end{enumerate}
  Then $R(V)$ contains $x$ and is smooth at $x$.
\end{lemma}

\begin{proof}
  That $R(V)$ contains $x$ is clear since $V \otimes \mathfrak{m}^{2}_{x} \neq 0$.

  For smoothness, we again work in the completion $\widehat{\O}_{X,x}$.
  After trivializing $L$, we assume $u, a_{1}, ..., b$ are elements of $\widehat{\O}_{X,x}$.
  We choose an element $z \in \widehat{\O}_{X,x}$ such that $(a_1, \dots, a_{r-1}, z)$ forms a system of coordinates, that is $\widehat{\O}_{X,x} \cong k\llbracket a_{1}, \dots , a_{r-1}, z \rrbracket$.
  With respect to the given basis of $V$ and the basis $1, a_1, \dots, a_{r-1}, z$ for $P(L)$, the evaluation map
  \begin{align*}
  e\from V \otimes \widehat{\O}_{X,x} \to P(L) \otimes \widehat{\O}_{X,x}
  \end{align*}
  has the matrix
\begin{align}\label{matrix2}
\begin{pmatrix}
  u & a_{1} & a_{2} & \dots & b \\
  \partial_{1}u & 1 & 0 & \dots & \partial_{1}b \\
  \partial_{2}u & 0 & 1 & \dots & \partial_{2}b \\
  \vdots & \vdots & \vdots & \vdots \\
  \partial_{z}u  & 0 & 0 & \dots & \partial_{z}b
\end{pmatrix}
\end{align}
The determinant of the matrix \eqref{matrix2} is the analytic local equation for $R(V)$.
It is given by
\begin{align*}
   \bar{u} \cdot \partial_{z}b \pm \partial_{z}u \cdot \bar{b},
 \end{align*} 
 where, for $r \in \widehat{\O}_{X,x}$ we set
 \[\bar{r} = r - a_{1}\partial_{1}r - a_{2}\partial_{2}r - \dots - z \partial_{z} r.\]
 Since $b \in {\mathfrak m}^{2}_{x}$, we get that $\bar{b} \in {\mathfrak m}^{2}_{x}$, and so $\partial_{z}b \in {\mathfrak m}_{x}$.
 Furthermore, since the tangent cone of $b$ is a non-degenerate quadric, we also get that $\partial_z b \not \in \mathfrak m_x^2$.
 Since $\overline{u}$ is a unit, we see that the tangent cone of $R(V)$ at $x$ is the hyperplane cut out by $\partial_z b \in \mathfrak m_x/\mathfrak m_x^2$.
 So $R(V)$ is smooth at $x$.
\end{proof}

We now have all the tools for the proof of \autoref{theorem:MainMain}. 
\begin{proof}[Proof of \autoref{theorem:MainMain}]
  We induct on $\dim W$.
  The base case $\dim W = r+1$ is clear.

  We now do the induction step.
  Suppose $\dim W \geq r+2$.
  Choose a general point $x \in X$ such that the induced linear series $(\widetilde L, \widetilde W)$ on $\widetilde X = \Bl_x X$ is non-defective as in \autoref{prop:blowuppoint}.
  Choose a general $(r+1)$-dimensional subspace $V \subset W \otimes \mathfrak m_x = \widetilde W$ that satisfies the hypotheses of \autoref{lemma:tangentconeRam}.
  By the induction hypothesis, $V$ considered as a projection of $(\widetilde L, \widetilde W)$ is an isolated point in the projection-ramification map for $\widetilde X$.
  We now show that it is also an isolated point in the projection-ramification map for $X$.

  Let $(C, 0)$ be a pointed smooth curve and $V \subset W \otimes \O_C$ a sub-bundle of rank $(r+1)$ such that
  \begin{enumerate}
  \item $V_{0} = V$, and 
  \item $V_{c} \neq W_{0}$ for $c \in C \setminus \{0\}$.
  \end{enumerate}
  We must show that $R(V_c) \neq R(V)$ for a general $c \in C$.

  Suppose $V_c \subset W \otimes \mathfrak m_x = \widetilde W$ for all $c \in C$.
  Denote by $\widetilde R(V_c)$ the ramification divisor of $V_c$ considered as a projection of $\widetilde X$.
  Since $V = V_0$ is an isolated point in the projection-ramification map for $\widetilde X$, we know that $\widetilde R(V_c) \neq \widetilde R(V_0)$ for a general $c \in C$.
  Clearly, $R(V_c)$ and $\widetilde R(V_c)$ agree away from the exceptional divisor, and hence we conclude that $R(V_c) \neq R(V_0)$ for a general $c \in C$.

  On the other hand, suppose $V_c \not \subset W \otimes \mathfrak m_x = \widetilde W$ for a general $c \in C$.
  Consider the evaluation maps
  \[ e_c \from V_c \to L / \mathfrak m_x^2 L \]
  between an $(r+1)$-dimensional source and $(r+1)$-dimensional target.
  Since $V = V_0$ satisfies the hypotheses of \autoref{lemma:tangentconeRam}, $\rk e_0 = r$.
  Therefore, by semi-continuity, $\rk e_c \geq r$ for all $c \in C$.
  If $\rk e_c = (r+1)$ for a general $c \in C$, then $x \not \in R(V_c)$, and hence $R(V_c) \neq R(V)$.
  Otherwise, by shrinking $C$ if necessary, assume $\rk e_c = r$ for all $c \in C$.
  In other words, $\dim (V_c \otimes \mathfrak m_x^2) = 1$ for all $c \in C$.
  Let $b_c \in V_C \otimes \mathfrak m_x^2$ be a non-zero element.
  Since $v(b_0)$ has an ordinary double-point singularity at $x$, so does $v(b_c)$.
  Also, since $\rk (e_c) = r$ and $V_c \not \in W \otimes \mathfrak m_x$ for a general $c$, there exists $u_c \in V_c$ not vanishing at $x$, and a set of $(r-1)$ other elements that vanish at $x$ but reduce to linearly independent elements modulo $\mathfrak m_x^2$.
  That is, $V_c$ satisfies the hypotheses of \autoref{lemma:basepointfree} for a general $c \in C$.
  But \autoref{lemma:basepointfree} implies that $R(V_c)$ is smooth at $x$.
  Since $R(V_0)$ is singular at $x$, we conclude that $R(V_0) \neq R(V_c)$.
  The induction step is now complete.
\end{proof}

We immediately get part~\eqref{item:dual} of \autoref{theorem:Main}.
\begin{corollary}
  \label{cor:maintheorem} Let $X \subset \P^{n}$ be a non-degenerate projective variety such that the dual variety $X^{*} \subset \P^{n*}$ is a hypersurface. Then $\rho_{X}$ is generically finite onto its image.
\end{corollary}
\begin{proof}
  By \autoref{prop:non-deg-dual} the linear series on $X$ that gives the embedding $X \subset \P^n$ is non-defective.
  Now apply \autoref{theorem:MainMain}.
\end{proof}


\section{Projection-ramification for varieties of minimal degree}\label{sec:minimaldegree}
In this section, we prove \autoref{theorem:minimaldegree}, which characterizes varieties of minimal degree in terms of the projection-ramification map.
We then prove \autoref{Thm:Counterexamples} by constructing examples of rational scrolls where maximal variation fails.
Finally, we obtain an alternate description of the projection-ramification map for scrolls, which is used in \autoref{sec:proof_of_theorem:rationalnormalscrolls}.

The following is an easy application of the Kodaira vanishing theorem.
\begin{proposition}
  Let $X$ be a smooth projective $r$-dimensional variety with $r > 0$ with a map $X \to \P^n$ that does not factor through 
  Suppose the pull-back $H$ of $\O(1)$ is ample.
  Then
  \[ h^0(X, K_X+rH) \geq n-r.\]
\end{proposition}
\begin{proof}
  We induct on $r$.
  The case $r = 1$ follows directly from the fact that a degree $d$ line bundle on a curve has at most $(d+1)$ linearly independent sections.

  For the induction step, assume $r > 1$, and let $D \subset X$ be the preimage of a general hyperplane.
  By Bertini's theorem, $D$ is smooth and connected.
  Consider the exact sequence
  \[
    0 \to \O_X(K_X (r-1)H) \to \O_X(K_X + rH) \to \O_D(K_D + (r-1)H) \to 0.
  \]
  Kodaira's vanishing theorem gives $h^1(\O_X(K_X +(r-1)H)) = 0$.
  Hence, by the long exact sequence of cohomology, we get
  \begin{equation}\label{eqn:khkd}
    h^0(X, K_X + rH) \geq h^0(D, K_D + (r-1)H)
  \end{equation}
  By the induction hypotheses, we have
  \[h^0(D, K_D+(r-1)H) \geq n-1-r.\]
  By combining the last inequality with \eqref{eqn:khkd}, we get the induction step.
\end{proof}

We now have the tools to prove \autoref{theorem:minimaldegree}, which we restate.
\begin{theorem}[\autoref{theorem:minimaldegree}]
  Let $X \subset \P^{n}$ be a smooth, non-degenerate projective variety of dimension $r$ over a field of characteristic zero.
  We have the inequality
  \[ \dim \Gr(n-r, n+1) \leq \dim |K_X + (r+1)H|,\]
  where equality holds if and only if $X$ is a variety of minimal degree, namely $\deg X = n-r+1$.
\end{theorem}
\begin{proof}
  Set $d = \deg X$.
  We induct on $r$.
  
  For $r = 1$, we have
  \[ \dim \Gr(n-r, n+1) = 2(n-2),\]
  and
  \[dim |K_X + (r+1)H| = h^0(K_X + 2H) = 2g_X-2+2d. \]
  Since $X$ is non-degenerate, we have $n \leq h^0(X, H) - 1$, with equality if and only if $X$ is embedded by the complete linear series.
  Furthermore, we always have $h^0(X, H) \leq d+1$, with equality if and only if $X$ is rational.
  The conclusion follows.

  For $r > 1$, let $D \subset X$ be a general hyperplane slice.
  Consider the long exact sequence on cohomology induced by
  \[ 0 \to \O_X(K_X+ rH) \to \O_X(K_X + (r+1)H) \to \O_D(K_D + rH) \to 0.\]
  By the Kodaira vanishing theorem and \autoref{lemma:KYmH}, we get
  \[
    h^0(X, K_X+(r+1)H) - h^0(X, K_X+rH) = h^0(D, K_D + rH) \geq n-r.
  \]
  
  
 We first show: if $\dim \Gr(n-r, n+1) \geq \dim |K_{X} + (r+1)H|$, then $X$ is
 a variety of minimal degree.  Then we argue: in the case of minimal
 varieties, this inequality is actually an equality.

  We proceed by intersecting with a hyperplane: let $X' = X \cap H$ be a general
  hyperplane section of $X$. By combining the Kodaira vanishing theorem, adjunction, and \autoref{lemma:KYmH} we get:
  \begin{align*}
    h^{0}(\O_{X}(K_{X}+ (r+1)H)) - h^{0}(\O_{X'}(K_{X'}+ rH)) \geq n-r.
  \end{align*}
Therefore, the inequality $$\dim \Gr(n-r, n+1) \geq \dim |K_{X} + (r+1)H|$$ implies $$\dim \Gr(n-r, n) \geq \dim |K_{X'} + rH|$$ and hence by the inductive hypothesis, $X'$ is a variety of minimal degree.  We deduce $X$ is itself a variety of minimal degree. 


Thus we are reduced to considering the case $r=1$: we leave it to the reader to translate this case into the well-known fact that a non-degenerate degree $n$ smooth curve in $\P^{n}$ is a rational normal curve.

To complete the proof we observe that if $X$ is a variety of minimal degree, and if $\dim \Gr(n-r,n+1) > \dim
|K_{X}+ (r+1)H|$, then by arguing in exactly the same way as above, we would
conclude the analogous strict inequality for its iterated hyperplane slices.
Again we reduce to the case of $X$ a rational normal curve ($r=1$), where such an inequality is clearly false. This completes the proof of
\autoref{theorem:minimaldegree}.
\end{proof}



The goal of this section is to prove \autoref{Thm:Counterexamples} and \autoref{theorem:minimaldegree}.


\section{Proof of \autoref{Thm:Counterexamples}} % (fold)
\label{sec:proof_of_second_result}

Our next objective is to prove \autoref{Thm:Counterexamples} by exhibiting some
examples.  Before doing so, we pose the general problem of maximal variation for rational normal scrolls in explicit affine coordinates. 

\subsection{The generalized Wronski map for scrolls in affine coordinates} % (fold)
 \label{sub:the_generalized_wronski_map}
 
 Fix variables $x_{1}, \dots, x_{r}, t$.
\begin{definition}
  \label{definition:V}
  Let $\underline{d} = (d_{1}, \dots, d_{r})$ denote an $r$-tuple of degrees. We define $V(\underline{d})$ to be the vector space of forms $\sum_{i=1}^{r}p_{i}(t)x_{i}$, where $\deg p_{i} \leq d_{i}$.
\end{definition}

\begin{remark}
  $V(\underline{d})$ is simply the space of global sections of the line bundle $\O_{\P E}(1)$ on the scroll $\P E$ over $\P^{1}$, where $E = \O(d_{1}) \oplus \dots \oplus \O(d_{r})$.
\end{remark}

Next, if $v_{1} \wedge \dots \wedge v_{r+1} \in \bigwedge^{r+1}V(\underline{d})$ is any pure tensor, we set 
\begin{align}
  Wr(v_{1} \wedge \dots \wedge v_{r+1}) := \det \begin{pmatrix}
    -- & v_{1} & -- & v_{1}'\\
    -- & v_{2} & -- & v_{2}'\\
    \vdots& \vdots & \vdots & \vdots\\
    -- & v_{r+1} & -- & v_{r+1}'
  \end{pmatrix} \in V(\underline{e})
\end{align}
where $\underline{e}= (e_{1}, \dots, e_{r})$ is given by $e_{i} = d_{i}-2 + \sum_{j=1}^{r}d_{j}$.

\begin{definition}
  \label{definition:Wronskian}
  The induced map 
  \begin{align}
    Wr_{\underline{d}}\from \Gr(r+1, V(\underline{d})) \dashrightarrow \P V(\underline{e})
  \end{align}
  is called the {\sl Wronskian} map.
\end{definition}

\begin{remark}
  $Wr_{\underline{d}}$ is the projection-ramification map for the scroll $X = \P E$ in coordinates.
\end{remark}

The dimensions of source and target of the Wronskian map are equal, hence we may pose the general question: 
\begin{problem}
\label{problem:DominantWronskian}
  For which degree vectors $\underline{d} = (d_{1}, ..., d_{r})$ is the Wronskian map $Wr_{\underline{d}}$ dominant?
\end{problem}

In the next section, we show \autoref{problem:DominantWronskian} has genuine content  by demonstrating that $Wr_{\underline{d}}$ fails to be dominant for degree vectors of the form $(1,1,1, \dots, k+1)$, $r \geq 4$.
 % subsection the_generalized_wronski_map (end) 




\subsection{Proof of \autoref{Thm:Counterexamples}}

Let $E = \O(1)^{r-1} \oplus \O(k+1)$ be the vector bundle over
$\P^{1}$, and set $X = \P E$. We will prove:
\begin{theorem}
    The projection-ramification map for the embedding of $X$ given by $\O_{E}(1)$ is not
    dominant once $k(r-3) > 1$.
    \label{NonDominance}
\end{theorem}

\begin{remark}
The basic phenomenon underlying this example is: a general point in the source Grassmannian
has trivial $Aut(X)$-stabilizer, yet every point of $|R|$ has positive dimensional stabilizer. 
\end{remark}

\begin{remark}
If $k = 1$ and $r \geq 5$, then $X$ is a balanced
scroll. Therefore, the non-dominance of projection-ramification is not directly
connected to the eccentricity of the splitting type of a scroll. Rather,
among balanced scrolls, non-dominance of $\rho_{X}$ happens only in ``low''
degree -- see \autoref{thm:prp1}.
\end{remark}


As an immediate corollary, we  get a result concerning  Grassmannians in their Plucker
embeddings. Recall that an $r$ dimensional variety $X \subset \P^{n}$ is
{\sl compressible} if there exists a $(n-r-1)$-dimensional linear space
$\Lambda \subset \P^{N}$ with the property that the projection $p_{\Lambda}: X
\dashrightarrow \P^{r}$ is not dominant. 

\begin{proof}[Proof of \autoref{NonDominance}] We will show: the general element in the ramification divisor class $|R|$ has a positive dimensional stabilizer under the action of $Aut(X)$.  We leave it to the reader to check that the Grassmannian $\Gr$ does not have generic stabilizer under $\Aut(X)$.  The theorem then follows by the $Aut(X)$-equivariance of $\rho\from \G \dashrightarrow |R|$.


In terms of the affine coordinates $(x_{1}, \dots, x_{r}, t)$ introduced in the previous section, we find ourselves in the situation corresponding to the degree vector $\underline{d} = (1, ..., 1, k+1)$.  The degree vector corresponding to ramification divisors is then $\underline{e} = (r+k+1, r+k+1, \dots, r+2k+1)$.  

In these affine coordinates, the substitutions
\begin{align}\label{substitutions}
	x_{1} \mapsto & x_{1} + p_{1}(t)x_{r}\\
	x_{2} \mapsto & x_{2} + p_{2}(t)x_{r}\\
	\vdots&\\
	x_{n} \mapsto & x_{r}
\end{align}
produce distinct automorphisms in $Aut(X)$ per choice of the $p_{i}$, where each $p_{i}(t)$ has degree $\leq k$.  

If $\sum^{r}_{j=1}a_{j}x_{j}$ represents a general element of $V(\underline{e})$, then the  substitutions \eqref{substitutions} have the effect of replacing the coefficient $a_{r}(t)$  with $a_{r} + \sum_{j=1}^{r-1}a_{j}p_{j}$, and preserving all other coefficients $a_{i}$. 

Now if $(r-3)k > 1$, then the dimension of the vector space of choices for the polynomials $p_{i}$ exceeds the dimension of degree $r+2k+1$ polynomials $a_{r}$. Hence there exists a particular choice of $p_{i}$'s (not all zero) such that the above automorphism fixes the equation $\sum^{r}_{j=1}a_{j}x_{j}$. Scaling these  $p_{i}$'s by constants produces the positive dimensional stabilizer mentioned in the theorem.
	
\end{proof}

We obtain an immediate corollary:


\begin{corollary}
    The Grassmannian $\Gr(m,n)$ is compressible if $5 \leq m \leq n-m$. \todo{Is this obvious? Can we find non-finite projections easily by hand?}
\end{corollary}









% subsection original_proof (end)










% section proof_of_second_result (end)


\section{Proof of \autoref{theorem:minimaldegree}} % (fold)
\label{sec:proof_of_theorem:minimaldegree} 



% section proof_of_theorem:minimaldegree (end)

\section{Proof of \autoref{theorem:rationalnormalscrolls}}% (fold)
    \label{sec:proof_of_theorem:rationalnormalscrolls}


In this section, we extend the projection ramification map to vector bundles on nodal curves using limit linear series.
We then use degeneration to a nodal curve to prove generic maximal variation for vector bundles on smooth curves.

\subsection{Limit linear series}\label{sec:lls}
A linear series on a curve of rank $r$, degree $d$, and dimension $k$ consists of a vector bundle $E$ on the curve of rank $r$ and degree $d$, and a $k$-dimensional subspace of the vector space $H^0(E)$.
A limit linear series is an extension of this idea to singular curves, done in a manner suitable for degeneration techniques.

Let $B$ be a DVR with special point $0$ and general point $\eta$.
Let $\pi \from X \to B$ be a family of connected projective curves of genus $g$, smooth over $\eta$, and at worst nodal over $0$, with non-singular total space $X$.
Assume the special fiber $X_0 = C$ is the nodal union of two curves $C_1$ and $C_2$ meeting at a unique point $p$.
Then $\pi \from X \to B$ is a particularly simple example of an almost local smoothing family \cite[\S~2.1--2.2]{oss:14}.

We recall the notion of a limit linear series from \cite{oss:14}, where it is called linked linear series.
In \cite{oss:14}, Osserman defines two types of linked linear series.
In our setting, where $C$ has only two components, both notions coincide \cite[Remark~3.4.15]{oss:14}.
We model our definition on the definition of the type II series.

Fix the following data:
\begin{enumerate}
\item positive integers $r$, $d$, and $k$;
\item integers $d_1$, $d_2$, and $b$ satisfying $d_1 + d_2 - rb = d$;
\item maps $\theta_v \from \O_X \to \O_X(C_v)$ for $v = 1, 2$ vanishing precisely on $C_v$;
\item integers $w_1$, $w_2$ satisfying $w_v \equiv d_v \pmod r$ and $w_1 + w_2 = d$.
\end{enumerate}
The integers $r$, $d$, and $k$ will denote the rank, the degree, and the dimension of the linear series.
The tuple $w = (w_1, w_2)$ will encode the multi-degree of the vector bundle in the limit linear series, and the integers $d_1$ and $d_2$ will encode its extremal twists.
The maps $\theta_v$ are unique up to an element of $\O_B^*$.
The choice of $w_v$, and $\theta_v$ is entirely auxilliary; different choices give isomorphisms between the corresponding moduli stacks of limit linear series.
The choice of $d_v$ and $b$ is also largely auxilliary; increasing them leads to open inclusions between the corresponding moduli stacks of limit linear series.

Let $S$ be a $B$-scheme, and let $\mathcal E$ be a vector bundle on $X_S$ of rank $r$ and degree $d$.
For every $n \in \Z$, define the vector bundle $\mathcal E_n$ by
\[ \mathcal E_n =
  \begin{cases}
    \mathcal E \otimes \O_X(C_1)^{n} & \text{if $n \geq 0$},\\
    \mathcal E \otimes \O_X(C_2)^{-n}  & \text{if $n < 0$}.
  \end{cases}
\]
Define maps
\[ \theta_n \from \mathcal E_{m} \to \mathcal E_{m+n}\]
by
\[
  \theta_n = 
  \begin{cases}
    \theta_1^n & \text{if $n \geq 0$,} \\
    \theta_2^{-n} & \text{if $n < 0$.}
  \end{cases}.
\]
We say that $\mathcal E$ has multi-degree $w$ if for every $s \in S$ mapping to $0 \in B$, the degree of $\mathcal E|_s$ on $C_v$ is $w_v$ for $v = 1, 2$.
Note that, if $\mathcal E$ has multi-degree $(w_1, w_2)$, then $\mathcal E_n$ has multi-degree $(w_1-rn, w_2+rn)$.

Let $n_1 \in \Z$  be such that
\[ (w_1 - n_1r, w_2+n_1r) = (d_1, d_2-rb),\]
and $n_2 \in \Z$ such that
\[ (w_1-n_2r, w_2+n_2r) = (d_1-rb, d_2).\]
Observe that $n_2 - n_1 = b$.
\begin{definition}[{Special case of \cite[Definition~3.3.2]{oss:14}}]
  \label{def:lls}
  Let $S$ be a $B$-scheme.
  A \emph{limit linear series} on $X_S$ consists of $(\mathcal E, V_n \mid n \in \Z)$, where $\mathcal E$ is a vector bundle of rank $r$, degree $d$, and multi-degree $w$ on $X_S$, and $V_n$ is a sub-bundle of $\pi_* \mathcal E_n$ of rank $k$ satisfying the following conditions.
  \begin{enumerate}
  \item (Vanishing)
    \label{lls:vanishing}
    For every $z \in S$ over $0 \in B$ and $v = 1,2$, we have
    \[ H^0(C_v, \mathcal E_{n_v}|_{C_v}(-(b+1)p)) = 0.\]
  \item (Compatibility)
    \label{lls:compatibility}
    For every $m, n \in \Z$, the map
    \[\pi_*\theta_n \from \pi_*\mathcal E_m \to \pi_*\mathcal E_{m+n}\] maps
    $V_m$ to $V_{m+n}$.
  \end{enumerate}
\end{definition}
The notion of a sub-bundle of a push-forward $\pi_* \mathcal E_n$ is as in \cite[Definition~B.2.1]{oss:14}, namely $V_n$ is a vector bundle with a map $V_n \to \pi_* \mathcal E_n$ which remains injective after arbitrary base-change.
\begin{remark}
  In our case, the various twists of $\mathcal E$ are indexed by integers $n$.
  In general, the twists are indexed by a graph $G_{II}$ that depends on the dual graph of $X_0$.
\end{remark}

Denote by ${\mathcal G}^k_{r,d,d_*, w_*}(X/B)$ the category fibered over the category of $B$-schemes whose objects over $S$ are the limit linear series on $S$ of rank $r$, degree $d$, and multi-degree $w$, and whose morphisms are isomorphisms over $S$, defined in the obvious way.

\begin{definition}\label{def:simple_lls}
Let $S = \spec K$, where $K$ is a field, and let $\lambda = (\mathcal E, V_n \mid n \in \Z)$ be a limit linear series on $S$.
We say  $\lambda$ is \emph{simple} if there exist integers $w_1, \dots, w_k$, not necessarily distinct, and elements $v_i \in V_{w_i}$ such that for every integer $w$, the images of $v_1, \dots, v_k$ in $V_w$ form a basis of $V_w$.
Here the maps $V_{w_i} \to V_w$ are as in \autoref{def:lls}~\eqref{lls:compatibility} .
\end{definition}
\begin{remark}\label{rem:simple_lls}
  By \cite[Lemma~3.4.14]{oss:14}, it suffices to check the basis condition for $w = n_1$ and $w = n_2$.
\end{remark}

Let $M_{r, d, w}(X/B)$ be the category fibered over the category of $B$-schemes whose objects over a $B$-scheme $S$ are vector bundles $\mathcal E$ of rank $r$, degree $d$, and multi-degree $w$ on $X_S$, and whose morphisms are isomorphisms over $S$.
Let $M_{r,d,d_*,w_*}(X/B) \subset M_{r,d,w}(X/B)$ be the full-subcategory  parametrizing bundles satisfying the vanishing condition in \autoref{def:lls}~\eqref{lls:vanishing}.
Then $M_{r,d,w}(X/B)$ is an Artin stack over $B$, locally of finite type.
By the semi-continuity of cohomology, $M_{r, d, d_*, w_*}(X/B) \subset M_{r,d,w}(X/B)$ is an open substack.

\begin{theorem}[{\cite[Theorem~3.4.7]{oss:14}}]
  \label{thm:lls}
  Retain the notation above.
  Then $\mathcal G^k_{r,d,d_*,w_*}(X/B)$ is an Artin stack over $B$.
  The natural forgetful map
  \[ \beta \from \mathcal G^k_{r,d,d_*,w_*}(X/B) \to M_{r,w,d_*}(X/B)\]
  is representable by schemes, which are projective locally on the target.
  The locus of simple limit linear series is an open substack of $\mathcal G^k_{r,d,d_*,w_*}(X/B)$; it has universal relative dimension at least $k(d-k-r(g-1))$ over $M_{r,w,d_*}(X/B)$.
\end{theorem}
The last statement implies: if $\lambda$ is a simple limit linear series such that the fiber of $\beta$ through $\lambda$ has dimension at most $k(d-k-r(g-1))$ at $\lambda$, then $\beta$ is open at $\lambda$ of relative dimension exactly $k(d-k-r(g-1))$.


Although \autoref{def:lls} requires specifying infinitely many vector bundles $V_n$, for $n \in \Z$, specifying finitely many determines the rest.
Set $I = [n_1, n_2] \cap \Z$.
Define an \emph{$I$-linear series} to be the data of $(\mathcal E, V_n \mid n \in I)$ satisfying the conditions \eqref{lls:vanishing} and \eqref{lls:compatibility} in \autoref{def:lls} whenever the subscripts lie in $I$.
\begin{proposition}\label{prop:lls_restricted}
  The natural forgetful map from the groupoid of limit linear series to the groupoid of $I$-linear series is an equivalence.
\end{proposition}

It is often enough to specify only the two extremal bundles, for $n = n_1$ and $n = n_2$, provided they satisfy certain compatibility conditions. 
This approach gives the original incarnation of the notion of limit linear series due to Eisenbud and Harris \cite{eis.har:86} for the rank 1 case and Teixidor i Bigas in the higher rank case \cite{tei-i-big:91}.

Let $\mathcal E$ be a vector bundle on $C$ of multi-degree $w$ satisfying the vanishing condition in \autoref{def:lls}.
\begin{definition}[{Adapted from \cite[Definition~4.1.2]{oss:14}}]
  \label{def:eht}
  Let $W_v \subset H^0(C, \mathcal E_{n_v}|_{C_v})$ be a $k$-dimensional subspace for $v = 1, 2$.
  We say  $(\mathcal E, W_1, W_2)$ is an \emph{EHT limit linear series} if the following conditions are satisfied.
  \begin{enumerate}
  \item
    \label{ieq:eht}
    If $a^v_1 \leq \cdots \leq a^v_k$ is the vanishing sequence for $(\mathcal E_{n_v}|_{C_v}, W_v)$ at $p$ for $v = 1, 2$, then for every $i = 1, \dots, k$ we have
    \[ a^1_i + a^2_{k+1-i} \geq b.\]
  \item\label{gluing:eht}
    There exist bases $s^v_1, \dots, s^v_k$ for $W_v$ for $v = 1, 2$, such that $s^v_i$ has order of vanishing $a^v_i$ at $p$, and if we have $a^1_i + a^2_{k+1-i} = b$ for some $i$, then
    \[ \widetilde \phi (s^1_i) = s^2_{k+1-i},\]
    where $\widetilde \phi \from \mathcal E_{n_1}(-a^1_{i}p)|_p \to \mathcal E_{n_2}(-a^2_{k+1-i}p) |_p$ is the isomorphism obtained by taking the appropriate twist of the identity map.
  \end{enumerate}
  We say  $(\mathcal E, W_1, W_2)$ is \emph{refined} if all the inequalities in \eqref{ieq:eht} are equalities.
\end{definition}
Due to the vanishing condition, the restriction map
\[ H^0(C, \mathcal E_{n_v}) \to H^0(C_v, \mathcal E_{n_v}|_{C_v})\]
is an isomorphism.
Via this isomorphism, we sometimes treat $W_v$ as a subspace of $H^0(C_v, \mathcal E_{n_v}|_{C_v})$.

It is possible to define a stack of EHT limit linear series so that the locus of refined EHT limit linear series forms an open substack \cite[\S~4]{oss:14}.

Let $\lambda = (\mathcal E, V_n \mid n \in \Z)$ be a limit linear series on $C$ in the sense of \autoref{def:lls}.
Set $W_1 = V_{n_1}$ and $W_2 = V_{n_2}$.
\begin{proposition}\label{prop:llseht}
  In the notation above, $(\mathcal E, W_1, W_2)$ is an EHT limit linear series on $C$.
  Conversely, given an EHT limit linear series $\mu = (\mathcal E, W_1, W_2)$, there exists a limit linear series $\lambda = (\mathcal E, V_n)$ on $C$ such that $W_1 = V_{n_1}$ and $W_2 = V_{n_2}$.
  Furthermore, if $\mu$ is refined, then $\lambda$ is unique and simple.
\end{proposition}
\begin{proof}
  This is a point-wise version of the stack theoretic statement \cite[Theorem~4.3.4]{oss:14}, plus the equivalence of type I and type II series in the two component case (\cite[Remark~3.4.15]{oss:14}).

  The assertion about refined series follows from the proof of \cite[Theorem~4.3.4]{oss:14}, but it is not explicitly stated there.
  So we offer a proof.
  
  Let $\mu$ be a refined EHT limit linear series.
  We now construct $V_n \subset H^0(\mathcal E_n)$.
  By \autoref{prop:lls_restricted}, it suffices to take $n \in [n_1, n_2]$.

  By composing the restriction $\mathcal E_n \to \mathcal E_n |_{C_v}$ and the inclusion $\mathcal E_n |_{C_v} \to \mathcal E_{n_v}|_{C_v}$, we get a map
  \[ \iota \from H^0(C, \mathcal E_n) \to H^0(C_1, \mathcal E_{n_1}|_{C_1}) \oplus H^0(C_2, \mathcal E_{n_2}|_{C_2}).\]
  The vanishing condition in \autoref{def:lls} shows $\iota$ is an injection.
  The compatibility condition in \autoref{def:lls} implies  $\iota(V_n) \subset W_1 \oplus W_2$.
  Therefore, the subspace $V_n \subset H^0(C, \mathcal E_n)$ lies in the kernel of the map
  \begin{equation}\label{eqn:iotabar}
    \overline{\iota} \from H^0(C, \mathcal E_n) \to H^0(C_1, \mathcal E_{n_1}|_{C_1})/W_1 \oplus H^0(C_2, \mathcal E_{n_2}|_{C_2}) / W_2.
  \end{equation}
  
  Next we show:
  \begin{equation}\label{eqn:keriotabar}
    \dim \ker \overline \iota = k.
  \end{equation}
  Suppose $s \in \ker\overline\iota$.
  Then $\iota(s)$ is a linear combination of $(s^1_1,0), \dots, (s^1_k,0), (0,s^2_1), \dots, (0,s^2_k)$.
  Writing $\iota(s) = (s_1, s_2)$, we have
  \begin{equation}\label{eqn:vanishing}
    \ord_p (s_1) \geq n-n_1 \text{ and } \ord_p(s_2) \geq n_2-n.
  \end{equation}
  Recall that $a^v_1 \leq \cdots \leq a^v_k$ is the vanishing sequence of $W_v$ for $v = 1, 2$.
  Let $i$ be the smallest such that
  \[ a^1_{i} \geq n - n_1,\]
  and let $i + c$ be the smallest such that
  \[ a^1_{i + c} > n - n_1.\]
  Since $\mu$ is a refined series and $n_2 - n_1 = b$, we get $j = k+1-i$ is the largest such that
  \[ a^2_j \leq n_2 - n,\]
  and $j - c$ the largest such that
  \[ a^2_{j + c } < n_2 - n.\]
  The vanishing conditions \eqref{eqn:vanishing} imply $\iota(s)$ must in fact be a linear combination of $(s^1_i,0), \dots, (s^1_k,0), (0,s^2_{k-i-c}), \dots, (0,s^2_k)$.
  But since $s_v$ for $v = 1, 2$ are the restriction to $C_v$ of a section on $C$, they satisfy a gluing condition at $p$.
  Write
  \[ \iota(s) = \sum_{\ell = i}^k \alpha_{\ell} (s^1_\ell,0) + \sum_{\ell = j}^{k} \beta_\ell (0,s^2_\ell),\]
  where $\alpha_\ell, \beta_\ell$ are in the base-field.
  By the condition \eqref{gluing:eht} in \autoref{def:eht}, the gluing condition for $s_1$ and $s_2$ is equivalent to
  \[ \alpha_{\ell} = \beta_{k+1-\ell}\]
  for $\ell = i, \dots, i+c-1$.
  Hence $\iota(s)$ must be a linear combination of the $k$ elements
  \[ (s^1_i , s^2_{k+1-i}), \dots, (s^1_{i+c-1} , s^2_{k+2-i-c}), (s^1_{i+c},0), \dots, (s^1_k,0), (0,s^2_{k+2-i}), \dots, (s^2_{k},0).\]
  It follows that $\ker \overline\iota$ is $k$-dimensional.

  Since $\ker \overline \iota$ is $k$-dimensional, there is a unique possible choice for $V_n$, namely $V_n = \ker \overline \iota$.
  It is easy to check that with this choice, the compatibility condition in \autoref{def:lls} is satisfied.
  Therefore, we get a limit linear series $\lambda$ whose associated EHT limit linear series is $\mu$.

  It remains to show $\lambda$ is simple.
  For $i = 1, \dots, k$, set $n_i = n-n_1-a^1_i$, and let $s_i \in V_{n_i} \subset H^0(C, \mathcal E_{n_i})$ to be the section whose image under $\iota$ is $(s^1_i , s^2_{k+1-i})$.
  Then the images of $s_1, \dots, s_k$ form a basis of $V_{n_1} = W_1$ and $V_{n_2} = W_2$.
  By \autoref{rem:simple_lls}, we conclude $\lambda$ is simple.
\end{proof}

\subsection{Projection-ramification for nodal curves}
\label{sec:prnodal}
Let $C$ be a smooth curve and $p \in C$ a point.
Let $E$ be a vector bundle on $C$ of rank $r$.
The projective spaces $\P E(np)|_p$, for $n \in \Z$, are canonically isomorphic to each other, so we identify them.

Suppose $\lambda \subset H^0(C, E)$ is an $(r+1)$-dimensional subspace with the vanishing sequence 
\begin{equation}\label{eqn:specialvs}
  (\underbrace{a, \dots, a}_{i}, \underbrace{a+1, \dots, a+1}_{r+1-i}),
\end{equation}
for some $i$ with $1 \leq i \leq r$, and $a \geq 0$.
Let $\Lambda_0 \subset E|_p \cong E(-ap)|_p$ be the image of $\lambda(-ap)$, and $\Lambda_1 \subset E|_p \cong E(-(a+1)p)|_p$ the image of $\lambda(-(a+1)p)$.
Then $\dim \Lambda_0 = i$ and $\dim \Lambda_1 = r+1-i$.
Assume that $\Lambda_0$ and $\Lambda_1$ satisfy the following genericity condition
\begin{equation}\label{eq:genericity}
  \dim (\Lambda_0 \cap \Lambda_1) = 1.
\end{equation}
Recall that the ramification $R(\lambda)$ is a section of $E \otimes \det E \otimes T_C$.
\begin{proposition}\label{prop:agreement}
  In the setup above, $R(\lambda)$ vanishes to order $(r+1)a + (r-i)$ at $p$.
  Write $\widetilde R = R(\lambda)/t^{(r+1)a + r-i}$, where $t$ is a uniformizer at $p$.
  Then, the one-dimensional subspace of $E|_p$ spanned by $\widetilde R|_p$ is $\Lambda_0 \cap \Lambda_1$.
\end{proposition}
\begin{proof}
  Let $\langle  s_1, \dots, s_r \rangle$ be a local trivialization for $E$ in an open set around $p$ such that in these local coordinates, we have
  \[ \lambda = \{t^as_1, \dots, t^as_i, t^{a+1}s_i, t^{a+1}s_{i+1}, \dots, t^{a+1}s_{r}\}.\]
  In these coordinates, $R(\lambda)$ is given by
  \begin{align*}
    R(\lambda) &= \det
    \begin{pmatrix}
      t^a & & & & & & at^{a-1}s_1\\
      & t^a & & & & &at^{a-1}s_2\\
      & & \ddots & & & &\vdots\\
      & & & t^a  & & &a t^{a-1}s_i\\
      & & & t^{a+1} & & &(a+1)t^a s_i \\
      & & & & \ddots & & \vdots\\
      & & & & &t^{a+1} &(a+1)t^{a}s_r      
    \end{pmatrix}\\
               &= (-1)^{r-i}t^{(r+1)a+r-i} \cdot s_i.
  \end{align*}
  Since $s_i$ spans $\Lambda_0 \cap \Lambda_1$, the proof is complete.
\end{proof}

Let $\pi \from X \to B$ be a family as in \autoref{sec:lls} with $X_\eta$ smooth and $X_0 = C$ a nodal union $C = C_1 \cup_p C_2$, with $g(C_v) = g_v$ for $v = 1,2$.
Fix $r$, $d$, $d_1$, $d_2$, $b$, $w_1$, $w_2$, $\theta_1$, and $\theta_2$ as in \autoref{sec:lls}, and take $k = r+1$.
Set
\begin{align*}
  r' &= r,\\
  d' &= d + r(d-2g+2),\\ 
  d'_1 &= d_1 + r(d_1-2g_1+1),\\
  d'_2 &= d_2 + r(d_2-2g_2+1),\\
  b' &= b(r+1),\\
  w'_1 &= w_1 + r(w_1-2g_1+1),\\
  w'_2 &= w_2 + r(w_1-2g_1+1),\\
  k' &= 1.
\end{align*}
Defining $n_1'$ and $n_2'$ analogously to $n_1$ and $n_2$, we get
\begin{align*}
  n_1' &= n_1 (1+r), \\
  n'_2 &= n_2(1+r).
\end{align*}
We define a rational map 
\begin{equation}\label{eq:Rtilde}
  \mathcal R \from {\mathcal G}^{r+1}_{r,d,d_*,w_*}(X/B)^{\rm red} \dashrightarrow {\mathcal G}^{1}_{r',d',d'_*,w'_*}(X/B)
\end{equation}
that extends the projection-ramification map on the generic fiber.
Let $\mathcal U \subset {\mathcal G}^{r+1}_{r,d,d_*,w_*}(X/B)$ be the open substack obtained by excluding the following closed loci:
\begin{enumerate}
\item the closure of the locus of linear series $(\mathcal E, \lambda)$ on the generic fiber $X_\eta$ such that $\lambda \otimes \O_{X_\eta} \to \mathcal E$ has generic rank less than $r$,
\item the locus of limit linear series $\lambda = (\mathcal E, V_n)$ on the central fiber such that the associated EHT limit linear series $\mu = (\mathcal E, W_1, W_2)$ is not refined, or does not have vanishing sequences as in \eqref{eqn:specialvs}, or does not satisfy the genericity condition $\dim (\Lambda_0 \cap \Lambda_1) = 1$ as in \eqref{eq:genericity}.
\end{enumerate}

Let $S$ be a $B$ scheme with a map to $\mathcal U$ given by the limit linear series $(\mathcal E, V_n)$.
On $X_S$, we have the diagram
\begin{equation}
  \label{eq:llspr}
  \begin{tikzcd}
    &\det \mathcal E_n^* \otimes \det V_n\ar{r}{j}\ar{d}{d} & V_n \otimes \O_{X_S}\ar{r}{i}\ar{d}{\widetilde i} & \mathcal E_n\ar[equal]{d}&\\
    0\ar{r} & \Omega_{X_S/S} \otimes \mathcal E_n\ar{r} & j_1 \mathcal E_n\ar{r} &\ar{r} \mathcal E_n \ar{r}& 0.
  \end{tikzcd}
\end{equation}
In the top row, the map $i$ is induced by the inclusion $V_n \to \pi_* \mathcal E_n$ on $S$, and the map $j$ is given by $j = \wedge^r i^* \otimes \det V_n$.
In the bottom row, the sheaf $j_1 \mathcal E_n$ is the first jet bundle of $\mathcal E_n$ along $X_S \to S$, and the row is the natural jet bundle exact sequence.
The map $\widetilde i$ is the canonical lift of $i$, and the map $d$ is the unique induced map owing to $i \circ j = 0$.
By composing $d$ through the inclusion $\Omega_{X_S/S} \to \omega_{X_S/S}$, and rearranging the line bundles, we obtain a map
\begin{equation}\label{eqn:Rn}
  R_n \from \det V_n \to \mathcal \pi_*(\mathcal E_n \otimes \det \mathcal E_n \otimes \omega^*_{X_S/S}).
\end{equation}

Set $\mathcal E' = \mathcal E \otimes \det \mathcal E \otimes \omega^*_{X_S/S}$.
We want to say that the sections given by $R_n$ of the various twists of $\mathcal E'$ define a limit linear series of dimension 1.
The catch is that the twists $\mathcal E_n \otimes \det \mathcal E_n \otimes \omega_{X_S/S}^*$ are only \emph{some} of the twists of $\mathcal E'$.
However, on the open set $\mathcal U$, this is more than enough information---just looking at the extremal twists suffices.
Observe that we have $\mathcal E_{n_v} \otimes \det \mathcal E_{n_v} \otimes \omega_{X_S/S} = \mathcal E'_{n'_v}$ for $v = 1,2$.
\begin{proposition}\label{prop:mapreduced}
  In the setup above, assume that $S$ is reduced.
  Then there is a unique simple limit linear series $(\mathcal E', L_n)$ of dimension 1 such that $L_{n_v'} = \det V_{n_v}$ and the map $L_{n_v'} \to \pi_* \mathcal E'_{n_v'}$ is given by $R_n$ for $v = 1,2$.
\end{proposition}
\begin{proof}
  First, suppose $S$ is a point mapping to $0 \in B$.
  The key is that the two extremal sections $L_{n_v} \to H^0(\mathcal E_{n_v} \otimes \det \mathcal E_{n_v} \otimes \omega_{X_S/S}^*)$ for $v = 1, 2$ define a refined EHT limit linear series.
  To see this, suppose $(\mathcal E_{n_1} |_{C_1}, V_{n_1})$ has vanishing sequence
  \[(\underbrace{a, \dots, a}_{i}, \underbrace{a+1, \dots, a+1}_{r+1-i}),\]
  at $p$ for some $i$ with $1 \leq i \leq r$ and $a \geq  0$.
  Then $(\mathcal E_{n_2}|_{C_2}, V_{n_2})$ has the vanishing sequence
  \[ (\underbrace{b-a-1, \dots, b-a-1}_{r+1-i}, \underbrace{b-a, \dots, b-a}_{i}).\]
  By construction, the section $R_{n_v}$ restricted to $C_v$ is the ramification of $(\mathcal E_{n_v}|_{C_v}, V_{n_v})$ composed with the inclusion $\omega^*_{C}|_{C_v} \to \Omega^*_{C_v}$.
  Therefore, by \autoref{prop:agreement}, $R_{n_1}$ vanishes at $p$ to order $a_1' = a(r+1)+(r-i)+1$ and $R_{n_2}$ to order $a_2' = (b-a-1)(r+1)+i$.
  Since $a'_1 + a'_2 = b'$, we have the equality required in condition \eqref{ieq:eht} of \autoref{def:eht}.
  Note that the spaces $\Lambda_0$ and $\Lambda_1$ are exchanged when we switch from $C_1$ to $C_2$, and so their intersection $\Lambda_0 \cap \Lambda_1$ remains the same.
  By \autoref{prop:agreement}, after dividing by the appropriate power of the uniformizer, the sections $R_{n_1}$ and $R_{n_2}$ at $p$ are proportional; they both span $\Lambda_0 \cap \Lambda_1$.
  Hence, we also have the gluing condition in required in \eqref{gluing:eht} in \autoref{def:eht}.

  For a general $S$, note that for every point $s \in S$, the map
  \[ H^0(X_s, \mathcal E'_n|_s) \to H^0(X_s,\mathcal E'_{n_1}|_s)/L_{n_1}|_s \oplus H^0(X_s,\mathcal E'_{n_2}|_s)/L_{n_2}|_s\]
  has kernel of dimension 1.
  If $s$ lies over the generic point of $\Delta$, then this is automatic.
  If $s$ lies over the special point of $\Delta$, then this follows from the fact that $(\mathcal E', L_{n_1}, L_{n_2})$ is a refined EHT linear series; see \eqref{eqn:keriotabar}.
  Since $S$ is reduced, we conclude that the kernel of the map
  \[ \pi_*(\mathcal E'_n) \to \pi_*(\mathcal E'_{n_1})/L_{n_1} \oplus \pi_*(\mathcal E'_{n_2})/L_{n_2} \]
  is a line bundle, say $L_n$, and the inclusion $L_n \to \pi_*(\mathcal E'_n)$ is a sub-bundle map.
  The data $(\mathcal E', L_n)$ is the unique simple limit linear series claimed in the statement.
\end{proof}

\begin{remark}
  A simple limit linear series of dimension 1 on a vector bundle $\mathcal E'$ on $C$ is simply a section of one of its twists that is non-zero on both components of $C$.
  From \autoref{prop:agreement}, we see that this twist is $\mathcal E'_{n_1+m}$ where
  \[ m = (r+1)a+ar-i.\]
  In particular, it is not one of the twists $\mathcal E'_{n_1+r(n-n_1)} = \mathcal E_n \otimes \det \mathcal E_n \otimes \omega^*$ in \eqref{eqn:Rn} that receive the image of the ramification section $R_n$.
\end{remark}

Thanks to \autoref{prop:mapreduced}, we have a morphism
\begin{equation}\label{eqn:llsR}
 \mathcal R \from \mathcal U^{\rm red} \to \mathcal G^1_{r',d',d'_*,w'_*(X/B)}
\end{equation}
defined by
\[ (\mathcal E, V_n) \mapsto (\mathcal E', L_n).\]

\subsection{Maximal variation}
Let $E$ be an ample vector bundle on $\P^1$ of rank $r$.
Fix a point $p \in \P^1$.
Consider  the locally closed subset $U \subset \Gr(r+1, H^0(E))$ consisting of linear series with vanishing sequence
\[ (0,\underbrace{1,\dots, 1}_{r})\]
over $p$.
Given $\lambda \in U$, let $\widetilde R(\lambda) \in \P H^0(E \otimes \det E \otimes T_{\P^1} \otimes \O(-(r-1)p)$ be the reduced ramification divisor; see \autoref{prop:agreement}.
The assignment $\lambda \mapsto \widetilde R(\lambda)$ gives a \emph{reduced} projection-ramification map
\begin{equation}\label{eqn:rrd}
  \widetilde R \from U \to \P H^0(E \otimes \det E \otimes T_{\P^1} \otimes \O(-(r-1)p))
\end{equation}
between varieties of the same dimension.

Given a one-dimensional subspace $\ell \subset E|_p$, define $E'_\ell$ by the exact sequence
\[ 0 \to E_\ell' \to E \to E|_p/\ell\to 0.\]
There exists a Zariski open subset of $\P_{\rm sub}(E|_p)$, such that for all $\ell$ in this set, the isomorphism class of $E'_{\ell}$ remains constant.
Denote this isomorphism class by $E'_{\rm gen}$.
\begin{proposition}\label{prop:domred}
  If the usual projection-ramification map
  \[ R\from \Gr(r+1, H^0(E'_{\rm gen})) \dashrightarrow \P H^0(E'_{\rm gen} \otimes \det E'_{\rm gen} \otimes T_{\P^1})\]
  is dominant, then so is the reduced projection-ramification map
  \[\widetilde R\from U \to \P H^0(E \otimes \det E \otimes T_{\P^1} \otimes \O(-(r-1)p)).\]
\end{proposition}
\begin{proof}
  Let $D \in \P H^0(E \otimes \det E \otimes T_{\P^1} \otimes \O(-(r-1)p))$ be a generic section.
  Let $\ell \subset E|_p$ be the one-dimensional subspace defined by $D|_p$, and set $E' = E'_{\ell}$.
  Since $D$ is generic, we may assume $E' \cong E'_{\rm gen}$.
  We have the inclusion
  \[
    E' \otimes \det E' \otimes T_{\P^1} \to E \otimes \det E \otimes \O(-(r-1)p) \otimes T_{\P^1},
  \]
  and by construction $D$ is the image of a section $D' \in \P H^0(E' \otimes \det E' \otimes T_{\P^1})$.
  Since $R$ is dominant for $E'$, there exists a sequence of subspaces $\lambda_n' \in \Gr(r+1, H^0(E'))$ such that $R(\lambda_n')$  limit to $D'$.
  Let $\lambda_n \subset \Gr(r+1, H^0(E))$ be the image of $\lambda'_n$.
  Then $\widetilde R(\lambda_n)$ limit to $D$.
  Since $D$ was generic, we get that $\widetilde R$ is dominant.
\end{proof}

\begin{corollary}\label{prop:domredexamples}
  The reduced projection-ramification map is dominant for the bundles $E = \O(1) \oplus \O(2)^{r-1}$ and $E = \O(2) \oplus \O(3)^{r-1}$.
\end{corollary}
\begin{proof}
  Follows from \autoref{prop:domred} and that the projection-ramification map is dominant for $E' = \O(1)^r$ and $E' = \O(2)^r$.
\end{proof}

For $v = 1, 2$, let $C_v$ be a smooth curve and $E_v$ a vector bundles of rank $r$ on $C_v$.
Let $p_v \in C_v$ be a point.
Suppose $\lambda_1 \in \Gr(r+1, H^0(E_1))$ is a linear series with vanishing sequence $(0, \dots, 0, 1)$ at $p_1$, and $\lambda_2 \in \Gr(r+1,H^0(E_2))$ is a linear series with vanishing sequence $(0,1,\dots,1)$ at $p_2$.

Let $C$ be the nodal union of $C_1$ and $C_2$ formed by identifying $p_1$ and $p_2$.
We construct a simple limit linear series $\lambda$ on $C$ of rank $r$ and degree $\deg E_1 + \deg E_2 - r$.
Choose an isomorphism $\phi \from E_1(-p)|_{p_1} \to E_2|_{p_2}$ that sends the image of $\lambda_1(-p)$ in $E_1(-p)|_{p_1}$ to the image of $\lambda_2$ in $E_2|_{p_2}$.
Let $\mathcal E$ be the vector bundle on $C$ constructed by gluing $E_1(-p)$ and $E_2$ by $\phi$.
Let $b=2m$ be large enough so that $H^0(E_1(-mp)) = 0$ and $H^0(E_2(-mp)) = 0$.
Set $d_1 = \deg E_1 + (m-1)r$ and $d_2 = \deg E_2 + mr$.
Then $w_1 = d_1 - r$, so $n_1 = m$; and $w_2 = d_2$, so $n_2 = -m$.
Let $V_{n_1} \subset H^0(C, \mathcal E_{n_1})$ be the subspace that restricts to $\lambda_1((m-1)p) \subset H^0(C_1, E_1((m-1)p))$ and $V_{n_2} \subset H^0(C, \mathcal E_{n_2})$ the subspace that restricts to $\lambda_2(mp) \subset H^0(C_2, E_2(mp))$.
Then the vanishing sequence of $V_{n_1}$ is $(m-1,\dots,m-1,m)$ and that of $V_{n_2}$ is $(m, m+1, \dots, m+1)$.
By the choice of $\phi$, we see that the two series are compatible at the node, and hence define a refined EHT limit linear series on $C$.
Let $\lambda$ be the associated unique simple limit linear series.

Let $X \to B$ be a smoothing of $C$, and $\mathscr{E}$ a vector bundle on $X$ whose restriction to $X_0 = C$ is $\mathcal E$.
Set $d = d_1 + d_2 - r$ and $g = g(C_1) + g(C_2)$.
\begin{proposition}\label{prop:attachtail}
  Suppose $\lambda_v$ is isolated in their respective projection-ramification maps, for $v = 1,2$.
  Then $\lambda$ is isolated in the projection-ramification map $\mathcal R$.
  Suppose, furthermore, that the dimension of the fiber of the forgetful map $\beta$ at $\lambda$ is $(r+1)(d-1-rg)$.
  Then the projection-ramification map is generically finite for the vector bundle $\mathscr E_\eta$ on $X_\eta$.
\end{proposition}
\begin{proof}
  The projection-ramification map for $\lambda$ reduces to the projection-ramification map for $\lambda_v$ (up to twists) on components $C_v$.
  So if both $\lambda_v$ are isolated in the fibers of their projection-ramification maps, so is $\lambda$.

  If the dimension condition holds, then $\beta$ is open at $\lambda$ by \autoref{thm:lls}.
  In particular, $\lambda$ is in the closure of $\Gr(r+1, H^0(X_\eta, \mathscr E_\eta))$.
  By the semi-continuity of fiber dimension, it follows that the projection-ramification map 
  \[ R \from \Gr(r+1, H^0(X_\eta, \mathscr E_\eta)) \to \P H^0(X_\eta, \mathscr E_\eta \otimes \det \mathscr E_\eta \otimes \omega_{X_\eta}^*)\]
  is generically finite.
\end{proof}

\begin{theorem}\label{thm:prp1}
  Let $E$ be a generic vector bundle on $\P^1$ of rank $r$ and degree $d = a(r-1) + b(2r-1)+1$, where $a, b$ are positive integers.
  Then the projection-ramification map is generically finite, and hence dominant, for $E$.
  In particular, the projection-ramification map is dominant for generic $E$ of degree $\geq (r-1)(2r-1)+1$.
\end{theorem}
\begin{proof}
  Let $E$ be a generic vector bundle of rank $d \geq 0$ such that the projection-ramification map is dominant for $E$.
  Set $C_1 = \P^1$ and $E_1 = E$.
  Take $C_2 = \P^1$ and $E_2 = \O(1)\oplus\O(2)^{r-1}$ or $E_2 = \O(2) \oplus \O(3)^{r-1}$.
  Pick points $p_v \in C_v$ for $v = 1, 2$.
  Let $\lambda_1 \subset H^0(E_1)$ be an $(r+1)$-dimensional subspace with vanishing sequence $(0, \dots, 0, 1)$ at $p_1$, and $\lambda_2 \subset H^0(E_2)$ an $(r+1)$-dimensional subspace with vanishing sequence $(0, 1, \dots, 1)$.
  Assume that $\lambda_v$ are isolated in the respective fibers of their projection-ramification maps.
  Furthermore, assume that the image $\ell$ of $\lambda_2$ in $E_2|_{p_2}$ is generic in the sense that the kernel of
  \[ E_2 \to E_2|_{p_2}/ \ell \]
  is the generic vector bundle $E_2^{\rm gen}$ (which will be either $\O(1)^r$ or $\O(2)^r$).
  Let $\lambda$ be the limit linear series on $C = C_1 \cup C_2$ constructed from $\lambda_1$ and $\lambda_2$ as above.
  Then, we get
  \begin{align*}
    \dim_\lambda \beta^{-1}(\beta(\lambda)) &= \dim \Gr(r+1, H^0(E_1)) + \dim \Gr(r+1, H^0(E_2^{\rm gen}))\\
                                            &= (r+1)(\deg E + \deg E_2^{\rm gen}-2) \\
                                            &= (r+1)(\deg E + \deg E_2 - r - 1) \\
                                            &= (r+1)(\deg \mathcal E - 1).
  \end{align*}
  Here is how the dimension count goes: it suffices to count dimensions for the refined EHT limit linear series associated to $\lambda$, since $\lambda$ can be recovered uniquely from it.
  For the EHT limit linear series, we begin by choosing an $(r+1)$-dimensional subspace of $H^0(E_1)$, giving us the first term in the dimension count.
  This choice gives a one-dimensional subspace $\Lambda_1 \in E_1|_{p_1} = \mathcal E|_p = E_2|_{p_2}$.
  We must choose an $(r+1)$-dimensional subspace of $H^0(E_2)$ with the complementary vanishing sequence and satisfying the compatibility condition over $p$.
  These two conditions force it to be a subspace of $H^0(E_2')$, where $E_2'$ is the kernel of
  \[E_2 \to E_2|_{p_2}/\Lambda_1.\]
  Since the kernel is isomorphic to $E_2^{\rm gen}$, we get the second term in the dimension count.
  
  Let $\pi \from X \to B$ be a smoothing of $B$.
  Every vector bundle on $C$ is the restriction of a vector bundle on $X$.
  Indeed, this is clearly true for line bundles on $C$, and hence for vector bundles of arbitrary rank, as these are direct sums of line bundles. 
  Let $\mathscr E$ be a vector bundle on $X$ whose restriction to $C$ is $\mathcal E$.
  By \autoref{prop:attachtail}, the projection-ramification map is generically finite, and hence dominant, for $\mathcal E_\eta$.
  By the semi-continuity of fiber dimension, the same is true for a generic vector bundle of rank $r$ and degree $\deg E + \deg E_2 - r $.
  The two choices of $E_2$ give $\deg E_2 - r = r-1$ and $\deg E_2 - r = 2r-1$.

  In summary, dominance for a generic bundle of rank $r$ and degree $d$ implies the same for a generic bundle of rank $r$ and degree $d+r-1$ and degree $d+2r-1$.
  Starting with the base case $d = 1$, namely $E = \O^{r-1} \oplus \O(1)$, we obtain the statement by induction.
\end{proof}



\subsection{Maximal variation for $\O(2)^r$}
The goal of this section is to establish dominance of the projection ramification morphism for $E = \O(2)^r$.
We do this by a tangent space calculation.
For simplicity, we work with inhomogeneous polynomials in $x = X/Y$ instead of homogeneous polynomials in $X$ and $Y$.

Consider the point $\lambda$ of $\Gr(r+1, H^0(E))$ represented by the $(r+1) \times r$ matrix
\[
  \Lambda =
  \begin{pmatrix}
    (x-a_1)^2 & 0 &  \cdots & 0 \\
    0 & (x-a_2)^2 & \cdots & 0\\
    0 & 0 & \ddots & 0 \\
    0 & 0 & \cdots & (x-a_r)^2\\
    p_1 & p_2 & \cdots & p_r
  \end{pmatrix},
\]
where $a_i \in \C$ and $p_j \in H^0(\O(2))$.
We claim that if the $a_i$ and the $p_j$ are generic, then the map on the tangent spaces induced by the projection-ramification construction is surjective at $\Lambda$.

Recall that if $M$ is an $(r+1) \times r$ matrix of polynomials in $x$, then the ramification divisor of the projection map represented by $M$ is given by the formula
\[
  R(M)  = \det \left(M \mid \xi(M)\right),
\]
where $\xi(M)$ is the vector given by
\[
  \xi(M)_i = \sum_{j = 1}^r \partial_x M_{i,j} \cdot X_j.
\]

To do the tangent space computation, we compute the ramification divisor $R$ for the matrix $M  = \Lambda + \epsilon \Delta$, where $\Delta$ is an $(r+1) \times r$ matrix of elements in $H^0(\O(2))$, assuming $\epsilon^2 = 0$.
The result will be of the form
\[ R(\Lambda + \epsilon \Delta) = R(\Lambda) + \epsilon S(\Lambda, \Delta),\]
where $S(R, \Delta)$ is an element of $H^0(2r) \otimes \langle  X_1, \dots, X_r \rangle$, linear in the entries of $\Delta$.
We must show that the linear map
\[ H^0(\O(2)) \otimes M_{r+1, r} \to H^0(\O(2r)) \otimes \langle  X_1, \dots, X_r \rangle\]
given by
\[ \Delta \mapsto S(\Lambda, \Delta)\]
is surjective.

We compute $R(\Lambda + \epsilon\Delta)$ for elementary matrices $\Delta$.
Denote by $E_{i,j}$ the elementary matrix with $1$ at the $(i,j)$th place, and $0$ everywhere else.

First, suppose
\[ \Delta = q E_{j,i},\]
where $i \neq j$, and $1 \leq j \leq r$.
By direct calculation, we obtain
\begin{equation}\label{eq:off_diagonal}
  S(\Lambda, \Delta) = \frac{(x-a_1)^2\cdots(x-a_r)^2 p_j}{(x-a_i)^2(x-a_j)^2} \cdot \left[q, (x-a_i)^2\right] \cdot X_i,
\end{equation}
where the notation $[a, b]$ means $ab' - a'b$.

Second, suppose
\[ \Delta = q E_{r,i},\]
where $1 \leq i \leq r$.
Again, by direct calculation, we obtain
\begin{equation}\label{eq:bottom}
  S(\Lambda, \Delta) = - \frac{(x-a_1)^2 \cdots (x-a_r)^2}{(x-a_i)^2} \cdot \left [q, (x-a_i)^2\right] \cdot X_i.
\end{equation}

Third, suppose
\[ \Delta = qE_{i,i},\]
where $1 \leq i \leq r$.
As before, by direct calculation, we obtain
\begin{equation}\label{eq:diagonal}
  S(\Lambda, \Delta) = R(\Lambda_i(q)),
\end{equation}
where $\Lambda_i(q)$ is obtained from $\Lambda$ by changing the $(i,i)$th entry from $(x-a_i)^2$ to $q$.

We want to show that the map
\begin{equation}\label{eqn:mainmap}
  H^0(\O(2)) \otimes M_{(r+1) \times r} \to H^0(\O(2r)) \otimes \langle  X_1, \dots, X_r \rangle
\end{equation}
given by
\[ \Delta \mapsto S(\Lambda, \Delta)\]
is surjective.
Fix a $i$ with $1 \leq i \leq r$ and consider the subspace of the domain given by
\[ H^0(\O(2)) \otimes \langle  E_{j,i}  \mid j \neq i \rangle.\]
By our calculations above, the image of this space lies in $H^0(\O(2r)) \otimes X_i$.
We begin by identifying the image.
For $1 \leq j \leq r$ and $j \neq i$, set
\[ Q_{i,j} = \frac{(x-a_1)^2\cdots (x-a_r)^2p_j}{(x-a_i)^2(x-a_j)^2}\]
and set
\[ Q_{i, r+1} = \frac{(x-a_1)^2 \cdots (x-a_r)^2}{(x-a_i)^2.}\]

\begin{lemma}\label{lem:no_linear_syzygy}
  For generic $p_1, \dots, p_r$, there is no non-trivial linear relation among the polynomials $Q_{i,j}$ for $j \in \{1, \dots, r+1\} \setminus \{i\}$.
\end{lemma}
\begin{proof}
  Suppose we had a linear relation
  \[ \sum l_j Q_{i,j} = 0.\]
  Divide throughout by $\frac{(x-a_1)^2 \dots (x-a_r)^2}{(x-a_i)^2}$.
  Then we get the relation
  \[ \sum_{j=1}^r l_j \frac{p_j}{(x-a_j)^2} + l_{r+1} = 0.\]
  If $l_j \neq 0$ for some $j$ with $1 \leq j \leq r$, then we have a pole on the left hand side at $x = a_j$, but not on the right hand side; a contradiction.
  Therefore, we must have $l_j = 0$ for all $j$ with $1 \leq j \leq r$, and hence also $l_{r+1} = 0$.
  Thus, the relation was trivial.
\end{proof}
\begin{lemma}\label{lem:jacobian}
  The image of the map
  \[ H^0(\O(2)) \to H^0(\O(2))\]
  given by
  \[ q \mapsto [q, (x-a)^2]\]
  is
  \[ (x-a) \cdot H^0(\O(1)).\]
\end{lemma}
\begin{proof}
  Straightforward.
\end{proof}
\begin{lemma}\label{lem:imageXi}
  The image of the map
  \[ H^0(\O(2)) \otimes \langle  E_{j,i} \mid j \in \{1, \dots, r+1\} \setminus \{i\} \rangle \to H^0(\O(2r)) \otimes X_i\]
  is
  \[ (x-a) \cdot H^0(\O(2r-1)) \otimes X_i.\]
\end{lemma}
\begin{proof}
  By the computation in \eqref{eq:off_diagonal} and \eqref{eq:bottom} and \autoref{lem:jacobian} the image of the map above is the same as the image of the multiplication map
  \[ \langle Q_{i,j} \mid j \in \{1, \dots, r+1\} \setminus \{i\} \rangle \otimes (x-a) \cdot H^0(\O(1)) \to H^0(\O(2r)) \otimes X_i.\]
  By \autoref{lem:no_linear_syzygy}, the map
  \[ \langle Q_{i,j} \mid j \in \{1, \dots, r+1\} \setminus \{i\} \rangle \otimes H^0(\O(1)) \to H^0(\O(2r-1))\]
  is injective.
  Since both sides have dimension $2r$, the map is an isomorphism.
  The proof is now complete.
\end{proof}

By \autoref{lem:imageXi}, the cokernel of the map
\[ H^0(\O(2)) \otimes \langle  E_{j,i} \mid j \in \{1, \dots, r+1\} \setminus \{i\} \rangle \to H^0(\O(2r)) \otimes X_i\]
is $\C \otimes X_i$, where the map $H^0(\O(2r)) \to \C$ is the evaluation at $a_i$.
Putting all these maps together for various $i$, we get that the cokernel of the map
\[ H^0(\O(2)) \otimes \langle  E_{j,i} \mid j \neq i \rangle \to H^0(\O(2r)) \otimes \langle  X_1, \dots, X_r \rangle\]
is $\C \otimes \langle  X_1, \dots, X_r \rangle$,
where the map
\begin{equation}\label{eqn:remaining}
  H^0(\O(2)) \otimes \langle  X_1, \dots, X_r \rangle \to \C \otimes \langle  X_1, \dots, X_r \rangle
\end{equation}
is given on $H^0(\O(2r)) \otimes X_i$ by the evaluation at $a_i$.

It remains to show that the map
\begin{equation}\label{eq:remaining_diagonal}
  H^0(\O(2)) \otimes \langle  E_{i,i} \rangle \to \C \otimes \langle  X_1, \dots, X_r \rangle,
\end{equation}
obtained by composing \eqref{eqn:mainmap} and \eqref{eqn:remaining}, is surjective.
Recall from \eqref{eq:diagonal} that the image of $q E_{i,i}$ is given by $R(\Lambda_i(q))$.
Suppose $q = (x-a_i)l$, where $l(a_i) \neq 0$.
Then, we have
\[ R(\Lambda_i(q))_{x = a_j} = 0\]
for $j \neq i$, and
\[ R(\Lambda_i(q))_{x = a_i} = \pm l(a_i) \prod_{j \neq i}(a_i-a_j)^2 p_i(a_i) X_i.\]
Thus, up to scaling, $qE_{i,i}$ maps to $X_i$ under \eqref{eq:remaining_diagonal}.
Therefore, the map \eqref{eq:remaining_diagonal} is surjective.




% section proof_of_theorem:rationalnormalscrolls (end)



\section{The Projection-Ramification enumerative problem} % (fold)
\label{sec:enumerativeproblems}

Our objectives in this section are to prove \autoref{Thm:Examples}, and also to relate special instances of the projection-ramification enumerative problem with constructions in classical algebraic geometry.


\subsection{Quadric hypersurfaces} % (fold)
\label{sub:a_quadric_surface}
A smooth quadric hypersurface $X \subset \P^{n}$ defined by an equation $F(x_{0},x_{1},x_{2}, ...) = 0$ induces the classical {\sl polarity isomorphism} $\P^{n} \longleftrightarrow (\P^{n})^{\smvee}$ given by 
\begin{align*}
  p = [p_{0}:p_{1}:p_{2}: ...] \mapsto [\partial_{0}F(p):\partial_{1}F(p):\partial_{2}F(p): ...]
\end{align*}
where $\partial_{i}$ denotes derivative with respect to the $i$-th variable $x_{i}$. The duality morphism is equal to $\rho_{X}$, and hence $\deg \rho_{X} = 1$.  


% subsection a_quadric_surface (end)

\subsection{The Veronese surface} % (fold)

Let $\P^{2} \simeq X \subset \P^{5}$ be the Veronese surface. Then the projection-ramification morphism 
\begin{align*}
  \rho_{X}: \Gr(3,6) \dashrightarrow \P^{9}
\end{align*}
assigns to a general net of conics $N$ the cubic curve $C \subset \P^{2}$ consisting of the nodes of the singular members of $N$.  We will outline the classical algebraic geometry (the relationship between cubic curves and their Hessians) underlying the claim that the degree of $\rho_{X}$ is $3$. 

Suppose $N = \langle Q_{1}, Q_{2}, Q_{3} \rangle$ is a general net of conics in $\P^{2}$, with $Q_{i}$ general ternary quadratic forms. For each line $L \subset \P^{2}$, the net $N$ restricts either to the complete linear series of two points on $L$ or it restricts to a pencil.   The latter type of line a {\sl Reye} line. 

\begin{lemma}
   \label{lem:specialLines}
   The set of Reye lines $C' \subset \P^{2*}$ is a smooth  cubic equipped with a fixed-point free involution $\tau$ with quotient isomorphic to $C$.
 \end{lemma}  

\begin{proof}
  Each Reye line $L$ arises from a unique singular conic of the net $N$, and hence possesses a conjugate line $L'$, which defines the fixed point free involution $\tau$. 

  Let $S \to \P^{2*}$ denote the rank $2$ tautological subbundle. The forms $Q_{1},Q_{2},Q_{3}$ define a map of vector bundles 
  \begin{align*}
    \O^{3} \to \Sym^{2}S^{*}
  \end{align*}
  The determinant of this map defines the locus of Reye lines, and a simple Chern class calculation reveals the locus is a cubic $C' \subset \P^{2*}$.  The quotient of $C'$ by the involution sending $L$ to $L'$ is clearly identifiable with $C \subset \P^{2}$.
\end{proof}


If $L$ is a Reye line, then $L$ is a component of a unique singular member of the net $N$, and therefore has a conjugate line $L'$.  On $L$, there are now three points of significance: $x = L \cap L'$, which is clearly a point on $C$, and the residual pair of points $a_{L}, b_{L} \in L \cap C$. Similarly for $L'$.  We may view $C'$ as a $2:1$ unramified cover of $C$. Lying above the points $a_{L},b_{L} \in C$ are points $a_{L}',a_{L}'', b_{L}', b_{L}'' \in C'$. 

\begin{lemma}
  \label{lem:lineEq}
  Maintain the setting above. Then on $C$ the following linear equivalences hold: $2a_{L} \sim 2b_{L} \sim 2a_{L'} \sim 2b_{L'}$.
\end{lemma}

\begin{proof}
  Attached to $a_{L}$ and $b_{L}$ are the dual lines $a_{L}^{*}, b_{L}^{*} \subset \P^{2*}$. The intersections $a_{L}^{*} \cap C'$ and $b_{L}^{*} \cap C'$ both contain the point $[L] \in \P^{2*}$. The residual intersections of $C'$ with $a_{L}^{*}$ and $b_{L}^{*}$ are the points $a_{L}', a_{L}''$ and $b_{L}', b_{L}''$ in $C'$. Hence, on $C'$ we get a linear equivalence $a_{L}'+ a_{L}'' \sim b_{L}' +  b_{L}''$. Pushing this linear equivalence forward under the quotient map $C' \to C$ gives $2a_{L} \sim 2b_{L}$. 

  Proceeding in a similar way, note that the points $[L], [L'] \in C'$ constitute the two points in $C'$ lying above $x \in C$. Further, on $C'$ we get the equivalence $[L] + a_{L}'+ a_{L}'' \sim [L'] + a_{L'}'+ a_{L'}''$ since both triads are collinear in $\P^{2*}$.  Pushing this equivalence forward to $C$ yields the equivalence $2a_{L} \sim 2a_{L'}.$ 
\end{proof}




\begin{lemma}
  \label{lem:2torsionclass}
  The class $\eta = a_{L}-b_{L} \in Jac(C)[2] \setminus \{0\}$ is independent of the point $x$ and the choice of Reye line $L$.
\end{lemma}
\begin{proof}
  For each $x \in C$ there are two Reye lines $L,L'$ containing $x$, and two pairs of points $a_{L},b_{L}$ and $a_{L'},b_{L'}$ respectively. 

  Now, if four points $p,q,r,s \in C$ satisfy $2p \sim 2q \sim 2r \sim 2s$, then it is always true that $p-q 
  \sim r-s$, as a straightforward divisor calculation shows. 

  The lemma now follows by the fact that there are only finitely many $2$-torsion divisor classes on $C$, combined with the fact that $C'$ is irreducible.
\end{proof}


The $2$-torsion class $\eta$ defines a translation on $C$ which takes a point $x \in C$ to the unique point denoted $\eta(x) \in C$ which is linearly equivalent to $x + \eta$. Therefore, \autoref{lem:2torsionclass} allows us to describe the set of Reye lines as the lines joining $p$ with $\eta(p)$ for all points $p \in C$.  




Thanks to \autoref{lem:2torsionclass}, we see the projection-ramification map $\rho_{X}$ factors as: 
\begin{align}\label{eq:compose}
  \rho_{X}: \Gr(3,6) \dashrightarrow Jac[2] \dashrightarrow \P^{9}
\end{align}
where $Jac[2]$ is the variety parametrizing pairs $(C,\eta)$ with $C$ a smooth plane cubic and $\eta \in Jac(C)[2]$ a non-trivial $2$-torsion element. 


To conclude, we argue the first map in \eqref{eq:compose} is birational by constructing its inverse.  To this end, suppose $C$ is a smooth plane cubic, and $\eta \in Jac(C)[2]$ a chosen non-trivial $2$-torsion element. We will create from this data a net of conics $N$ whose set of nodes is $C$. Again, we think of $\eta$ as a translation $C \to C$ in the usual way.

  For every point $p \in C$, we get a line $L_{p} \subset \P^{2}$ joining $p$ and $\eta(p)$. In this way, we obtain a map $f: C \to \P^{2*}$ which is $2:1$ onto its image, since $L_{p} = L_{q}$ if and only if $p=q$ or $\eta(p)=q$.  Since  $\eta:C \to C$ is fixed-point free, it is easy to see that $f$ is also unramified. Hence, the image of $f$ must be a smooth cubic. 

  If $\beta \neq \eta \in Jac(C)[2]$ is any other non-trivial 2-torsion element, the pair of points $\beta(p), \beta(\eta(p))$ span a well-defined second line $L'_{p}$ containing the point $p$.  

  The collection of singular conics $L_{p} \cup L_{p}'$ parametrized by $p \in C$ induces a map $C \to \P^{5}$, whose degree is $3$, since  through a general point in $\P^{2}$ there pass $3$ of the lines $L_{p}$.  Furthermore, a divisor class computation shows that the node point $L_{p} \cap L_{p'}$ is again on $C$. Hence the image of $C \to \P^{5}$ spans a plane which by construction is the desired net of conics $N$ whose locus of nodes is $C$.


\subsection{Quartic surface scroll} Our next objective is to prove that $ \deg \rho_{X} = 2$ for a generic quartic surface scroll $X \subset \P^{5}$.  Our proof uncovers a  rich geometric picture similar to the case of the Veronese surface in the previous subsection.  


We begin with the following seemingly unrelated geometric figure: $C \subset \P^{2}$ is a smooth cubic curve, $a \in \P^{2} \setminus X$ a point, and $Q \subset \P^{2}$ the {\sl polar conic} of $a$ with respect to $C$ -- the unique conic which passes through the six points of ramification on $C$ of the projection from $a$.   We assume $a$ is chosen so that $Q$ is a smooth conic. 

To set notation moving forward, if $x \in \P^{2}$ is any point, we let $P_{x}(C)$ denote the polar conic of $x$ with respect to $C$.  Similarly, we let $P_{x}(Q)$ denote the polar line of $x$ with respect to the conic $Q$.

\begin{lemma}\label{lemma:basicsaboutHessian}
The Hessian $Hess(C) \subset \P^{2}$ consists of the points $x$ such that $P_{x}(C)$ is singular, and if $C$ is not a Fermat cubic $P_{x}(C)$ is the union of two distinct lines for every $x \in Hess(C)$.  Furthermore, if $x \in Hess(C)$, then the unique singularity $s(x)$ of $P_{x}(C)$ lies on $Hess(C)$, and the map $x \mapsto s(x)$ is translation by a $2$-torsion point $\eta \in \Jac( Hess(C))$.
\end{lemma} 

\begin{proof}
    Standard. \todo{Reference, probably Dolgachev}
\end{proof} 

\begin{proposition}\label{proposition:polarline}
Suppose $x \in Hess(C)$. Then the line $P_{x}(C)$ passes through the point $x + \eta \in Hess(C)$.
\end{proposition}

\begin{proof}
    We have: 
    \begin{align*}
        P_{x}(Q) = P_{x}P_{a}(C) = P_{a}P_{x}(C).
    \end{align*}

Since $x \in Hess(C)$, $P_{x}(C)$ is a singular conic. Hence $P_{x}(Q)$ must pass through the singularity $\sing P_{x}(C)$, which by \autoref{lemma:basicsaboutHessian} is the point $x + \eta$.
\end{proof}

Next suppose $\ell_{1},m_{1}, \ell_{2}, m_{2}, \ell_{3}, m_{3}$ are six distinct lines in $\P^{2}$ with the  properties: 
\begin{enumerate}
    \item The three singular conics $\ell_{i} \cup m_{i}$ are polars of $C$.
    \item The triangle $\ell_{1}\ell_{2} \ell_{3}$ is {\sl conjugate} to the triangle $m_{1}  m_{2}  m_{3}$ with respect to $Q$.
\end{enumerate}
The second condition above simply means that the vertices of one triangle are polar to the lines of the other triangle.  By basic projective geometry, the two triangles are then in {\sl linear perspective}, i.e. the three points $x_{1} := \ell_{1} \cap m_{1}, x_{2} := \ell_{2} \cap m_{2}, x_{3} := \ell_{3} \cap m_{3}$ are collinear.

\begin{proposition}\label{proposition:importantReyeLineFact}
Maintain the notation above, and recall the definition of Reye line from the previous subsection. The lines $P_{x_{i}}(Q)$ are Reye lines of the net of polar conics of $C$.
\end{proposition}

\begin{proof}
    The triangles $\ell_{1}\ell_{2}\ell_{3}$ and $m_{1}m_{2}m_{3}$ are conjugate with respect to $Q$.  Hence it follows that the polar line $P_{x_{3}}(Q)$ equals $\ell := \overline{\ell_{12}m_{12}}$, where $\ell_{ij} = \ell_{i} \cap \ell_{j}$ and $m_{ij} = m_{i} \cap m_{j}$.
    
    We will prove that $\ell$ is one of the three Reye lines of the net of polars of $C$ which pass through the point $\ell_{12}$, the other two Reye lines being $\ell_{1}$ and $\ell_{2}$. From \autoref{lemma:basicsaboutHessian}, we can find points $y,z \in C$ and write: 
    \begin{align*}
        \ell_{1} = \overline{y, y + \eta}\\
        \ell_{2} = \overline{z, z + \eta}
    \end{align*}
Then a divisor class computation shows that the third Reye line through $\ell_{12}$ must be $\overline{w,w + \eta}$, with $w$ satisfying 
\begin{align*}
    y+z+w \sim H + \epsilon,
\end{align*}
where $\epsilon$ is any one of the two non-trivial $2$-torsion elements on $Hess(C)$ differing from $\eta$.  We let $s \in Hess(C)$ denote the third point of intersection of the line $\overline{w,w + \eta}$ with $C$. (Notice that $w$ depends on the choice of $\epsilon$, but the line $\overline{w,w + \eta}$ is independent of this choice.)

From this setup, we get: 
\begin{align*}
    x_{1} &\sim H-2y-\eta\\
    x_{2} &\sim H-2z - \eta\\
    s &\sim H-2w - \eta
\end{align*}
from which we get:
\begin{align*}
    s &\sim H-2w - \eta\\
    &\sim H - 2[H+\epsilon -y-z] - \eta\\
    &\sim 2y+2z-H-\eta\\
    &\sim H-x_{1}-\eta + H-x_{2}-\eta-H-\eta\\
    &\sim H-x_{1}-x_{2} - \eta.
\end{align*}

Therefore, to prove that $\ell$ is a Reye line, it suffices to show that the points $\ell_{12}, m_{12}$, and $s \sim H-x_{1}-x_{2}-\eta$ are collinear.  But, this is true if and only if their respective polar lines $P_{\ell_{12}}(Q), P_{m_{12}}(Q), P_{s}(Q)$ are concurrent.  The latter is true if and only if the lines $m_{3}, \ell_{3}, P_{s}(Q)$ are concurrent, which in turn  translates to the condition that $x_{3} \in P_{s}(Q)$.  But, $x_{3} \sim H-x_{1}-x_{2}$, and $s \sim H-x_{1}-x_{2}+\eta$, and so by \autoref{proposition:importantReyeLineFact}, we conclude that indeed $x_{3} \in P_{s}(Q)$, which is what we needed to show.
\end{proof}

\subsubsection{Returning to the projection ramification problem} Our next objective is to relate the geometry in the previous subsection to 



\subsection{Rational curves, the differential construction, and the case of Segre varieties} % (fold)
\label{sec:rational_curves_in_projective_space}
\autoref{problem:degree} connects with an old story involving rational curves in projective space.  

Let $\gamma: \P^{1} \to \P^{n}$ be a degree $d$ morphism. Its derivative 
\begin{align*}
    d\gamma:T_{\P^{1}} \to \gamma^{*}(T_{\P^{n}})
  \end{align*}  
may be viewed as a global section of the rank $r$ vector bundle $\gamma^{*}(T_{\P^{n}}) \otimes T_{\P^{1}}^{\smvee}$.  The splitting of $\gamma^{*}(T_{\P^{n}})$ is known to be balanced for a general morphism $\gamma$. In particular, if the divisibility
\begin{align*}
  n \mid d
\end{align*}
holds, and if we set $\ell := d+d/n-2$, then a general $\gamma$ satisfies: 
\begin{align*}
  (\gamma^{*}T_{\P^{n}}) \otimes T_{\P^{1}}^{\smvee} \simeq \bigoplus_{i=1}^{n} \O_{\P^{1}}(\ell).
\end{align*} 
The direct sum decomposition is not canonical, it is only defined up to the
action of $GL_{n}(k)$.  

Assuming $\gamma$ is an immersion, the element $d \gamma \in
H^{0}(\P^{1},\bigoplus_{i=1}^{n} \O_{\P^{1}}(\ell))$ does not vanish anywhere,
and hence defines a degree $\ell$ map $$D(\gamma) : \P^{1} \to \P^{n-1},$$
only well-defined up to the action of post-composition by $PGL_{n}(k)$. 

\begin{definition}
  Let $M^{n}_{d}$ denote the moduli stack parametrizing $PGL_{n+1}(k)$
  equivalence classes of degree $d$ maps $\gamma : \P^{1} \to \P^{n}$, and let
  $U^{n}_{d} \subset M^{n}_{d}$ denote the open substack parametrizing local immersions with $\gamma^{*}(T_{\P^{n}})$ balanced.
\end{definition}

\begin{remark}
Notice:  $\dim M^{n}_d= (k+1)(n+1) - (n+1)^{2} = (n+1)(k-n) = \dim \G(n, k).$  Furthermore, notice  $PGL_{2}(k)$ acts on $U^{n}_{d}$ and $M^{n}_{d}$ by pre-composition. 
\end{remark}  

\begin{remark}
  Though $M^{n}_{d}$ is an Artin stack, the open substack $U^{n}_{d}$ is a scheme, provided $n \leq d$, represented by an open subset of $\Gr(n+1,d+1)$. 
\end{remark}




When $n \mid d$, and $\ell := d+d/n-2$, we get the morphism of stacks: 
\begin{align*}
  D^{n}_{d} &: U^{n}_{d} \to M^{n-1}_{\ell}\\
  \gamma &\longmapsto D(\gamma)
\end{align*}
which we call the {\sl differential construction}. Interestingly, the dimensions of the domain and codomain of the differential construction are equal, and this leads to another collection of enumerative problems: 
\begin{problem}\label{problem:differential}
   Compute the degrees of the differential constructions $D^{n}_{d} : U^{n}_{d} \to M^{n-1}_{\ell}$. 
 \end{problem} 

\begin{remark}
  The maps $D^{n}_{d}$ are clearly $PGL_{2}(k)$ equivariant.  The image of the differential construction $D^{n}_{d}$ need not be the open set $U^{n-1}_{\ell}$. \todo{Sure?}
\end{remark}


 The $n=d$ instances of \autoref{problem:differential} are immediate: 

 \begin{proposition}\label{proposition:trivialdegree}
   The degree of the differential construction $D^{d}_{d}$ is $1$.
 \end{proposition}
\begin{proof}
  The space $U^{d}_{d}$ is a single $PGL_{2}(k)$ orbit.
\end{proof}
 
\begin{definition}\label{definition:pointlinescroll}
  Let $\gamma: \P^{1} \to \P^{n}$ be any map. We define the {\sl point-hyperplane scroll} of $\gamma$ to be 
  \begin{align*}
        X_{\gamma} := \big\{(t, \Lambda) \mid \gamma(t) \in \Lambda \big\} \subset \P^{1} \times (\P^{n})^{\smvee}
      \end{align*}
      We denote by $\pi_{1}, \pi_{2}$ the projections of $X_{\gamma}$ to $\P^{1}$ and $(\P^{n})^{\smvee}$ respectively. Finally, we set $X_{\gamma}^{\smvee} := \P(\gamma^{*}T_{\P^{n}})$.   
\end{definition}

\begin{remark}
  The $\P^{n-1}$-bundle $X_{\gamma}$ is isomorphic to $\P(\gamma^{*}T^{\smvee}_{\P^{n}})$. Hence, for a general map $\gamma: \P^{1} \to \P^{n}$, $X_{\gamma}$ and $X_{\gamma}^{\smvee}$ are balanced scrolls.
\end{remark}

\begin{proposition}\label{proposition:transfer}
 Let $\gamma: \P^{1} \to \P^{n}$ be a non constant map.

 \begin{enumerate} 
  \item The image of $\gamma: \P^{1} \to \P^{n}$ is non-degenerate if and only if $\pi_{2} : X_{\gamma} \to (\P^{n})^{\smvee}$ is finite; in any case,  $\deg \pi_{2} = \deg \gamma.$ 
  \item The ramification divisor $R(\pi_{2}) \subset X_{\gamma}$ is a smooth, codimension $1$ subscroll of $X_{\gamma}$ if and only if $\gamma$ is an immersion. 
  \item Assuming $\gamma$ is an immersion, the dual section $R^{\smvee}(\pi_{2}) \subset X^{\smvee}_{\gamma}$ is induced by the inclusion $d \gamma: T_{\P^{1}} \hookrightarrow \gamma^{*}T_{\P^{n}}$.
\end{enumerate}
\end{proposition}

\begin{proof}
  \todo{PROVE}
\end{proof}

Let $X = \P^{1} \times \P^{n-1}$, and denote by $h$ and $f$ the divisor classes of the pullback of a hyperplane in $\P^{n-1}$ and a point in $\P^{1}$, respectively.  When $n \mid k$,  \autoref{proposition:transfer} sets up a commuting diagram: 

\begin{center}
\begin{tikzcd}
  & U^{n}_{k} \ar[rr, leftrightarrow, "\text{duality}"] \ar[dd, "D^{n}_{k}"] & &PGL_{n+1}\Big\backslash \left \{ \begin{tabular}{c} \text{Deg. $k$ maps} \\$X \to (\P^{n})^{\smvee}$ \\ \text{induced by $|h+\frac{k}{n}f|$} \end{tabular} \right \}\Big/PGL_{n} \ar[dd, "\rho_{X}"] \\
  &  &  & \\
  & M^{n-1}_{\ell}  \ar[rr, leftrightarrow, "\text{duality}"]  && \left \{ \begin{tabular}{c} \text{Smooth divisors $R \subset X$} \\\text{with div. class $|h + \ell f|$} \end{tabular} \right \} \big/ PGL_{n}
\end{tikzcd}
\end{center}

From this, we conclude: 
\begin{proposition}\label{proposition:equivalence}
  Let $k = nm$, and let $X \subset \P^{n(m+1)-1}$ be the variety $\P^{1} \times \P^{n-1}$ embedded by the linear series $|h+mf|$. Then 
  \begin{align*}
    \deg \rho_{X} = \deg D^{n}_{k}.
  \end{align*}
\end{proposition}

\begin{corollary}
  If $X \subset \P^{2n-1}$ is a Segre embedding of $\P^{1} \times \P^{n-1}$, then $\deg \rho_{X} = 1$.
\end{corollary} 

\begin{proof}
  The corollary follows at once from \autoref{proposition:equivalence} and \autoref{proposition:trivialdegree}.
\end{proof}

\subsection{Quartic surface scrolls}





\subsubsection{The explicit differential construction for trinodal quartics}

A trinodal quartic $R$ can be obtained as an abstract curve by identifying three pairs of points $\{a',a''\}, \{b',b''\}, \{c',c''\}$ on $\P^{1}$. These pairs can be encoded by the three binary quadratic forms (up to scale)  defining them.  In terms of these three quadratic forms, we will now describe the differential construction $D^{2}_{4}$. 


In what follows, we let $\{ q_{1},q_{2},q_{3} \}$ denote a point in $\Sym^{3} \P H^{0}(\O_{\P^{1}}(2))$.

\begin{definition}
  \label{definition:nodemap}
  Let
  \begin{align*}
    \nu: \Sym^{3}\P H^{0}(\O_{\P^{1}}(2)) \dashrightarrow \Gr(3,5)
  \end{align*}
  denote the map given by the formula:
  \begin{align*}
    \nu \left(\{q_{1},q_{2},q_{3}\}\right ) =    \left\{ \begin{tabular}{c}
      v. space of meromorphic $1$-forms $\omega$ on $\P^{1}$ with at worst \\
      simple poles at the zeros of $q_{i}$ and with {\sl opposite} residues\\
      at the pairs of zeros of $q_{i}$, for all $i = 1,2,3$
    \end{tabular}\right\}
  \end{align*}
 
\end{definition}



\begin{proposition}
  \label{proposition:symtwoptwo}
The map $\nu$ is birational.
\end{proposition}
\begin{proof}
  Suppose a general three dimensional space $W \subset H^{0}(\O_{\P^{1}}(4))$ is given. Then the induced degree four map $\P^{1} \to \P W^{\smvee}$ is the normalization of a trinodal quartic $R$. The vector space $W$ is naturally identified with the sections of the dualizing sheaf of $R$, which consist of meromorphic $1$-forms on $\P^{1}$ with the properties stated in the proposition.
\end{proof}

\begin{definition}
  \label{definition:pi}
  Let
  \begin{align*}
    \pi: \Sym^{3}\P^{2} \dashrightarrow \Gr(2,5)
  \end{align*}
  be given by the formula
  \begin{align*}
\pi\left (\{q_{1},q_{2},q_{3}\}\right ) =    \left\{ \begin{tabular}{l}
      v. space of meromorphic $1$-forms $\omega$ with at worst simple\\
      poles at the zeros of $q_{i}$ and with {\sl equal} residues\\
      at the pairs of zeros of $q_{i}$, for all $i = 1,2,3$
    \end{tabular}\right\}
  \end{align*}
 
\end{definition}

\begin{proposition}
  \label{proposition:reinterpretTangent}
  The rational map $\pi \circ \nu^{-1} : \Gr(3,5) \dashrightarrow \Gr(2,5)$ is the differential construction $D^{2}_{4}$.
\end{proposition}
\begin{proof}
  Let $\gamma : \P^{1} \to \P^{2}$ be a general map induced by a three dimensional vector space $W \subset H^{0}(\O_{\P^{1}}(4))$ having image $R$, and let $(q_{1},q_{2},q_{3})$ be $\nu^{-1}(\varphi)$. The pencil $D(\gamma)$ is cut out by the perspective conics. \todo{Why?} According to \autoref{theorem:perspectiveconics}, the linear series on $R$ cut out by perspective conics is $\O_{R}(1) \otimes \eta$, where $\eta$ is the distinguished element $(-1,-1,-1) \in \Pic(R)[2]$.
  If the space of sections of the line bundle $\O_{R}(1)$ is identified with $\nu(q_{1},q_{2},q_{3})$, then it follows that the space of sections of the twist $\O_{R}(1)\otimes \eta$ equals $\pi(q_{1},q_{2},q_{3})$. 
\end{proof}

\begin{definition}
  \label{definition:jacobiantwoquadrics}
  Let $\{a(x,y),b(x,y)\}$ be two homogeneous quadratic polynomials with no common zeros. Their {\sl Jacobian} is
  \begin{align*}
    J(a,b):= a_{x}b_{y}-a_{y}b_{x}.
  \end{align*}

\end{definition}

Note that the Jacobian vanishes precisely at the two branch points of the map $[x : y] \mapsto [a(x,y):b(x,y)]$.

\begin{theorem}
  \label{theorem:onlyapencil}
  Let $\left\{ q_{1},q_{2},q_{3} \right\} \in \Sym^{3}\P^{2}$ have six distinct roots. Then the vector space
  \begin{align*}
    \langle q_{1}J(q_{2},q_{3}), q_{2}J(q_{1},q_{3}), q_{3}J(q_{1},q_{2}) \rangle 
  \end{align*}
  is equal to  $\pi(q_{1},q_{2},q_{3}) \in \Gr(2,5)$.
\end{theorem}
\begin{proof}
  By $SL_2(k)$-equivariance, it suffices to prove the theorem for three quadratic functions $\left\{ xy, q_{2}, q_{3} \right\}$ where $q_{2}$ and $q_{3}$ are general.
 
  Let $\alpha_{1}, \alpha_{2}$, and $\beta_{1}, \beta_{2}$ denote the roots of $q_{2}, q_{3}$. Note that these roots are assumed to be in ${\bf A}^{1} \subset \P^{1}$.  We let $t = x/y$ denote the affine coordinate.

  The vector space $\Pi := \pi(t, q_{2}(t), q_{3}(t))$ is equal to the vector space of forms
  \begin{align*}
    \omega = \frac{f(t)dt}{tq_{2}(t)q_{3}(t)},
  \end{align*}
  with $\deg(f) \leq 4$, and with the additional constraints

 \begin{align*}
    \res_{\alpha_{1}}\omega = \res_{ \alpha_{2}} \omega\\
    \res_{\beta_{1}}\omega = \res_{ \beta_{2}} \omega\\
    \res_{0}\omega = \res_{\infty} \omega
  \end{align*}

  Since we know a priori that the space of such forms is two dimensional, we conclude in particular that there exists a nonzero $\omega \in \Pi$ which is nonzero and vanishing at $\alpha_{1}$.  However, the first residue condition then forces $\omega$ to vanish at $\alpha_{2}$ as well. (This is clear from the geometry: an element of the pencil of perspective conics is cut out by a (possibly singular) conic in $\P^{2}$. If it contains a node, then its pullback to $\P^1$ must vanish at both points above the node.)

  Therefore, there exists an $\omega \in \Pi$ of the form $$ \omega = \frac{(t-\alpha_{1})(t-\alpha_{2})g(t)dt}{tq_{2}q_{3}} = \frac{g(t)dt}{tq_{3}}.$$
  The residue conditions at $\beta_{i}$, and $0, \infty$ together imply, up to nonzero scaling,
  \begin{align*}
    g(t) = t^{2} - \beta_{1}\beta_{2}.
  \end{align*}
  The roots $\pm \sqrt{\beta_{1}\beta_{2}}$ are precisely the branch points of the map $[x:y] \to [xy : q_{3}]$.  Therefore $\omega$ vanishes at the roots of the quartic polynomial $q_{1}j(xy,q_{3})$. The theorem follows by arguing in the same manner for the two other pairs of roots.

\end{proof}

Given a general triple $\{a,b,c\}$ of binary quadratic forms, we can create the three quartic binary forms $a[b,c], b[c,a], c[a,b]$, where $[p,q]$ denotes $p_{x}q_{y} - p_{y}q_{x}$.  As we know, these three forms are actually linearly dependent, yielding a pencil of binary quartics. 

In this way, we obtain an {\sl a priori} rational map
\begin{align*}
 	D: \Hilb^{3}(\P^{2}) \dashrightarrow \Gr(2,5)
 \end{align*} 
 where the domain is the Hilbert scheme of $3$ points on $\P^{2}$. 

 The main observation is: 
 \begin{proposition}
 	\label{proposition:Dregular}
 	The rational map $D$ extends to a regular map.
 \end{proposition}

 \begin{proof}
 	This is best seen by describing $D$ geometrically, and noting that the geometric construction makes sense at every point of $H$.

 	If $\{a,b,c\}$ is a general subset of $\P^{2}$, then the quartic pencil $D(\{a,b,c\})$ is obtained as follows.  Recall that in $\P^{2}$ we have the canonical discriminant conic $C$ parametrizing square forms. A point $a \in \P^{2}$ defines a line $Pol(a) \subset \P^{2}$ spanned by the two points of $C$ which correspond to the roots of $a$. Furthermore, a pair of points $b,c \in \P^{2}$ defines the line $\overline{b,c} \subset \P^{2}$. 

 	To the triple $\{a,b,c\}$ we attach the triple of pairs of lines $Pol(a) \cup \overline{b,c}$ (and permutations), which cut the conic $C$ at $3$ members of a degree $4$ pencil. 

 	This geometric construction works even for non-reduced schemes.  For example, if $Z \subset \P^{2}$ is a fat point concentrated at a point $a \in \P^{2}$, we assign the degree $4$ pencil on $C$ as: The degree $2$ pencil corresponding to $Pol(a)$ with two base points at $Pol(a) \cap C$.
 \end{proof}

 The map $D$ is only generically finite; the locus of collinear triples is contracted, and has the same image as the locus of fat schemes.   However, it is easy to exhibit a point in $G$ over which there are exactly two preimages. 

 \begin{lemma}
 	\label{lemma:TwoPreimages}
 	Let $\Lambda \in \Gr(2,5)$ denote the unique pencil of binary quartics with simple base points at $0,1,\infty$ in $\P^{1}$. Then the preimage $D^{-1}(\Lambda)$ consists of two non-reduced points.
 \end{lemma}

 \begin{proof}
 	The two configurations are described as follows: View the three points $0,1,\infty$ on the diagonal conic $C$. Then the triple $\{0,1,\infty\}$ clearly maps to $\Lambda$, as does the triangle created by $Pol(0), Pol(1), Pol(\infty)$. 

 	A simple infinitesimal calculation shows any non-trivial first-order deformation of either of these configurations will have the effect of either removing the base-points, or moving their location. 

 	Furthermore, it is clear that these are the only two possible configurations giving rise to the pencil $\Lambda$. 
 \end{proof}

The previous lemma immediately gives:

\begin{theorem}
  	Let $X \subset \P^{5}$ be a balanced quartic surface scroll. Then $\deg \rho_{X} = 2$.
  \end{theorem}  


\subsection{Eccentric surface scrolls} % (fold)
\label{sub:surfaces}

% subsection surfaces (end)
Let $E = \O(1) \oplus \O(k+1)$, $X = \P E$. Choose an affine coordinate $t$ on $\P^{1}$, and consider the projection-ramification enumerative problem for $X \subset \P^{k+3}$.  We claim: 
\begin{proposition}\label{proposition:rhobirationalsurfaces}
Maintaining the setting above,	$\rho_{X}$ is birational.
\end{proposition}


Let $A= H^{0}(\O_{X}(1))$. This vector space will be identified with the space of expressions of the form $\ell(t)x_{1} + q_{k+1}(t)x_{2}$, where $\ell, q_{k+1}$ are polynomials of degrees at most $1$ and $k+1$ respectively. In what follows, subscripts of polynomials in $t$ represent the degree.

If $W \subset A$ is a general three dimensional vector space, then there will be a unique triple of elements in $W$ of the form
\begin{align*}
	w_{0} &= t(x_{1} + q_{k}(t)x_{2})\\
	w_{\infty} &= (x_{1} + r_{k}(t)x_{2})\\
	w_{*} &= s_{k+1}(t)x_{2}
\end{align*}

The Wronski determinant of this triple is: 

\begin{align}\label{equation:jacobsurface}
	sx_{1} + \left[s(qt)' - s'(qt) - t(r's-s'r)\right]x_{2}
\end{align}

\begin{proof}[Proof of \autoref{proposition:rhobirationalsurfaces}]
	Let $r := \sigma x_{1} + \tau x_{2} \in H^{0}(X,\O(R))$ be a general element, we can extract the unique vector space $W$ obeying $\rho_{X}(W) = [r] \in |R|$ as follows: First, we set $s := \sigma$. Secondly, given $s$, the equation $\left[s(qt)' - s'(qt) - t(r's-s'r)\right] = \tau$ is a system of $2k+2$ linear equations involving the $2k+2$ coefficients of the pair $(q,r)$. We know (from \autoref{theorem:Main})  this system has a finite, positive number of solutions. Hence it must have a unique solution, proving the proposition.
\end{proof}

\subsection{Eccentric threefold scrolls} % (fold)
\label{sub:eccentric_threefolds}
Now let $E = \O(1) \oplus \O(1) \oplus \O(k+1)$, $k \geq 0$, and set $X := \P E$.  Embed $X \subset \P^{k+5}$ via the natural $\O(1)$ on $X$. Again, we choose affine coordinate $t \in \P^{1}$ and relative coordinates $x_{1},x_{2},x_{3}$ on $X$ corresponding to the three factors of the splitting of $E$. 

\begin{proposition}\label{proposition:threefold}
Maintain the setting above. Then	$\rho_{X}$ is birational.
\end{proposition}
% subsection eccentric_threefolds (end)

Suppose $W \subset H^{0}(E)$ is a general $4$ dimensional vector space. Then the projection $W \to H^{0}(\O(1)\oplus \O(1))$ will be an isomorphism.  Hence, there will be $4$ uniquely defined elements of $W$ of the form: 
\begin{align*}
	x_{1} + ax_{3}\\
	x_{2} + bx_{3}\\
	tx_{1} + cx_{3}\\
	tx_{2} + d x_{3}
\end{align*}
where $a,b,c,d$ are degree $\leq k+1$ polynomials in $t$.
The Wronski determinant for this tuple of equations is: 
\begin{align}\label{eq:jacobianthreefold}
	\alpha x_{1} + \beta x_{2} + \gamma x_{3} = (d-bt)x_{1} + (at-c)x_{2} + \left[a't(bt-d) + b't(c-at) + c'(d-bt)+ d'(at-c) \right]x_{3}.
\end{align}

\begin{proof}[Proof of \autoref{proposition:threefold}]
	We replace the Grassmannian $\Gr(4,H^{0}(E))$ with the affine open subset $\A^{4k+8}$ parametrizing quadruples $(a,b,c,d)$. Then the ramification divisor equation \eqref{eq:jacobianthreefold} defines a map 
	\begin{align*}
	 	\rho^{*}: \A^{4k+8} \to \A^{4k+9}
	 \end{align*} 
	 where the latter $\A^{4k+9}$ is the vector space of triples $(\alpha,\beta,\gamma)$ with $\deg \alpha, \beta \leq k+2$ and $\deg \gamma \leq 2k+2$. The projection-ramification $\rho_{X}$ map $\rho$ is recovered by composing $\rho^{*}$ with the projection $\A^{4k+9} \dashrightarrow \P^{4k+8}$. 

	 First, if $(a,b,c,d)$ are general, then one can directly use the relative primeness of $d-bt$ and $at-c$ (we omit this simple calculation) to conclude that $\rho^{*}$ is generically injective on tangent spaces, and hence the generic fiber of $\rho^{*}$ is finite.


	 We next show  $\rho_{X}$ is dominant. In light of the previous paragraph, it suffices to prove: If $(\alpha, \beta, \gamma)$ is a general point in the image of $\rho^{*}$, and $\lambda \neq 0,1$ is a constant, then $\lambda(\alpha, \beta, \gamma)$ is not in the image of $\rho^{*}$. 

	 To this end, suppose $(a,b,c,d)$ is a general point in $\A^{4k+8}$. Then  $\alpha := d-bt$ and $\beta := at-c$ will be degree $k+2$ polynomials which are relatively prime.  


	 For any polynomial $p(t)$, let $p^{+}$ denote the highest degree coefficient of $p$. Observe that $\beta^{+} = a^{+}$.  Furthermore, the expression for $\gamma$ is easily seen to be 
	 \begin{align}\label{gammaEq}
	  	\gamma = (\alpha'\beta - \beta' \alpha) + \alpha a + \beta b
	  \end{align} 
	  where $'$ denotes $d/dt$. 

	  If we scale by $\lambda$, we get: 
	  \begin{align}
	      \label{firstEquations}
	  	\lambda \alpha &= \lambda (d-bt)\\
	  	\lambda \beta &= \lambda (at-c) \nonumber\\
	  	\lambda \gamma &= \lambda(\alpha'\beta - \beta' \alpha) + \lambda \alpha a + \lambda \beta b \nonumber
	  \end{align}

	  At the same time, if $\lambda(\alpha, \beta, \gamma)$ is also realized by some quadruple $(\tilde{a}, \tilde{b}, \tilde{c}, \tilde{d})$ then we get the equations: 
	  \begin{align}\label{secondEquation}
	  	\lambda \alpha &= \tilde{d} - \tilde{b}t\\
	  	\lambda \beta &= \tilde{a}t - \tilde{c} \nonumber\\
	  	\lambda \gamma &= \lambda^{2}(\alpha'\beta - \beta' \alpha) + \lambda \alpha \tilde{a} + \lambda \beta \tilde{b}\nonumber
	  \end{align}
	  The second equation gives $\tilde{a}^{+} = \lambda \beta^{+}$.  The last equation gives: $\gamma = \lambda(\alpha'\beta - \beta' \alpha) + \alpha \tilde{a} + \beta \tilde{b}$.  Combining with \eqref{gammaEq}, we get 
	  \begin{align*}\label{alphasbetas}
	    	\alpha (a - \beta') + \beta (b + \alpha') &= \alpha(\tilde{a} - \lambda\beta') + \beta(\tilde{b} + \lambda \alpha').
	    \end{align*} 
	    Since $\alpha$ and $\beta$ are relatively prime and have degree greater than $a,b,\tilde{a},\tilde{b}$, we deduce:
	    \begin{align*}
	     	a-\beta' &= \tilde{a} - \lambda \beta'\\
	     	b+\alpha' &= \tilde{b} + \lambda \alpha'
	     \end{align*} 
	     By examining top coefficients, and using $a^{+} = \beta^{+}$, $\tilde{a}^{+} = \lambda \beta^{+}$ we get: 
	     \begin{align*}
	     	\beta^{+} - (k+2)\beta^{+} &= \lambda\beta^{+} - \lambda(k+2)\beta^{+}
	     \end{align*}
	     or 
	     \begin{align*}
	     	(1-\lambda)\beta^{+} &= (1-\lambda)(k+2)\beta^{+}
	     \end{align*}
	     Given our assumption on $\lambda$, this is only possible if $\beta^{+} = 0$.  However, since $(a,b,c,d)$ were chosen generically, $\beta^{+} = a^{+}$ would not be zero, providing our desired contradiction.

	     Finally, we argue $\deg \rho_{X} = 1$. It suffices to show that a general ramification equation $\alpha x_{1} + \beta x_{2} + \gamma x_{3}$ of the form \eqref{eq:jacobianthreefold} arises from a unique choice of polynomials $(a,b,c,d)$.  The conditions $d-bt = \alpha$ and $at-c=\beta$ produce an affine linear subspace $\Lambda$ in the vector space of choices $(a,b,c,d)$. With respect to linear coordinates on $\Lambda,$ the expression for $\gamma$ is also linear, and hence the available choices of $(a,b,c,d)$ producing \autoref{eq:jacobianthreefold} is an intersection of affine linear spaces.  Since we already know $\rho^{*}$ is generically finite, it follows that $\deg \rho_{X} = 1$ as desired.

\end{proof} 

Since every smooth three dimensional rational normal scroll specializes isotrivially to the scroll $X$ in \autoref{proposition:threefold}, we immediately get: 

\begin{corollary}\label{corollary:maxVariation3Scrolls}
	The projection-ramification map $\rho_{X}$ is dominant for every smooth three dimensional rational normal scroll $X \subset \P^{n}$.
\end{corollary}

\subsection{Recasting the projection-ramification map for scrolls} Let $E$ be a rank $r$ ample vector bundle on $\P^{1}$, and set $X = \P E$.  Then a general $r+1$-dimensional subspace 
\begin{align*}
    W \subset H^{0}(X, \O(1)) = H^{0}(\P^{1},E)
\end{align*}
yields a short exact sequence 
\begin{align*}
    0 \to (\det E)^{-1} \to W \otimes \O_{\P^{1}} \to E \to 0
\end{align*}
which corresponds to an element $w$ (up to scalar) of the extension space $\Ext^{1}(E,(\det E)^{-1})$. The assignment $W \mapsto [w] \in \P(\Ext^{1}(E,(\det E)^{-1}))$ is easily seen to be a birational map between $\G := \Gr(n+1,H^{0}(E))$ and $\P(\Ext^{1}(E,(\det E)^{-1}))$.

The ramification linear series $|R|$ is the projectivization of the vector space $V = H^{0}(E \otimes \det E \otimes \Omega_{\P^{1}})$.  By Serre duality, $V$ is dual to $\Ext^{1}(E,(\det E)^{-1})$.  Therefore, the projection-ramification map $\rho_{X}$ may be recast as a map
\begin{align*}
    \delta_{X}: \P(V^{*}) \dashrightarrow \P(V)
\end{align*}



\section{Further Questions}

\begin{enumerate}
    \item How many of our theorems are valid in characteristic $p > 0$?  
    \item When $\dm \Gr < \dm |R|$ and $\rho_{X}$ is generically finite onto its image, then is $\rho_X$ birational onto its image?
    \item Is $\Gr(2,4)$ the only incompressible Grassmannian?
    \item Is it possible to classify the scrolls for which $\deg \rho_{X} = 1$?
    \item Is there an analogous characterization of varieties of minimal degree using ``higher codimension" ramification loci?
\end{enumerate}









 \bibliographystyle{amsalpha}
   \bibliography{CommonMath}

\end{document}
